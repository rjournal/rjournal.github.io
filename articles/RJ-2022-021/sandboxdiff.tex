% !TeX root = RJwrapper.tex
\title{Advancing \DIFdelbegin \DIFdel{reproducible research }\DIFdelend \DIFaddbegin \DIFadd{Reproducible Research }\DIFaddend by \DIFdelbegin \DIFdel{publishing }\DIFdelend \DIFaddbegin \DIFadd{Publishing }\DIFaddend R \DIFdelbegin \DIFdel{markdown notebooks }\DIFdelend \DIFaddbegin \DIFadd{Markdown Notebooks }\DIFaddend as
\DIFdelbegin \DIFdel{interactive sandboxes using }\DIFdelend \DIFaddbegin \DIFadd{Interactive Sandboxes Using }\DIFaddend the \DIFdelbegin \DIFdel{`}\DIFdelend learnr \DIFdelbegin \DIFdel{' package}\DIFdelend \DIFaddbegin \DIFadd{Package}\DIFaddend }
\author{by Chak Hau Michael Tso, Michael Hollaway, Rebecca Killick, Peter Henrys, Don Monteith, John Watkins, and Gordon Blair}

\maketitle

\DIFdelbegin %DIFDELCMD < \abstract{%
%DIFDELCMD < Various R packages and best practices have played a pivotal role to
%DIFDELCMD < promote the FAIR principles of open science. For example, (1)
%DIFDELCMD < well-documented R scripts and notebooks with rich narratives are
%DIFDELCMD < deposited at a trusted data centre, (2) R markdown interactive notebooks
%DIFDELCMD < can be run on-demand as a web service, and (3) R Shiny web apps provide
%DIFDELCMD < nice user interfaces to explore research outputs. However, notebooks
%DIFDELCMD < require users to go through the entire analysis and can easily break it,
%DIFDELCMD < while Shiny apps do not expose the underlying code and require extra
%DIFDELCMD < work for UI design. We propose using the \emph{learnr} package to expose
%DIFDELCMD < certain code chunks in R markdown so that users can readily experiment
%DIFDELCMD < with them in guided, isolated and resettable code sandboxes. Our
%DIFDELCMD < approach does not replace the existing use of notebooks and Shiny apps,
%DIFDELCMD < but it adds another level of abstraction between them to promote
%DIFDELCMD < reproducible science.
%DIFDELCMD < }
%DIFDELCMD < %%%
\DIFdelend \DIFaddbegin \abstract{%
Various R packages and best practices have played a pivotal role to
promote the Findability, Accessibility, Interoperability, and Reuse
(FAIR) principles of open science. For example, (1) well-documented R
scripts and notebooks with rich narratives are deposited at a trusted
data centre, (2) R Markdown interactive notebooks can be run on-demand
as a web service, and (3) R Shiny web apps provide nice user interfaces
to explore research outputs. However, notebooks require users to go
through the entire analysis, while Shiny apps do not expose the
underlying code and require extra work for UI design. We propose using
the `learnr' package to expose certain code chunks in R Markdown so that
users can readily experiment with them in guided, editable, isolated,
executable, and resettable code sandboxes. Our approach does not replace
the existing use of notebooks and Shiny apps, but it adds another level
of abstraction between them to promote reproducible science.
}
\DIFaddend 

\hypertarget{introduction}{%
\subsection{Introduction}\label{introduction}}

There has been considerable recognition to the need to promote open and
reproducible science in the past decade. The FAIR principles
\citep{Wilkinson2016a, Stall2019}
(\url{https://www.go-fair.org/fair-principles/}) of reproducble research
are now known to most scientists. While significant advances has been
made through the adoption of various best practices and policies
(e.g.~requirements from funders and publishers to archive data and
source code, metadata standards), there remains considerable barriers to
further advance open science and meet reproducible science needs. One of
such issues the availability of various levels of abstraction of the
same underlying analysis and code base to collaborate and engage with
different stakeholders of diverse needs \citep{Blair2019, Hollaway2020}.
For complex analysis or analysis that utilize a more advanced computing
environment, it is essential to provide the capability to allow users to
interact with the analysis at a higher level.

Existing approach to reproducible research focuses on either documenting
an entire analysis or allows user-friendly interaction. Within the R
ecosystem, R \DIFdelbegin \DIFdel{script }\DIFdelend \DIFaddbegin \DIFadd{scripts }\DIFaddend and notebooks allow reserachers to work together
and others to view the entire workflow, while R Shiny apps \citep{shiny}
allows rapid showcase of methods and research outcomes to users with
less experience. R Shiny has been widely adopted to share research
output and engage stakeholders since its conception in 2013. A recent
review \citep{Kasprzak} shows that bioinformatics is the subject with
the most Shiny apps published in journals while earth and envrironmental
science ranks second. Shiny apps are especially helpful to create
reproducible analysis \citep[e.g.~examples in][]{Hollaway2020} and
explore different scenarios \citep[e.g.][]{Whateley2015, Mose2018}.
Finally, the interactivity of Shiny apps makes it a excellent tool for
teaching\citep[e.g.][]{Williams2017, adventr}. However, not all \DIFdelbegin \DIFdel{user
needs }\DIFdelend \DIFaddbegin \DIFadd{users
}\DIFaddend fit nicely into this dichotomy. Some users may only want to adopt a
small fraction of an analysis for their work, while others may simply
want to modify a few parts of the analysis in order to test alternative
hypothesis. Current use of notebooks do not seem to support such diverse
needs as notebook \DIFaddbegin \DIFadd{output }\DIFaddend \emph{elements} \DIFaddbegin \DIFadd{(e.g.~figures and tables) }\DIFaddend are
not easily reproducible. This issue essentially applies to all coding
languages.

One potential way to address the problem described above is to allow
users to experiment with the code in protected computing environment.
This is not limited to creating instances for users to re-run the entire
code. Rather, this can also be done by exposing specific parts of a
notebook as editable \DIFaddbegin \DIFadd{and executable }\DIFaddend code boxes, as seen in many
interactive tutorial webpages for various coding languages. Recenty,
while discussing next steps for fostering reproducible research in
artificial intelligence, \citet{Carter2019} lists creating a protected
computing environment (`data enclave' or `sandbox') for reviewers to log
in and explore as one of the solutions. In software engineering, a
sandbox is a testing environment that isolates untested code changes and
outright experimentation from the production environment or repository,
in the context of software development including Web development and
revision control. Making a sandbox environment available for users to
test and explore various changes to the code that leads to research
outputs is a great step to further open science. Current practice of
open science largely requires users to assemble the notebooks, scripts
and data files provided in their own computing environment, which
requires significant amount of time and effort. A sandbox environment
can greatly reduce such barriers and if such sandboxes are available as
a web service, users can explore and interact with the code that
generates the research outputs at the convience of their own web browser
on demand.

In this paper, we describe a rapid approach to create and publish
``\DIFdelbegin \DIFdel{sandbox'' }\DIFdelend \DIFaddbegin \DIFadd{interactive sandboxes'' R }\DIFaddend Shiny apps from R \DIFdelbegin \DIFdel{markdown }\DIFdelend \DIFaddbegin \DIFadd{Markdown }\DIFaddend documents using
the \DIFdelbegin \emph{\DIFdel{learnr}}
%DIFAUXCMD
\DIFdelend \DIFaddbegin \CRANpkg{learnr} \DIFaddend package, with the aim to bridge the gap between
typical R \DIFdelbegin \DIFdel{markdown
}\DIFdelend \DIFaddbegin \DIFadd{Markdown }\DIFaddend notebook and typical Shiny apps in terms of levels of
abstraction. \DIFaddbegin \DIFadd{While code and markdown documents gives full details of the
code, standard R Shiny apps has too much limitations on users to
interact with the code and users often cannot see the underlying code.
Our approach allows users to interact with selected parts of the code in
an isolated manner, by specifying certain code chunks in a R Markdown
document as executable code boxes.
}\DIFaddend 

\hypertarget{the-learnr-r-package}{%
\subsection{\texorpdfstring{The \emph{learnr} R
package}{The learnr R package}}\label{the-learnr-r-package}}

\DIFdelbegin \emph{\DIFdel{learnr}} %DIFAUXCMD
\DIFdelend \DIFaddbegin \CRANpkg{learnr} \DIFaddend \citep{learnr} is an R package developed by RStudio to
rapidly create interactive tutorials. It follows the general \emph{\DIFdelbegin \DIFdel{Rmarkdowon}\DIFdelend \DIFaddbegin \DIFadd{R
Markdown}\DIFaddend } (the file has .Rmd extensions\DIFaddbegin \DIFadd{,
}\url{https://rmarkdown.rstudio.com/index.html}\DIFaddend ) architecture and
essentially creates a pre-rendered Shiny document similar to the way
Shiny UI components can be added to any R \DIFdelbegin \DIFdel{markdown documents.
}\DIFdelend \DIFaddbegin \DIFadd{Markdown documents.
Pre-rendered Shiny documents
(}\url{https://rmarkdown.rstudio.com/authoring_shiny_prerendered.HTML}\DIFadd{)
is a key enabling technology for the }\CRANpkg{learnr} \DIFadd{package since it
allows users to specify the execution context in each code chunk of a R
Markdown document that is used to render a R Shiny web app. Its use
circumvents the need of a full document render for each end user browser
session so that this type of R Shiny apps can load quickly. }\DIFaddend To create a
\DIFdelbegin \emph{\DIFdel{learnr}} %DIFAUXCMD
\DIFdel{tutorial }\DIFdelend \DIFaddbegin \CRANpkg{learnr} \DIFadd{tutorial in RStudio after }\CRANpkg{learnr} \DIFadd{is
installed}\DIFaddend , the user chooses a \DIFdelbegin \emph{\DIFdel{learnr}} %DIFAUXCMD
\DIFdel{R markdown
template }\DIFdelend \DIFaddbegin \CRANpkg{learnr} \DIFadd{R Markdown template from
a list after clicking the `create new R Markdown document' button}\DIFaddend . This
template is not different from other .Rmd files, except it requires
additional chunk arguments to control the sandbox appearances. The two
main \DIFdelbegin \DIFdel{feature of the }\emph{\DIFdel{learnr}} %DIFAUXCMD
\DIFdelend \DIFaddbegin \DIFadd{features of the }\CRANpkg{learnr} \DIFaddend package are the ``exercise'' and
``quiz'' options. The former allow users to directly type in code,
execute it, and see its results to test their knowlege while the latter
allow other question types such as multiple choice. Both of these
options include auto-graders, hints, and instructor feedback options.
Additonal overall options include setting time limits and \DIFaddbegin \DIFadd{an }\DIFaddend option to
forbid users to skip sections. Like any Shiny apps, \DIFdelbegin \emph{\DIFdel{learnr}} %DIFAUXCMD
\DIFdel{sites }\DIFdelend \DIFaddbegin \CRANpkg{learnr}
\DIFadd{apps }\DIFaddend can be easily embedded to other \DIFdelbegin \DIFdel{sites}\DIFdelend \DIFaddbegin \DIFadd{webpages}\DIFaddend , as seen in this example
\citep{rmrwr}.

Although the \DIFdelbegin \emph{\DIFdel{learnr}} %DIFAUXCMD
\DIFdelend \DIFaddbegin \CRANpkg{learnr} \DIFaddend package has existed for a few years now,
it is relatively not well known \DIFdelbegin \DIFdel{and }\DIFdelend \DIFaddbegin \DIFadd{to scientists as a potential use of R
Shiny Apps and it }\DIFaddend has mostly been used for simple tutorial apps designed
for R beginners. We propose a novel application of the \DIFdelbegin \emph{\DIFdel{learnr}} %DIFAUXCMD
\DIFdelend \DIFaddbegin \CRANpkg{learnr}
\DIFaddend package to advance reproducible research, which we outline in the next
section.

\hypertarget{approach-using-learnr-for-reproducible-research-sandboxes}{%
\subsection{\texorpdfstring{Approach: Using \emph{learnr} for
reproducible research
`sandboxes'}{Approach: Using learnr for reproducible research `sandboxes'}}\label{approach-using-learnr-for-reproducible-research-sandboxes}}

\DIFaddbegin \CRANpkg{learnr} \DIFadd{allows users to create executable code boxes. }\DIFaddend Our
approach is to publish R notebooks and serve parts of the notebooks as
interactive sandboxes to allow users to re-create certain elements of a
published notebook containing research outputs. We \DIFdelbegin \DIFdel{disable }\DIFdelend \DIFaddbegin \DIFadd{do not use }\DIFaddend the
auto-graders and any quiz-like functionality of \DIFdelbegin \emph{\DIFdel{learnr}} %DIFAUXCMD
\DIFdelend \DIFaddbegin \CRANpkg{learnr} \DIFaddend while
keeping the sandboxes. Notebook authors can go through their notebook
and select the code chunks that they would allow users to experiment,
while the others are rendered as static code snippets.

Recognizing \DIFdelbegin \emph{\DIFdel{learnr}} %DIFAUXCMD
\DIFdelend \DIFaddbegin \CRANpkg{learnr} \DIFaddend documents are themselves \DIFdelbegin \emph{\DIFdel{shiny}} %DIFAUXCMD
\DIFdel{sites}\DIFdelend \DIFaddbegin \DIFadd{R Shiny web apps}\DIFaddend ,
our approach essentially allows the publication of notebooks in the form
of web apps. However, unlike a typical \DIFdelbegin \emph{\DIFdel{shiny}} %DIFAUXCMD
\DIFdel{site, they are
pre-rendered shiny documents, meaning the user }\DIFdelend \DIFaddbegin \DIFadd{R Shiny web app, users }\DIFaddend do not
need to prepare \DIFaddbegin \DIFadd{a }\DIFaddend seprate UI (i.e.~user interface) \DIFdelbegin \DIFdel{and Server elements}\DIFdelend \DIFaddbegin \DIFadd{layout}\DIFaddend . Advanced
users can modify the site appearance by supplying custom design in .css
files.

Here, we first show the skeleton of a R \DIFdelbegin \DIFdel{markdown }\DIFdelend \DIFaddbegin \DIFadd{Markdown }\DIFaddend (.Rmd) file for a
\DIFdelbegin \emph{\DIFdel{learnr}} %DIFAUXCMD
\DIFdel{document .
}%DIFDELCMD < 

%DIFDELCMD < \begin{Schunk}
%DIFDELCMD < \begin{figure}
%DIFDELCMD < 

%DIFDELCMD < {\centering \subfloat[A typical .Rmd file\label{fig:fig_skeleton-1}]{\includegraphics[width=.49\linewidth]{skeleton-md} }\subfloat[A sandbox app .Rmd file\label{fig:fig_skeleton-2}]{\includegraphics[width=.49\linewidth]{skeleton} }
%DIFDELCMD < 

%DIFDELCMD < }
%DIFDELCMD < 

%DIFDELCMD < %%%
%DIFDELCMD < \caption[A comparison of minimal examples of a typical .Rmd document and a .Rmd document for an interactive sandbox app]{%
{%DIFAUXCMD
\DIFdelFL{A comparison of minimal examples of a typical .Rmd document and a .Rmd document for an interactive sandbox app.}}%DIFAUXCMD
%DIFDELCMD < \label{fig:fig_skeleton}
%DIFDELCMD < \end{figure}
%DIFDELCMD < \end{Schunk}
%DIFDELCMD < 

%DIFDELCMD < %%%
\DIFdelend \DIFaddbegin \CRANpkg{learnr} \DIFadd{document (Figure 1). }\DIFaddend Notice that it is very similar to
a typical .Rmd file where there is a mixture of narratives written in
markdown and R code chunks, in addtional to a YAML header. However,
there are a couple of important exceptions, namely the use of the
\texttt{exercise} chunk \DIFdelbegin \DIFdel{options and }\DIFdelend \DIFaddbegin \DIFadd{option (i.e.~editable and executable code boxes)
and }\DIFaddend different output type in the YAML header.

\DIFaddbegin \begin{Schunk}
\begin{figure}

{\centering \subfloat[A typical .Rmd file\label{fig:fig_skeleton-1}]{\includegraphics[width=.49\linewidth]{skeleton-md} }\subfloat[A sandbox app .Rmd file\label{fig:fig_skeleton-2}]{\includegraphics[width=.49\linewidth]{skeleton} }

}

\caption[A comparison of minimal examples of a typical .Rmd document and a .Rmd document for an interactive sandbox app]{\DIFaddFL{A comparison of minimal examples of a typical .Rmd document and a .Rmd document for an interactive sandbox app.}}\label{fig:fig_skeleton}
\end{figure}
\end{Schunk}

\DIFaddend Next, we outline the steps an author needs to take to publish R
notebooks as interactive sandboxes:

\begin{enumerate}
\def\labelenumi{\arabic{enumi}.}
\tightlist
\item
  All research output is included in the form of a well-documented R
  \DIFdelbegin \DIFdel{markdown }\DIFdelend \DIFaddbegin \DIFadd{Markdown }\DIFaddend document
\item
  Open a new \DIFdelbegin \emph{\DIFdel{learnr}} %DIFAUXCMD
\DIFdel{R markdown }\DIFdelend \DIFaddbegin \CRANpkg{learnr} \DIFadd{R Markdown }\DIFaddend template. Copy the content of
  the original notebook
\item
  For the code chunks that you would like to become sandboxes, add
  \texttt{exercise=TRUE}. Make sure it has a unique chunk name. It may
  look something like this:
\end{enumerate}

\DIFdelbegin \emph{\DIFdel{```\{r fig2, warning=FALSE, exercise=TRUE,
exercise.lines=30,fig.fullwidth=TRUE\}}}
%DIFAUXCMD
\DIFdelend %DIF > \begin{tabular}{ll}
%DIF > A & B \\
%DIF > A & B \\
%DIF > \end{tabular}

\DIFaddbegin \code{```\{r fig2, warning=FALSE, exercise=TRUE, exercise.lines=30,fig.fullwidth=TRUE\}}

\DIFaddend \begin{enumerate}
\def\labelenumi{\arabic{enumi}.}
\setcounter{enumi}{3}
\tightlist
\item
  Before any interactive code chunks, call the first code chunk `setup'.
  This will pre-load everything that will be used later.
\item
  Check whether you would like to link any of the interactive code
  snippets (by default each of them are independent, and only depends on
  the output of the `setup' chunk) You may want to modify your code
  chunks accordingly.
\item
  Done! Knit the notebook to view outputs as an interactive webpage.
  Publish it just like a Shiny app.
\end{enumerate}

The entire process \DIFdelbegin \DIFdel{should take less than an hour }\DIFdelend \DIFaddbegin \DIFadd{took us a few hours of effort }\DIFaddend and can be incorporated
to the proofreading of a R \DIFdelbegin \DIFdel{markdown document. }\DIFdelend \DIFaddbegin \DIFadd{Markdown document. However, we note that as
in any preparation of research output or web development several
iterations are often needed and the time required increases accordingly
as the compmlexity of the analysis increases.
}\DIFaddend 

In our implementation in DataLabs
(\url{https://datalab.datalabs.ceh.ac.uk/}), the environment and folder
to create the research is made available to the Shiny \DIFdelbegin \DIFdel{site }\DIFdelend \DIFaddbegin \DIFadd{app }\DIFaddend in a read-only
fashion. Therefore, the authors do not have to worry about versions of
packages of data or a different software setup. Using DataLabs
straightforward visual tools to publish \DIFdelbegin \DIFdel{Shiny sites}\DIFdelend \DIFaddbegin \DIFadd{R Shiny apps}\DIFaddend , we can publish an
R \DIFdelbegin \DIFdel{markdown }\DIFdelend \DIFaddbegin \DIFadd{Markdown }\DIFaddend notebook with interactive code snippets to reproduce certain
parts of research readily in a few clicks.

\hypertarget{deployment}{%
\subsection{Deployment}\label{deployment}}

In general, \DIFdelbegin \emph{\DIFdel{learnr}} %DIFAUXCMD
\DIFdelend \DIFaddbegin \CRANpkg{learnr} \DIFaddend tutorial apps can be published the same way
as \DIFaddbegin \DIFadd{R }\DIFaddend Shiny web apps in Shiny servers, such as the ones provided by cloud
service providers or \url{https://shinyapps.io}. The \DIFdelbegin \emph{\DIFdel{learnr}}
%DIFAUXCMD
\DIFdelend \DIFaddbegin \CRANpkg{learnr}
\DIFaddend package vignettes provide additional help on deployment.

We also describe our deployment of these \DIFdelbegin \DIFdel{sites }\DIFdelend \DIFaddbegin \DIFadd{apps }\DIFaddend in DataLabs, a UK NERC
virtual research environment that is being developed. DataLabs is a
collobarative virtual research environment \citep{Hollaway2020}
\DIFaddbegin \DIFadd{(}\url{https://datalab.datalabs.ceh.ac.uk/}\DIFadd{) }\DIFaddend for environmental scientist
to work together where data, software, and methods are all centrally
located in projects. DataLabs provide a space for scientists from
different domains (data science, statisticians, environmental science
and computer science) to work together and draw on each other's
expertise. It includes an easy-to-use user interface where users can
publish \DIFdelbegin \DIFdel{Shiny sites }\DIFdelend \DIFaddbegin \DIFadd{R Shiny apps }\DIFaddend with a few clicks, and this applies to these
notebooks with interactive code chunks as well. Importantly, when
provisioning a instance of R Shiny, this is deployed in a Docker
container with read-only access to the project datastore being used for
analysis. This allows an unprecended level of transparency as parts of
the analysis are readily exposed for users to experiment from the exact
environments, datasets (can be large and includes many files), and
versions of software that created the analysis. The use of Docker
deployed onto a Kubernetes infrastructure allows strict limits to be
placed on what visitors can do through the use of resource constraints
and tools such as \DIFdelbegin \DIFdel{AppArmor }\DIFdelend \DIFaddbegin \CRANpkg{RAppArmor} \DIFaddend \citep{RAppArmor}. While access to
project files is read-only, some author discretion is still advised to
ensure that visitors should not be able to view or list any private code
or data. \DIFaddbegin \DIFadd{We also note that future relases of }\CRANpkg{learnr} \DIFadd{will
contain external exercise evaluators, so that the code sandboxes can be
executed by an independent engine (such as Google Cloud) and give the
benefit of not having to rely on }\CRANpkg{RAppArmor}\DIFadd{.
}\DIFaddend 

\hypertarget{example-gb-rainfall-paper}{%
\subsection{Example: GB rainfall
paper}\label{example-gb-rainfall-paper}}

To demonstrate our concept, we have turned a R \DIFdelbegin \DIFdel{markdown }\DIFdelend \DIFaddbegin \DIFadd{Markdown }\DIFaddend notebook for a
paper we \DIFdelbegin \DIFdel{are publishing (}\emph{\DIFdel{pending publication}}%DIFAUXCMD
\DIFdel{) into a }\emph{\DIFdel{learnr}} %DIFAUXCMD
\DIFdelend \DIFaddbegin \DIFadd{have submitted for publication into a }\CRANpkg{learnr} \DIFaddend site
(\url{https://cptecn-sandboxdemo.datalabs.ceh.ac.uk/}) using the
procedures described in the previous sections. The paper investigates
the effect of weather and rainfall types on rainfall chemistry in the
UK. As can be seen in Figure 2, the code chunks to generate certain
parts of the paper is exposed. But unlike a static notebook site, the
code chunk is not only available for copy and paste but allows users to
modify and run on-demand. This makes it very straightforward for user to
experiment with various changes of the original analysis, thereby
promoting transparancy and trust.

Since \DIFdelbegin \emph{\DIFdel{learnr}} %DIFAUXCMD
\DIFdel{apps are Shiny apps}\DIFdelend \DIFaddbegin \CRANpkg{learnr} \DIFadd{apps are R Markdown documents}\DIFaddend , Shiny UI elements
can be easily added. We repeat one of the examples by replacing the
interactive code box by a simple selector, with minimal modification of
the code itself. This approach to publish Shiny apps requires
significantly less work than typical \DIFdelbegin \DIFdel{Shiny }\DIFdelend \DIFaddbegin \DIFadd{R Shiny web }\DIFaddend apps since no UI design
is needed and researchers can rapidly turn an R \DIFdelbegin \DIFdel{markdown }\DIFdelend \DIFaddbegin \DIFadd{Markdown }\DIFaddend document to a \DIFdelbegin \DIFdel{Shiny }\DIFdelend \DIFaddbegin \DIFadd{R
Shiny web }\DIFaddend app. For some cases, the use of certain datasets may require a
license, as in this example. A pop-up box is shown when the site is
loaded and visitors are required to check the boxes to acknowledge the
use of the approapriate data licenses (an alternative is to require
users to register and load a token file) before they can proceed.

\begin{Schunk}
\begin{figure}
\includegraphics[width=\textwidth]{GB_notebook_screenshot} \caption[A screenshot of the GB rainfall interactive notebook site]{A screenshot of the GB rainfall interactive notebook site. The main feature is the code box. When the site loads, the code that generates published version of the figure is in the box and published version of the figure is below it. Users can make edits and re-run the code in the code box and the figure will update accordingly. Users can use the "Start Over" button to see the published version of the code at any point without refreshing the entire site.}\label{fig:fig2}
\end{figure}
\end{Schunk}

\hypertarget{evaluation}{%
\subsection{Evaluation}\label{evaluation}}

The main strength of our approach is that it \DIFdelbegin \DIFdel{fits }\DIFdelend \DIFaddbegin \DIFadd{fills }\DIFaddend nicely the gap of
existing approaches in terms of levels of abstraction. While code and
markdown documents gives full details of the code, standard \DIFaddbegin \DIFadd{R }\DIFaddend Shiny apps
has too much limitations on users to interact with the code (Figure 3)
and users often cannot see the underlying code. Recently, it has become
popular to publish `live' Jupyter notebooks on Binder and Google Colab.
While this is a great contribution to open science, users are still
required to go run and go through the entire notebook step-by-step and
it can be easy to break it if users change something in between. Our
approach allows users to interact with portions of the code in a guided
and isolated manner, without the need to understand all the other parts
of a notebook or the fear to break it (Table 1). We emphasize that R
scripts/notebooks and R Shiny apps work well for their intended uses,
but our approach adds an additional level of accessibility to users.

The openness and ease-to-access our approach provides can benefit many
different stakeholders (Table 2). Researchers can more rapidly reproduce
\emph{parts} of the analysis of their choice without studying the entire
notebook or installing software or downloading all the data. They can
quickly test alternative hypothesis and stimulate scientific
discussions. For funders, encouraging the use of this approach means
less time is needed for future projects to pick up results from previous
work. And since this is based on \DIFdelbegin \emph{\DIFdel{learnr}} %DIFAUXCMD
\DIFdelend \DIFaddbegin \CRANpkg{learnr} \DIFaddend which is originally
designed as a tutorial tool, this approach will no doubt speed up the
process to train other users to use similar methods. Overall, it
promotes open science and make a better value of public \DIFdelbegin \DIFdel{money}\DIFdelend \DIFaddbegin \DIFadd{funds}\DIFaddend .

An obvious limitation of our approach is that it does not work well for
ideal conditions where other R file formats are designed for. For
instance, R scripts and R notebooks are much better suited for more
complex analysis for users to adopt to their own problems. Meanwhile, R
Shiny \DIFaddbegin \DIFadd{web apps }\DIFaddend provides a much richer user experience and is most suited
when the exposed code is generally not useful to stakeholders.
Nevertheless, as discussed above, our approach is designed for users to
reproduce \emph{elements} of an analysis. The user should evaluate these
options carefully, paying special attention to the needs of intended
users.

\DIFdelbegin %DIFDELCMD < \begin{Schunk}
%DIFDELCMD < \begin{figure}
%DIFDELCMD < \includegraphics[width=\textwidth]{learnr_abstraction} %%%
%DIFDELCMD < \caption[The various levels of abstraction of various types of R documents]{%
{%DIFAUXCMD
\DIFdelFL{The various levels of abstraction of various types of R documents. Our approach fills nicely the gap between R markdown or Jupyter notebooks and R Shiny apps.}}%DIFAUXCMD
%DIFDELCMD < \label{fig:fig1}
%DIFDELCMD < \end{figure}
%DIFDELCMD < \end{Schunk}
%DIFDELCMD < 

%DIFDELCMD < \begin{table}
%DIFDELCMD <   %%%
%DIFDELCMD < \caption{%
{%DIFAUXCMD
\DIFdelFL{Advantages of the proposed approach over existing approaches}}
  %DIFAUXCMD
%DIFDELCMD < \label{tbl:table1}
%DIFDELCMD <   \includegraphics[width=\linewidth]{GB_notebook_table2}
%DIFDELCMD < \end{table}
%DIFDELCMD < 

%DIFDELCMD < \begin{table}
%DIFDELCMD <   %%%
%DIFDELCMD < \caption{%
{%DIFAUXCMD
\DIFdelFL{Advantages of the proposed approach to various stakeholders}}
  %DIFAUXCMD
%DIFDELCMD < \label{tbl:table2}
%DIFDELCMD <   \includegraphics[width=\linewidth]{GB_notebook_table1}
%DIFDELCMD < \end{table}
%DIFDELCMD < 

%DIFDELCMD < %%%
\DIFdelend Serving notebooks as a web service will inevitably face provenance
issues. It is surely beneficial if the author's institution can host
these interactive notebooks for a few years after its publication (and
that of its related publications). In the future, publishers and data
centres may consider providing services to provide longer term
provenance of serving these interactive notebooks online. \DIFaddbegin \DIFadd{As for any web
apps, funding for the computation servers can be a potential issue. This
work uses DataLabs computation time which is part of the UK research
funding that develops it. However, a more rigorous funding model may be
needed in the future to ensure provenance of these notebooks.
}\DIFaddend 

\DIFaddbegin \DIFadd{Our approach focuses on improving reproducibility by exposing parts of R
script for users to run them live on a R Shiny web app, leveraging the
option to render R Markdown documents as R Shiny web apps and the
}\CRANpkg{learnr} \DIFadd{package. It focuses on the R scripts and R Markdown
documents. Users, however, may want to improve reproducibility from the
opposite direction, namely to allow outputs from an R Shiny web app to
be reproducible outside of the Shiny context. For such a requirement, we
recommend the use of the }\CRANpkg{shinymeta} \DIFadd{\mbox{%DIFAUXCMD
\citep{shinymeta} }\hspace{0pt}%DIFAUXCMD
package,
which allows users to capture the underlying code of selected output
elements and allows users to download it as well as the underlying data
to re-create the output in their own R instance. The }\CRANpkg{shinymeta}
\DIFadd{approach can be more involved and requires more effort than
}\CRANpkg{learnr} \DIFadd{so we think it is more suitable for users that are
focusing their effort on the R Shiny app (particularly the UI). In
summary, these two approaches complements each other and we recommend
users to consider them to improve reproducibility of their work.
}

\begin{Schunk}
\begin{figure}
\includegraphics[width=\textwidth]{learnr_abstraction} \caption[The various levels of abstraction of various types of R documents]{\DIFaddFL{The various levels of abstraction of various types of R documents. Our approach fills nicely the gap between R Markdown or Jupyter notebooks and R Shiny apps.}}\label{fig:fig1}
\end{figure}
\end{Schunk}

\begin{longtable}[]{@{}ll@{}}
\caption{\DIFadd{Advantages of the proposed approach to various
stakeholders}}\tabularnewline
\toprule
\begin{minipage}[b]{0.42\columnwidth}\raggedright
\DIFadd{Stakeholders}\strut
\end{minipage} & \begin{minipage}[b]{0.43\columnwidth}\raggedright
\DIFadd{Advantages}\strut
\end{minipage}\tabularnewline
\midrule
\endfirsthead
\toprule
\begin{minipage}[b]{0.42\columnwidth}\raggedright
\DIFadd{Stakeholders}\strut
\end{minipage} & \begin{minipage}[b]{0.43\columnwidth}\raggedright
\DIFadd{Advantages}\strut
\end{minipage}\tabularnewline
\midrule
\endhead
\begin{minipage}[t]{0.42\columnwidth}\raggedright
\DIFadd{Authors}\strut
\end{minipage} & \begin{minipage}[t]{0.43\columnwidth}\raggedright
\DIFadd{• Very little extra work required in additional to writing R markdown
document.}\\
\DIFadd{• No experience to generate web interfaces required.}\\
\DIFadd{• Much greater impact in research output.}\\
\strut
\end{minipage}\tabularnewline
\begin{minipage}[t]{0.42\columnwidth}\raggedright
\DIFadd{Other researchers (those wanting to try or compare the method)}\strut
\end{minipage} & \begin{minipage}[t]{0.43\columnwidth}\raggedright
\DIFadd{• A much more enriched experience to try methods and data and to test
alternative hypothesis and scenarios.}\\
\DIFadd{• No need to download data and scripts/notebooks and install packages to
try a method.}\\
\DIFadd{• More efficient to learn the new method.}\\
\strut
\end{minipage}\tabularnewline
\begin{minipage}[t]{0.42\columnwidth}\raggedright
\DIFadd{Other researchers (those curious about the results)}\strut
\end{minipage} & \begin{minipage}[t]{0.43\columnwidth}\raggedright
\DIFadd{• Try running different scenarios quickly than the published ones
without the hassle of full knowledge of the code, downloading the code
and data, and setting up the software environment.}\\
\DIFadd{• Quickly reset to the published version of code snippet. No need to
worry about breaking the code.}\\
\strut
\end{minipage}\tabularnewline
\begin{minipage}[t]{0.42\columnwidth}\raggedright
\DIFadd{Data Centres}\strut
\end{minipage} & \begin{minipage}[t]{0.43\columnwidth}\raggedright
\DIFadd{• A new avenue to demonstrate impact to funders if end users try methods
or datasets hosted by them in sandboxes.}\\
\strut
\end{minipage}\tabularnewline
\begin{minipage}[t]{0.42\columnwidth}\raggedright
\DIFadd{Funders}\strut
\end{minipage} & \begin{minipage}[t]{0.43\columnwidth}\raggedright
\DIFadd{• Better value of investment if even small parts of a research is
readily reproducible.}\\
\DIFadd{• Time saving to fund related work that builds on research documented
this way.}\\
\strut
\end{minipage}\tabularnewline
\begin{minipage}[t]{0.42\columnwidth}\raggedright
\DIFadd{Wider research community and general public)}\strut
\end{minipage} & \begin{minipage}[t]{0.43\columnwidth}\raggedright
\DIFadd{• Promotes trust and confidence in research through transparency.}\\
\strut
\end{minipage}\tabularnewline
\bottomrule
\end{longtable}

\begin{longtable}[]{@{}lll@{}}
\caption{\DIFadd{Advantages of the proposed approach to existing
approaches}}\tabularnewline
\toprule
\begin{minipage}[b]{0.20\columnwidth}\raggedright
\DIFadd{Technology}\strut
\end{minipage} & \begin{minipage}[b]{0.20\columnwidth}\raggedright
\DIFadd{Potential Issues}\strut
\end{minipage} & \begin{minipage}[b]{0.32\columnwidth}\raggedright
\DIFadd{How our approach can help?}\strut
\end{minipage}\tabularnewline
\midrule
\endfirsthead
\toprule
\begin{minipage}[b]{0.20\columnwidth}\raggedright
\DIFadd{Technology}\strut
\end{minipage} & \begin{minipage}[b]{0.20\columnwidth}\raggedright
\DIFadd{Potential Issues}\strut
\end{minipage} & \begin{minipage}[b]{0.32\columnwidth}\raggedright
\DIFadd{How our approach can help?}\strut
\end{minipage}\tabularnewline
\midrule
\endhead
\begin{minipage}[t]{0.20\columnwidth}\raggedright
\DIFadd{R script}\strut
\end{minipage} & \begin{minipage}[t]{0.20\columnwidth}\raggedright
\DIFadd{• Limited narrative. Needs to run all the scripts.}\\
\strut
\end{minipage} & \begin{minipage}[t]{0.32\columnwidth}\raggedright
\DIFadd{• Much richer narrative and interactive experience.}\\
\strut
\end{minipage}\tabularnewline
\begin{minipage}[t]{0.20\columnwidth}\raggedright
\DIFadd{Static notebooks}\strut
\end{minipage} & \begin{minipage}[t]{0.20\columnwidth}\raggedright
\DIFadd{• Needs to download the code, data and package to try it out.}\\
\strut
\end{minipage} & \begin{minipage}[t]{0.32\columnwidth}\raggedright
\DIFadd{• Can instantly try out the code in a controlled manner, using the
published data/packages/software environment.}\\
\strut
\end{minipage}\tabularnewline
\begin{minipage}[t]{0.20\columnwidth}\raggedright
\DIFadd{Web apps ( e.g.~R Shiny)}\strut
\end{minipage} & \begin{minipage}[t]{0.20\columnwidth}\raggedright
\DIFadd{• While web apps helpful to some stakeholders, it can be too high-level
to some.}\\
\DIFadd{• Lots of extra to create web interface.}\\
\DIFadd{• Does not expose the code to generate results.}\\
\strut
\end{minipage} & \begin{minipage}[t]{0.32\columnwidth}\raggedright
\DIFadd{• Users can interact with the code within the code snippet sandboxes
themselves.}\\
\DIFadd{• The published version of the code is shown to users.}\\
\DIFadd{• Users can run the code snippets live.}\\
\strut
\end{minipage}\tabularnewline
\begin{minipage}[t]{0.20\columnwidth}\raggedright
\DIFadd{Binder/Google Colab}\strut
\end{minipage} & \begin{minipage}[t]{0.20\columnwidth}\raggedright
\DIFadd{• Users change the entire notebook.}\\
\DIFadd{• Users need to run all cells about the section they are interested.}\\
\strut
\end{minipage} & \begin{minipage}[t]{0.32\columnwidth}\raggedright
\DIFadd{• A much more enriched and guided experience.}\\
\DIFadd{• Users can choose to only run the sandboxes they are interested.}\\
\strut
\end{minipage}\tabularnewline
\bottomrule
\end{longtable}

\DIFaddend \hypertarget{summary-and-outlook}{%
\subsection{Summary and outlook}\label{summary-and-outlook}}

We have proposed and demonstrated a rapid approach to publish R \DIFdelbegin \DIFdel{markdown
}\DIFdelend \DIFaddbegin \DIFadd{Markdown
}\DIFaddend notebooks as interactive sandboxes to allow users to experiment with
changes with various elements of a research output. It provides an
additional level of abstraction for users to interact with research
outputs and the codes that generates down. Since it can be linked to the
environment and data that generated the published output and has
independent document object identifiers (DOI), it is a suitable
candidate to preserve research workflow while exposing parts of it to
allow rapid experimentation by users. Our work is a demonstration on how
we may publish a notebook from virtual research environments such as
DataLabs, with data, packages, and workflow pre-loaded in a coding
envrionment, accompanied by rich narratives. While this paper outlines
the approach using R, the same approach can benefit other coding
languages such as Python. \DIFdelbegin \DIFdel{This can be achieved either through the
}\emph{\DIFdel{reticulate}} %DIFAUXCMD
\DIFdel{(\mbox{%DIFAUXCMD
\citet{reticulate}}\hspace{0pt}%DIFAUXCMD
) R package, or writing a python
package similar to }\emph{\DIFdel{learnr}}%DIFAUXCMD
\DIFdelend \DIFaddbegin \DIFadd{In fact, this can already be achieved as
}\CRANpkg{learnr} \DIFadd{can run Python chunks (as well as other execution
engines }\CRANpkg{knitr} \DIFadd{supports such as SAS and mySQL) as long as the
users generate and host the document using R}\DIFaddend . This paper contributes to
the vision towards publishing interactive notebooks as standalone
research outputs and the advancement of open science practices.

\DIFdelbegin %DIFDELCMD < \hypertarget{data-availability}{%
%DIFDELCMD < \subsection{Data availability}\label{data-availability}}
%DIFDELCMD < %%%
\DIFdelend \DIFaddbegin \hypertarget{data-availability-and-acknowledgments}{%
\subsection{Data availability and
acknowledgments}\label{data-availability-and-acknowledgments}}
\DIFaddend 

The GB rainfall example notebook is accessible via this URL
(\url{https://cptecn-sandboxdemo.datalabs.ceh.ac.uk/}) and the R
\DIFdelbegin \DIFdel{markdown file is included as supplementary information}\DIFdelend \DIFaddbegin \DIFadd{Markdown file is depositied in the NERC Environmental Information Data
Centre (EIDC) \mbox{%DIFAUXCMD
\citep{EIDC}}\hspace{0pt}%DIFAUXCMD
}\DIFaddend . The DataLab code stack is available at
\url{https://github.com/NERC-CEH/datalab}. We thank the DataLabs
developers team (especially Iain Walmsley, UKCEH) for the assitance to
deploy interactive \DIFdelbegin \DIFdel{Rmd }\DIFdelend \DIFaddbegin \DIFadd{R Markdown }\DIFaddend documents on DataLabs\DIFaddbegin \DIFadd{. This work is
supported by NERC Grant NE/T006102/1, Methodologically Enhanced Virtual
Labs for Early Warning of Significant or Catastrophic Change in
Ecosystems: Changepoints for a Changing Planet, funded under the
Constructing a Digital Environment Strategic Priority Fund. Additional
support is provided by the UK Status, Change and Projections of the
Environment (UK-SCAPE) programme started in 2018 and is funded by the
Natural Environment Research Council (NERC) as National Capability
(award number NE/R016429/1). The initial development work of DataLabs
was supported by a NERC Capital bid as part of the Environmental Data
Services (EDS)}\DIFaddend .

\bibliography{sandbox.bib}

\address{%
Chak Hau Michael Tso\\
UK Centre for Ecology and Hydrology\\%
Lancaster Environment Centre\\ Lancaster LA1 4YQ, United Kingdom\\
%
%
\\\textit{ORCiD: \href{https://orcid.org/0000-0002-2415-0826}{0000-0002-2415-0826}}%
\\\href{mailto:mtso@ceh.ac.uk}{\nolinkurl{mtso@ceh.ac.uk}}
}

\address{%
Michael Hollaway\\
UK Centre for Ecology and Hydrology\\%
Lancaster Environment Centre\\ Lancaster LA1 4YQ, United Kingdom\\
%
%
\\\textit{ORCiD: \href{https://orcid.org/0000-0003-0386-2696}{0000-0003-0386-2696}}%
\\\href{mailto:mhollaway@ceh.ac.uk}{\nolinkurl{mhollaway@ceh.ac.uk}}
}

\address{%
Rebecca Killick\\
Department of Statistics, Lancaster University\\%
Flyde College\\ Lancaster LA1 4YF, United Kingdom\\
%
%
\\\textit{ORCiD: \href{https://orcid.org/0000-0003-0583-3960}{0000-0003-0583-3960}}%
\\\href{mailto:r.killick@lancaster.ac.uk}{\nolinkurl{r.killick@lancaster.ac.uk}}
}

\address{%
Peter Henrys\\
UK Centre for Ecology and Hydrology\\%
Lancaster Environment Centre\\ Lancaster LA1 4YQ, United Kingdom\\
%
%
\\\textit{ORCiD: \href{https://orcid.org/0000-0003-4758-1482}{0000-0003-4758-1482}}%
\\\href{mailto:pehn@ceh.ac.uk}{\nolinkurl{pehn@ceh.ac.uk}}
}

\address{%
Don Monteith\\
UK Centre for Ecology and Hydrology\\%
Lancaster Environment Centre\\ Lancaster LA1 4YQ, United Kingdom\\
%
%
%
\\\href{mailto:donm@ceh.ac.uk}{\nolinkurl{donm@ceh.ac.uk}}
}

\address{%
John Watkins\\
UK Centre for Ecology and Hydrology\\%
Lancaster Environment Centre\\ Lancaster LA1 4YQ, United Kingdom\\
%
%
\\\textit{ORCiD: \href{https://orcid.org/0000-0002-3518-8918}{0000-0002-3518-8918}}%
\\\href{mailto:jww@ceh.ac.uk}{\nolinkurl{jww@ceh.ac.uk}}
}

\address{%
Gordon Blair\\
School of Computing and Communications, Lancaster University\\%
InfoLab21\\ Lancaster LA1 4WA, United Kingdom\\
%
%
\\\textit{ORCiD: \href{https://orcid.org/0000-0001-6212-1906}{0000-0001-6212-1906}}%
\\\href{mailto:g.blair@lancaster.ac.uk}{\nolinkurl{g.blair@lancaster.ac.uk}}
}

