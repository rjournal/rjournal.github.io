\documentclass{standalone}
\usepackage{xcolor}
\usepackage{verbatim}
\usepackage[T1]{fontenc}
\usepackage{graphics}
\usepackage{hyperref}
\newcommand{\code}[1]{\texttt{#1}}
\newcommand{\R}{R}
\newcommand{\pkg}[1]{#1}
\newcommand{\CRANpkg}[1]{\pkg{#1}}%
\newcommand{\BIOpkg}[1]{\pkg{#1}}
\usepackage{amsmath,amssymb,array}
\usepackage{booktabs}
\usepackage{thumbpdf,lmodern}
\usepackage{framed}
\usepackage{multirow}
\usepackage{mathtools}
\usepackage{afterpage}
\usepackage{lscape}
\usepackage{graphicx}
\usepackage{subcaption}
\usepackage{setspace}
\usepackage{caption}
\usepackage{textcomp}
\usepackage{tikz}
\usepackage{lmodern}
\newcommand*{\QEDA}{${\blacksquare}$}
\newcommand*{\QEDB}{{$\square}$}
\newtheorem{mythm}{Theorem}
\usetikzlibrary{matrix,calc,shapes,arrows}

\begin{document}
\nopagecolor
	\tikzset{
		treenode/.style = {shape=circle,
						   draw,align=center,
						   top color=white,
						   inner sep=4ex},
		decision/.style = {treenode, inner sep=4pt},
		root/.style     = {treenode},
		env/.style      = {treenode},
		ginish/.style   = {root},
		dummy/.style    = {circle,draw}
	  }
\begin{tikzpicture}[-latex]
  \matrix (chart)
    [
      matrix of nodes,
      column sep      = 4em,
      row sep         = 4ex,
      column 1/.style = {nodes={decision}},
      column 2/.style = {nodes={env}}
    ]
    {
       |[decision]|$Y_1$   &|[decision]|$Y_7$   &                 &|[decision]|$Y_3$   &|[decision]|$Y_4$   \\
       |[decision]|$Y_5$   &|[decision]|$Y_6$   &|[decision]|$Y_9$&|[decision]|$Y_{11}$&|[decision]|$Y_{12}$\\
			 |[decision]|$Y_{10}$&|[decision]|$Y_{13}$&|[decision]|$Y_2$&                    &|[decision]|$Y_8$   \\
    };
		\draw
		 (chart-1-2) edge (chart-1-1); %1<-7
		\draw
		 (chart-2-1) edge (chart-1-1); %1<-5
		\draw[draw=red,fill=red]
		 (chart-2-4) edge (chart-1-4); %3<-11
		\draw[draw=red,fill=red]
	   (chart-1-4) edge (chart-1-5); %4<-3
		\draw[draw=red,fill=red]
		 ([xshift= 10pt]chart-3-1) edge ([xshift= 5pt]chart-2-1.south); %5<-10
		\draw[draw=red,fill=red]
		 (chart-2-3) edge (chart-2-2); %6<-9
		\draw
		 (chart-2-1) edge (chart-2-2); %6<-5
		\draw[draw=red,fill=red]
		 (chart-2-2) edge (chart-1-2); %7<-6
		\draw
		 (chart-2-5) edge (chart-3-5); %8<-12
		\draw[draw=red,fill=red]
		 (chart-1-2) edge (chart-2-3); %9<-7
		\draw[draw=red,fill=red]
		 ([xshift= -10pt]chart-2-1) edge ([xshift= -5pt]chart-3-1.north); %10<-5
		\draw[draw=red,fill=red]
		 ([yshift= 10pt]chart-2-4) edge ([yshift= 5pt]chart-2-5.west); %11<-12
		\draw[draw=red,fill=red]
		 ([yshift= -10pt]chart-2-5) edge ([yshift= -5pt]chart-2-4.east); %12<-11
		\draw[draw=red,fill=red]
		 (chart-1-5) edge (chart-2-5); %12<-4
		\draw
		 (chart-3-3) edge (chart-3-2); %13<-2
		\draw
		 (chart-2-2) edge (chart-3-2); %13<-6
  \end{tikzpicture}
\end{document}
