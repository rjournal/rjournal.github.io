\documentclass{standalone}
\usepackage{xcolor}
\usepackage{verbatim}
\usepackage[T1]{fontenc}
\usepackage{hyperref}
\newcommand{\code}[1]{\texttt{#1}}
\newcommand{\R}{R}
\newcommand{\pkg}[1]{#1}
\newcommand{\CRANpkg}[1]{\pkg{#1}}%
\newcommand{\BIOpkg}[1]{\pkg{#1}}
\usepackage{amsmath,amssymb,array}
\usepackage{booktabs}
\usepackage{listings}
\usepackage{algorithm2e}
\usepackage{bm}
\usepackage{doi}

\begin{document}
\nopagecolor
\begin{algorithm}[H]
\SetAlgoLined\KwData{1) a matrix where the pathways are in the columns and the genes inside the pathways are in the rows; 2) a data frame where the nodes are presented in the columns and the rows represent the edges \\}
\KwResult{ a list of genes with high degree centrality for each pathway  }
Being ($i \in N$) \& ($i \in P$) where $P$ is a vector containing the genes inside a pathway  of size $k$ and
     $N$ is an indirect graph of size $m$\;
     
\For{ all nodes i in N}
{ calculate the degree centrality $d$\textsubscript{$iN$}\;
}
\For{ all nodes i in P}
{ calculate the degree centrality $d$\textsubscript{$iP$}, being the neighbors of $i, $ i$\textsubscript{$ng$}  \in P$\;
}
Calculate degree centrality expected $d$\textsubscript{$iE$} in $P$ \\
\eIf{ $d$\textsubscript{$iE$} < $d$\textsubscript{$iP$}/ $k$\textsubscript{$p$} }{ $i \longleftarrow  $potential  gene drivers of $ P$\;}
{$i \longleftarrow i+1$\;}
\end{algorithm}
\end{document}
