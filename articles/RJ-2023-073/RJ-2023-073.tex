% !TeX root = RJwrapper.tex
\title{mathml: Translate R Expressions to MathML and LaTeX}


\author{by Matthias Gondan and Irene Alfarone}

\maketitle

\abstract{%
This R~package translates R~objects to suitable elements in MathML or LaTeX, thereby allowing for a pretty mathematical representation of R~objects and functions in data analyses, scientific reports and interactive web content. In the R~Markdown document rendering language, R~code and mathematical content already exist side-by-side. The present package enables use of the same R~objects for both data analysis and typesetting in documents or web content. This tightens the link between the statistical analysis and its verbal description or symbolic representation, which is another step towards reproducible science. User-defined hooks enable extension of the package by mapping specific variables or functions to new MathML and LaTeX entities. Throughout the paper, examples are given for the functions of the package, and a case study illustrates its use in a scientific report.
}

\hypertarget{introduction}{%
\section{Introduction}\label{introduction}}

The R~extension of the markdown language (Xie, Allaire, and Grolemund 2018; Allaire et al. 2023)
enables reproducible statistical reports with
nice typesetting in HTML, Microsoft Word, and LaTeX. Moreover, since
recently (R Core Team 2022, version 4.2), Rs manual pages include
support for mathematical expressions (Sarkar and Hornik 2022; Viechtbauer 2022),
which is already a big improvement. However, except
for special cases such as regression models (Anderson, Heiss, and Sumners 2023)
and R's own plotmath annotation, rules for the mapping of
built-in language elements to their mathematical representation are
still lacking. So far, R~expressions such as \texttt{pbinom(k,~N,~p)} are
printed as they are and pretty mathematical formulae such as
\(P_{\mathrm{Bi}}(X \le k; N, p)\) require explicit LaTeX commands like
\texttt{P\_\{\textbackslash{}mathrm\{Bi\}\}\textbackslash{}left(X~\textbackslash{}le~k;~N,~p\textbackslash{}right)}. Except for very basic use
cases, these commands are tedious to type and their source code is hard
to read.

The present R~package defines a set of rules for the automatic
translation of R~expressions to mathematical output in R~Markdown
documents (Xie, Dervieux, and Riederer 2020) and Shiny Apps (Chang et al. 2022).
The translation is done by an embedded Prolog interpreter
that maps nested expressions recursively to MathML and LaTeX/MathJax,
respectively. User-defined hooks enable extension of the set of rules,
for example, to represent specific R~elements by custom mathematical
signs.

The main feature of the package is that the same R~expressions and
equations can be used for both mathematical typesetting and
calculations. This saves time and potentially reduces mistakes, as will
be illustrated below. Readers should have basic knowledge of knitr and
R~Markdown to be able to follow this article (Xie 2023; Allaire et al. 2023),
while to extend and customize the package, some basic knowledge
of Prolog is needed.

The paper is organized as follows. We start with a description of the
technical background of the package, including the two main classes of
rules for translating R~objects to mathematical expressions. The next
section illustrates the main features of the
\href{https://CRAN.R-project.org/package=mathml}{mathml} package, potential
issues and their workarounds using examples from the day-to-day
perspective of a user. A case study follows with a scientific report
written with the help of the package. The last section concludes with a
discussion and ideas for further development.

\hypertarget{background}{%
\section{Background}\label{background}}

Similar to other high-level programming languages, R is homoiconic, that
is, R~commands (i.e., R~``calls'') are, themselves, symbolic data
structures that can be created, parsed and modified. Because the default
response of the R~interpreter is to evaluate a call and return its
result, this property is not transparent to the general user. There
exists, however, a number of built-in R~functions (e.g., \texttt{quote()},
\texttt{call()} etc.) that allow the user to create R~calls which can be stored
in regular variables and then, for example, evaluated at a later stage
or in a specific environment (Wickham 2019). The present package
includes a set of rules that translate such calls to a mathematical
representation in MathML and LaTeX. For a first illustration of the
mathml package, we consider the binomial probability.

\begin{verbatim}
term <- quote(pbinom(k, N, p))
\end{verbatim}

The term is quoted to avoid its immediate evaluation (which would raise
an error anyway since the variables \texttt{k}, \texttt{N}, \texttt{p} have not yet been
defined). Experienced R~users will recognize that the expression is a
short form for

\begin{verbatim}
term <- call("pbinom", as.name("k"), as.name("N"), as.name("p"))
term

#> pbinom(k, N, p)
\end{verbatim}

As can be seen from the output, to the variable \texttt{term} is not assigned
the result of the calculation, but instead an R~call (see, e.g., Wickham 2019,
for details on ``non-standard evaluation''), which can eventually be
evaluated with \texttt{eval()},

\begin{verbatim}
k <- 10
N <- 22
p <- 0.4
eval(term)

#> [1] 0.77195
\end{verbatim}

The R~package mathml can now be used to render the call in MathML or in
MathJax/LaTeX. MathML is the dialect for mathematical elements on HTML
webpages, whereas LaTeX is typically used for typesetting printed
documents, as shown below.

\begin{verbatim}
library(mathml)
substr(mathml(term), 1, 70)

#> [1] "<math><mrow><msub><mi>P</mi><mtext>Bi</mtext></msub><mo>&af;</mo><mrow"

mathjax(term)

#> [1] "${P}_{\\mathrm{Bi}}{\\left({{X}{\\le}{k}}{{;}{{N}{{,}{p}}}}\\right)}$"
\end{verbatim}

Some of the curly braces are not really needed in the LaTeX output, but
are necessary in edge cases. The package also includes a function
\texttt{mathout()} that wraps a call to \texttt{mathml()} for HTML output and
\texttt{mathjax()} for LaTeX output. Moreover, the function \texttt{math(x)} adds the
class \texttt{"math"} to its argument, such that a special knitr printing
function is invoked (see the vignette on custom print methods in Xie 2023).
An R~Markdown code chunk with \texttt{mathout(term)} thus produces:

\({P}_{\mathrm{Bi}}{\left({{X}{\le}{k}}{{;}{{N}{{,}{p}}}}\right)}\)

Similarly, \texttt{inline()} produces inline output, \texttt{r~inline(term)} yields
\({P}_{\mathrm{Bi}}{\left({{X}{\le}{k}}{{;}{{N}{{,}{p}}}}\right)}\).

\hypertarget{package-in-practice}{%
\section{Package mathml in practice}\label{package-in-practice}}

The currently supported R~objects are listed below, roughly following
the order proposed by Murrell and Ihaka (2000).

\hypertarget{basic-elements}{%
\subsection{Basic elements}\label{basic-elements}}

mathml handles the basic elements of everyday mathematical expressions,
such as integers, floating-point numbers, Latin and Greek letters,
multi-letter identifiers, accents, subscripts, and superscripts.

\begin{verbatim}
term <- quote(1 + -2L + a + abc + "a" + phi + Phi + varphi + roof(b)[i, j]^2L)
math(term)
\end{verbatim}

\({{{{{{{{1.00}{+}{{-}{2}}}{+}{a}}{+}{abc}}{+}{\mathrm{a}}}{+}{\phi}}{+}{\Phi}}{+}{\varphi}}{+}{{\hat{b}}_{{i}{{\mathrm{}}{j}}}^{2}}\)

\begin{verbatim}
term <- quote(round(3.1415, 3L) + NaN + NA + TRUE + FALSE + Inf + (-Inf))
math(term)
\end{verbatim}

\({{{{{{3.142}{+}{nan}}{+}{na}}{+}{T}}{+}{F}}{+}{\infty}}{+}{\left({-}{\infty}\right)}\)

An expression such as \texttt{1~+~-2} may be considered aesthetically
unsatisfactory. It is correct R~syntax, though, and is reproduced
accordingly, without the parentheses. Parentheses around negative
numbers or symbols can be added as shown above for \texttt{+~(-Inf)}.

To avoid name clashes with package \texttt{stats}, \texttt{roof()} is used to put a
hat on a symbol (see next section for further decorations). Note that an
R~function \texttt{roof()} does not exist in base R, it is provided by the
package for convenience and points to the identity function.

\hypertarget{decorations}{%
\subsection{Decorations}\label{decorations}}

The package offers some support for different fonts as well as accents
and boxes etc. Internally, these decorations are implemented as identity
functions, so that they can be introduced into R expressions without
side-effects.

\begin{verbatim}
term <- quote(bold(b[x, 5L]) + bold(b[italic(x)]) + italic(ab) + italic(42L))
math(term)
\end{verbatim}

\({{{{\mathbf{b}}_{{\mathbf{x}}{{\mathrm{}}{5}}}}{+}{{\mathbf{b}}_{\mathit{x}}}}{+}{\mathit{ab}}}{+}{42}\)

\begin{verbatim}
term <- quote(tilde(a) + mean(X) + box(c) + cancel(d) + phantom(e) + prime(f))
math(term)
\end{verbatim}

\({{{{{\tilde{a}}{+}{\overline{X}}}{+}{\boxed{c}}}{+}{\cancel{d}}}{+}{\phantom{e}}}{+}{{f^\prime}}\)

Note that the font styles only affect the display of identifiers,
whereas numbers, character strings etc. are left untouched.

\hypertarget{operators-and-parentheses}{%
\subsection{Operators and parentheses}\label{operators-and-parentheses}}

Arithmetic operators and parentheses are translated as they are, as
illustrated below.

\begin{verbatim}
term <- quote(a - ((b + c)) - d*e + f*(g + h) + i/j + k^(l + m) + (n*o)^{p + q})
math(term)
\end{verbatim}

\({{{{{{a}{-}{\left[\left({b}{+}{c}\right)\right]}}{-}{{d}{{}}{e}}}{+}{{f}{\cdot}{\left({g}{+}{h}\right)}}}{+}{{i}{/}{j}}}{+}{{k}^{\left({l}{+}{m}\right)}}}{+}{{\left({n}{{}}{o}\right)}^{{p}{+}{q}}}\)

\begin{verbatim}
term <- quote(dot(a, b) + frac(1L, nodot(c, d + e)) + dfrac(1L, times(g, h)))
math(term)
\end{verbatim}

\({{{a}{\cdot}{b}}{+}{\frac{1}{{c}{{}}{\left({d}{+}{e}\right)}}}}{+}{\displaystyle{\frac{1}{{g}{\times}{h}}}}\)

For multiplications involving only numbers and symbols, the
multiplication sign is omitted. This heuristic does not always produce
the desired result; therefore, mathml defines alternative R~functions
\texttt{dot()}, \texttt{nodot()}, and \texttt{times()}. These functions calculate a product
and produce the respective multiplication signs. Similarly, \texttt{frac()} and
\texttt{dfrac()} can be used for small and large fractions.

For standard operators with known precedence, mathml is generally able
to detect if parentheses are needed; for example, parentheses are
automatically placed around \texttt{d~+~e} in the \texttt{nodot}-example. However, we
note unnecessary parentheses around \texttt{l~+~m} above. These parentheses are
a consequence of \texttt{quote(a\^{}(b~+~c))} actually producing a nested R~call
of the form \texttt{\^{}(a,~(b~+~c))} instead of \texttt{\^{}(a,~b~+~c)}:

\begin{verbatim}
term <- quote(a^(b + c))
paste(term)

#> [1] "^"       "a"       "(b + c)"
\end{verbatim}

For the present purpose, this R~feature is unfortunate because the extra
parentheses around \texttt{b~+~c} are not needed. The preferred result is
obtained by the functional form \texttt{quote(\^{}(k,~l~+~m))} of the power, or
curly braces as a workaround (see \texttt{p~+~q} above).

\hypertarget{custom-operators}{%
\subsection{Custom operators}\label{custom-operators}}

Whereas in standard infix operators, the parentheses typically follow
the rules for precedence, undesirable results may be obtained in custom
operators.

\begin{verbatim}
term <- quote(mean(X) %+-% 1.96 * s / sqrt(N))
math(term)
\end{verbatim}

\({{\left({\overline{X}}{\pm}{1.96}\right)}{\cdot}{s}}{/}{\sqrt{N}}\)

\begin{verbatim}
term <- quote('%+-%'(mean(X), 1.96 * s / sqrt(N))) # functional form of '%+-%'
term <- quote(mean(X) %+-% {1.96 * s / sqrt(N)})   # the same
math(term)
\end{verbatim}

\({\overline{X}}{\pm}{{{1.96}{{}}{s}}{/}{\sqrt{N}}}\)

The example is a reminder that it is not possible to define the
precedence of custom operators in R, and that expressions with such
operators are evaluated strictly from left to right. Again, the problem
can be worked around by the functional form of the operator or a curly
brace to hide the parenthesis, and, at the same time, enforce the
correct operator precedence.

More operators are shown in Table
\protect\hyperlink{tab:custom-operators}{1}, including the suggestions by Murrell
and Ihaka (2000) for graphical annotations and arrows in R~figures.

\hypertarget{tab:custom-operators}{}
\begin{longtable}[]{@{}llllll@{}}
\caption{Custom operators in mathml}\tabularnewline
\toprule\noalign{}
Operator & Output & Operator & Output & Operator & Arrow \\
\midrule\noalign{}
\endfirsthead
\toprule\noalign{}
Operator & Output & Operator & Output & Operator & Arrow \\
\midrule\noalign{}
\endhead
\bottomrule\noalign{}
\endlastfoot
A \%*\% B & \({A}{\times}{B}\) & A != B & \({A}{\ne}{B}\) & A \%\% B & \({A}{\leftrightarrow}{B}\) \\
A \%.\% B & \({A}{\cdot}{B}\) & A \textasciitilde{} B & \({A}{\sim}{B}\) & A \%-\textgreater\% B & \({A}{\rightarrow}{B}\) \\
A \%x\% B & \({A}{\otimes}{B}\) & A \%\textasciitilde\textasciitilde\% B & \({A}{\approx}{B}\) & A \%\textless-\% B & \({A}{\leftarrow}{B}\) \\
A \%/\% B & \(\lfloor{{A}{/}{B}}\rfloor\) & A \%==\% B & \({A}{\equiv}{B}\) & A \%up\% B & \({A}{\uparrow}{B}\) \\
A \%\% B & \(mod{\left({A}{{,}{B}}\right)}\) & A \%=\textasciitilde\% B & \({A}{\cong}{B}\) & A \%down\% B & \({A}{\downarrow}{B}\) \\
A \& B & \({A}{\land}{B}\) & A \%prop\% B & \({A}{\propto}{B}\) & A \%\textless=\textgreater\% B & \({A}{\iff}{B}\) \\
A \textbar{} B & \({A}{\lor}{B}\) & A \%in\% B & \({A}{\in}{B}\) & A \%=\textgreater\% B & \({A}{\Rightarrow}{B}\) \\
xor(A, B) & \({A}{\veebar}{B}\) & intersect(A, B) & \({A}{\cap}{B}\) & A \%\textless=\% B & \({A}{\Leftarrow}{B}\) \\
!A & \({\lnot}{A}\) & union(A, B) & \({A}{\cup}{B}\) & A \%dblup\% B & \({A}{\Uparrow}{B}\) \\
A == B & \({A}{=}{B}\) & crossprod(A, B) & \({{A}^{\mathrm{T}}}{\times}{B}\) & A \%dbldown\% B & \({A}{\Downarrow}{B}\) \\
A \textless- B & \({A}{=}{B}\) & is.null(A) & \({A}{=}{\emptyset}\) & & \(\mathrm{}\) \\
\end{longtable}

\hypertarget{tab:base-stats}{}
\begin{longtable}[]{@{}llll@{}}
\caption{R functions from base and stats}\tabularnewline
\toprule\noalign{}
Function & Output & Function & Output \\
\midrule\noalign{}
\endfirsthead
\toprule\noalign{}
Function & Output & Function & Output \\
\midrule\noalign{}
\endhead
\bottomrule\noalign{}
\endlastfoot
sin(x) & \(\sin{x}\) & dbinom(k, N, pi) & \({P}_{\mathrm{Bi}}{\left({{X}{=}{k}}{{;}{{N}{{,}{\pi}}}}\right)}\) \\
cosh(x) & \(\cosh{x}\) & pbinom(k, N, pi) & \({P}_{\mathrm{Bi}}{\left({{X}{\le}{k}}{{;}{{N}{{,}{\pi}}}}\right)}\) \\
tanpi(alpha) & \(\tan{\left({\alpha}{{}}{\pi}\right)}\) & qbinom(p, N, pi) & \({\arg\min}_{k}{\left[{{P}_{\mathrm{Bi}}{\left({{X}{\le}{k}}{{;}{{N}{{,}{\pi}}}}\right)}}{>}{p}\right]}\) \\
asinh(x) & \({\sinh}^{{-}{1}}{x}\) & dpois(k, lambda) & \({P}_{\mathrm{Po}}{\left({{X}{=}{k}}{{;}{\lambda}}\right)}\) \\
log(p) & \(\log{p}\) & ppois(k, lambda) & \({P}_{\mathrm{Po}}{\left({{X}{\le}{k}}{{;}{\lambda}}\right)}\) \\
log1p(x) & \(\log{\left({1}{+}{x}\right)}\) & qpois(p, lambda) & \({{\arg\max}}_{k}{\left[{{P}_{\mathrm{Po}}{\left({{X}{\le}{k}}{{;}{\lambda}}\right)}}{>}{p}\right]}\) \\
logb(x, e) & \({\log}_{e}{x}\) & dexp(x, lambda) & \({f}_{\mathrm{Exp}}{\left({x}{{;}{\lambda}}\right)}\) \\
exp(x) & \(\exp{x}\) & pexp(x, lambda) & \({F}_{\mathrm{Exp}}{\left({x}{{;}{\lambda}}\right)}\) \\
expm1(x) & \({\exp{x}}{-}{1}\) & qexp(p, lambda) & \({F}_{\mathrm{Exp}}^{{-}{1}}{\left({p}{{;}{\lambda}}\right)}\) \\
choose(n, k) & \(\binom{n}{k}\) & dnorm(x, mu, sigma) & \(\phi{\left({x}{{;}{{\mu}{{,}{\sigma}}}}\right)}\) \\
lchoose(n, k) & \(\log{\binom{n}{k}}\) & pnorm(x, mu, sigma) & \(\Phi{\left({x}{{;}{{\mu}{{,}{\sigma}}}}\right)}\) \\
factorial(n) & \({n}{!}\) & qnorm(alpha/2L) & \({\Phi}^{{-}{1}}{\left({\alpha}{/}{2}\right)}\) \\
lfactorial(n) & \(\log{{n}{!}}\) & 1L - pchisq(x, 1L) & \({1}{-}{{F}_{{\chi}^{2}{\left({1}{{\,}{\mathrm{df}}}\right)}}{\left(x\right)}}\) \\
sqrt(x) & \(\sqrt{x}\) & qchisq(1L - alpha, 1L) & \({F}_{{\chi}^{2}{\left({1}{{\,}{\mathrm{df}}}\right)}}^{{-}{1}}{\left({1}{-}{\alpha}\right)}\) \\
mean(X) & \(\overline{X}\) & pt(t, N - 1L) & \(P{\left({{T}{\le}{t}}{{;}{{{N}{-}{1}}{{\,}{\mathrm{df}}}}}\right)}\) \\
abs(x) & \({\left\vert{x}\right\vert}\) & qt(alpha/2L, N - 1L) & \({T}_{{\alpha}{/}{2}}{\left({{N}{-}{1}}{{\,}{\mathrm{df}}}\right)}\) \\
\end{longtable}

\hypertarget{builtin-functions}{%
\subsection{Builtin functions}\label{builtin-functions}}

There is support for most functions from package \texttt{base}, with adequate
use and omission of parentheses.

\begin{verbatim}
term <- quote(sin(x) + sin(x)^2L + cos(pi/2L) + tan(2L*pi) * expm1(x))
math(term)
\end{verbatim}

\({{{\sin{x}}{+}{{\left(\sin{x}\right)}^{2}}}{+}{\cos{\left({\pi}{/}{2}\right)}}}{+}{{\tan{\left({2}{{}}{\pi}\right)}}{\cdot}{\left({\exp{x}}{-}{1}\right)}}\)

\begin{verbatim}
term <- quote(choose(N, k) + abs(x) + sqrt(x) + floor(x) + exp(frac(x, y)))
math(term)
\end{verbatim}

\({{{{\binom{N}{k}}{+}{{\left\vert{x}\right\vert}}}{+}{\sqrt{x}}}{+}{\lfloor{x}\rfloor}}{+}{\exp{\left(\frac{x}{y}\right)}}\)

A few more examples are shown in Table
\protect\hyperlink{tab:base-stats}{2},
including functions from \texttt{stats}.

\hypertarget{custom-functions}{%
\subsection{Custom functions}\label{custom-functions}}

For self-written functions, the matter is somewhat more complicated. For
instance, if we consider a function such as \texttt{g~\textless{}-~function(...)~...},
the name \emph{g} is not transparent to R, because only the function body
is represented. We can still display functions in the form \texttt{head(x)~=~body}
if we embed the object to be shown into a call \texttt{"\textless{}-"(head,~body)}.

\begin{verbatim}
sgn <- function(x)
{
  if(x == 0L) return(0L)
  if(x < 0L) return(-1L)
  if(x > 0L) return(1L)
}

math(sgn)
\end{verbatim}

\(\left\{\begin{array}{l}{{0},\ \mathrm{if}\ {{x}{=}{0}}}\\ {{{-}{1}},\ \mathrm{if}\ {{x}{<}{0}}}\\ {{1},\ \mathrm{if}\ {{x}{>}{0}}}\end{array}\right.\)

\begin{verbatim}
math(call("<-", quote(sgn(x)), sgn))
\end{verbatim}

\({\mathrm{sgn}\,{x}}{=}{\left\{\begin{array}{l}{{0},\ \mathrm{if}\ {{x}{=}{0}}}\\ {{{-}{1}},\ \mathrm{if}\ {{x}{<}{0}}}\\ {{1},\ \mathrm{if}\ {{x}{>}{0}}}\end{array}\right.}\)

The function body is generally a nested R~call of the form \texttt{\{(L)}, with
\texttt{L} being a list of commands (the semicolon, not necessary in R, is
translated to a newline). The example also illustrates that mathml
provides limited support for control structures such as \texttt{if} that is
internally represented as \texttt{if(condition,~action)}.

\hypertarget{indices-and-powers}{%
\subsection{Indices and powers}\label{indices-and-powers}}

Indices in square brackets are rendered as subscripts, powers are
rendered as superscript. Moreover, mathml defines the functions
\texttt{sum\_over(x,~from,~to)}, and \texttt{prod\_over(x,~from,~to)} that simply return
their first argument. The other two arguments serve as decorations (\emph{to}
is optional), for example, for summation and product signs.

\begin{verbatim}
term <- quote(S[Y]^2L <- frac(1L, N) * sum(Y[i] - mean(Y))^2L)
math(term)
\end{verbatim}

\({{S}_{Y}^{2}}{=}{{\frac{1}{N}}{\cdot}{{\sum{\left({{Y}_{i}}{-}{\overline{Y}}\right)}}^{2}}}\)

\begin{verbatim}
term <- quote(log(prod_over(L[i], i==1L, N)) <- sum_over(log(L[i]), i==1L, N))
math(term)
\end{verbatim}

\({\log{{\prod}_{{i}{=}{1}}^{N}{{L}_{i}}}}{=}{{\sum}_{{i}{=}{1}}^{N}{\log{{L}_{i}}}}\)

\hypertarget{ringing-back-to-r}{%
\subsection{Ringing back to R}\label{ringing-back-to-r}}

Rs \texttt{integrate} function takes a number of arguments, the most important
ones being the function to integrate, and the lower and the upper bound
of the integration.

\begin{verbatim}
term <- quote(integrate(sin, 0L, 2L*pi))
math(term)
\end{verbatim}

\(\int_{0}^{{2}{{}}{\pi}}{\sin{x}}\,{d{x}}\)

\begin{verbatim}
eval(term)

#> 2.221482e-16 with absolute error < 4.4e-14
\end{verbatim}

For mathematical typesetting in the form of \(\int f(x)\, dx\), mathml
needs to find out the name of the integration variable. For that
purpose, the underlying Prolog bridge provides a predicate \texttt{r\_eval/2}
that calls R~from Prolog. This predicate is used to evaluate
\texttt{formalArgs(args(sin))} and returns the names of the arguments of
\texttt{sin()}, namely, \emph{x}.

Above, the quoted term is an abbreviation for \texttt{call("integrate",~quote(sin),~...)},
with \texttt{sin} being an R~symbol, not a function. While the R~function
\texttt{integrate()} can handle both symbols and functions, mathml needs the
symbol because it is unable to determine the function name of custom
functions.

\hypertarget{names-and-order-of-arguments}{%
\subsection{Names and order of arguments}\label{names-and-order-of-arguments}}

One of R's great features is the possibility to refer to function
arguments by their names, not only by their position in the list of
arguments. At the other end, the Prolog handlers for R~calls are rather
rigid, for example, \texttt{integrate/3} accepts exactly three arguments in a
particular order and without names, that is,
\texttt{integrate(lower=0L,~upper=2L*pi,~sin)}, would not print the desired
result.

To ``canonicalize'' function calls with named arguments and arguments in
unusual order, mathml provides an auxiliary R~function
\texttt{canonical(f,~drop)} that reorders the argument list of calls to known
R~functions and, if \texttt{drop=TRUE} (which is the default), also removes the
names of the arguments.

\begin{verbatim}
term <- quote(integrate(lower=0L, upper=2L*pi, sin))
canonical(term)

#> integrate(sin, 0L, 2L * pi)

math(canonical(term))
\end{verbatim}

\(\int_{0}^{{2}{{}}{\pi}}{\sin{x}}\,{d{x}}\)

This function can be used to feed mixtures of partially named and
positional arguments into the renderer. For details, see the R~function
\texttt{match.call()}.

\hypertarget{matrices-and-vectors}{%
\subsection{Matrices and Vectors}\label{matrices-and-vectors}}

Of course, mathml also supports matrices and vectors.

\begin{verbatim}
v <- 1:3
math(call("t", v))
\end{verbatim}

\({\left({1}{{\,}{2}{{\,}{3}}}\right)}^{\mathrm{T}}\)

\begin{verbatim}
A <- matrix(data=11:16, nrow=2, ncol=3)
B <- matrix(data=21:26, nrow=2, ncol=3)
term <- call("+", A, B)
math(term)
\end{verbatim}

\({\left(\begin{array}{ccc}11 & 13 & 15\\ 12 & 14 & 16\\ \end{array}\right)}{+}{\left(\begin{array}{ccc}21 & 23 & 25\\ 22 & 24 & 26\\ \end{array}\right)}\)

Note that the seemingly more convenient \texttt{term~\textless{}-~quote(A~+~B)} yields
\(A + B\) in the output---instead of the desired matrix representation.
This behavior is expected because quotation of R calls also quote the
components of the call (here, \emph{A} and \emph{B}).

\hypertarget{short-mathematical-names-for-r-symbols}{%
\subsection{Short mathematical names for R symbols}\label{short-mathematical-names-for-r-symbols}}

In typical R~functions, variable names are typically longer than just
single letters, which may yield unsatisfactory results in the
mathematical output.

\begin{verbatim}
term <- quote(pbinom(successes, Ntotal, prob))
math(term)
\end{verbatim}

\({P}_{\mathrm{Bi}}{\left({{X}{\le}{successes}}{{;}{{Ntotal}{{,}{prob}}}}\right)}\)

\begin{verbatim}
hook(successes, k)
hook(quote(Ntotal), quote(N), quote=FALSE)
hook(prob, pi)
math(term)
\end{verbatim}

\({P}_{\mathrm{Bi}}{\left({{X}{\le}{k}}{{;}{{N}{{,}{\pi}}}}\right)}\)

To improve the situation, mathml provides a simple hook that can be used
to replace elements (e.g., verbose variable names) of the code by
concise mathematical symbols, as illustrated in the example. To simplify
notation, \texttt{hook()} uses non-standard evaluation of its arguments. If the
\texttt{quote} flag of \texttt{hook()} is set to \texttt{FALSE}, the user has to provide the
quoted expressions. Care should be taken to avoid recursive hooks such
as \texttt{hook(s,~s{[}"A"{]})} that endlessly replace the \(s\) from
\(s_{\mathrm{A}}\) as in \(s_{\mathrm{A}_{\mathrm{A}_{\mathrm{A}\cdots}}}\).

The hooks can also be used for more complex elements such as R~calls,
with dotted symbols representing Prolog variables.

\begin{verbatim}
hook(pbinom(.K, .N, .P), sum_over(dbinom(i, .N, .P), i=0L, .K))
math(term)
\end{verbatim}

\({\sum}_{{i}{=}{0}}^{k}{{P}_{\mathrm{Bi}}{\left({{X}{=}{i}}{{;}{{N}{{,}{\pi}}}}\right)}}\)

Further customization requires the assertion of new Prolog rules
\texttt{math/2}, \texttt{ml/3}, \texttt{jax/3}, as shown in the Appendix.

\hypertarget{abbreviations}{%
\subsection{Abbreviations}\label{abbreviations}}

We consider the \(t\)-statistic for independent samples with equal
variance. To avoid clutter in the equation, the pooled variance
\(s^2_{\mathrm{pool}}\) is abbreviated, and a comment is given with the
expression for \(s^2_{\mathrm{pool}}\). For this purpose, mathml provides
a function \texttt{denote(abbr,~expr,~info)}, with \texttt{expr} actually being
evaluated, \texttt{abbr} being rendered, plus a comment of the form ``with
\texttt{expr} denoting \texttt{info}''.

\begin{verbatim}
hook(m_A, mean(X)["A"]) ; hook(s2_A, s["A"]^2L) ;
hook(n_A, n["A"])
hook(m_B, mean(X)["B"]) ; hook(s2_B, s["B"]^2L)
hook(n_B, n["B"]) ; hook(s2_p, s["pool"]^2L)

term <- quote(t <- dfrac(m_A - m_B, 
    sqrt(denote(s2_p, frac((n_A - 1L)*s2_A + (n_B - 1L)*s2_B, n_A + n_B - 2L),
                "the pooled variance.") * (frac(1L, n_A) + frac(1L, n_B)))))
math(term)
\end{verbatim}

\({t}{=}{\displaystyle{\frac{{{\overline{X}}_{\mathrm{A}}}{-}{{\overline{X}}_{\mathrm{B}}}}{\sqrt{{{s}_{\mathrm{pool}}^{2}}{\cdot}{\left({\frac{1}{{n}_{\mathrm{A}}}}{+}{\frac{1}{{n}_{\mathrm{B}}}}\right)}}}}}\),
with
\({{s}_{\mathrm{pool}}^{2}}{=}{\frac{{{\left({{n}_{\mathrm{A}}}{-}{1}\right)}{\cdot}{{s}_{\mathrm{A}}^{2}}}{+}{{\left({{n}_{\mathrm{B}}}{-}{1}\right)}{\cdot}{{s}_{\mathrm{B}}^{2}}}}{{{{n}_{\mathrm{A}}}{+}{{n}_{\mathrm{B}}}}{-}{2}}}\)
denoting the pooled variance.

The term is evaluated below. \texttt{print()} is needed because the return
value of an assignment of the form \texttt{t~\textless{}-~dfrac(...)} is not visible in
R.

\begin{verbatim}
m_A <- 1.5; s2_A <- 2.4^2; n_A <- 27; m_B <- 3.9; s2_B <- 2.8^2; n_B <- 20
print(eval(term))

#> [1] -3.157427
\end{verbatim}

\hypertarget{context-dependent-rendering}{%
\subsection{Context-dependent rendering}\label{context-dependent-rendering}}

Consider an educational scenario in which we want to highlight a certain
element of a term, for example, that a student has forgotten to subtract
the null hypothesis in a \(t\)-ratio:

\begin{verbatim}
t <- quote(dfrac(omit_right(mean(D) - mu[0L]), s / sqrt(N)))
math(t, flags=list(error="highlight"))
\end{verbatim}

\(\displaystyle{\frac{{\overline{D}}{{\,}{\cancel{{-}{{\,}{{\mu}_{0}}}}}}}{{s}{/}{\sqrt{N}}}}\)

\begin{verbatim}
math(t, flags=list(error="fix"))
\end{verbatim}

\(\displaystyle{\frac{{\overline{D}}{{\,}{\boxed{{-}{{\,}{{\mu}_{0}}}}}}}{{s}{/}{\sqrt{N}}}}\)

The R function \texttt{omit\_right(a~+~b)} uses non-standard evaluation
techniques (e.g., Wickham 2019) to return only the left part an
operation, and cancels the right part. This may not always be desired,
for example, when illustrating how to fix the mistake.

For this purpose, the functions \texttt{mathml()}, \texttt{mathjax()}, \texttt{mathout()} and
\texttt{math()} have an optional argument \texttt{flags} which is a list with named
elements. In this example, we use this argument to tell mathml how to
render such erroneous expressions using the flag \texttt{error} which can be
``asis'', ``highlight'', ``fix'', or ``ignore''. For more examples, see Table
\protect\hyperlink{tab:mistakes}{3}.

\hypertarget{tab:mistakes}{}
\begin{longtable}[]{@{}lllll@{}}
\caption{Highlighting elements of a term}\tabularnewline
\toprule\noalign{}
Operation & error~=~asis & highlight & fix & ignore \\
\midrule\noalign{}
\endfirsthead
\toprule\noalign{}
Operation & error~=~asis & highlight & fix & ignore \\
\midrule\noalign{}
\endhead
\bottomrule\noalign{}
\endlastfoot
omit\_left(a + b) & \(b\) & \({\cancel{{a}{{\,}{+}}}}{{\,}{b}}\) & \({\boxed{{a}{{\,}{+}}}}{{\,}{b}}\) & \({a}{+}{b}\) \\
omit\_right(a + b) & \(a\) & \({a}{{\,}{\cancel{{+}{{\,}{b}}}}}\) & \({a}{{\,}{\boxed{{+}{{\,}{b}}}}}\) & \({a}{+}{b}\) \\
list(quote(a), quote(omit(b))) & \({a}{{\,}{\mathrm{}}}\) & \({a}{{\,}{\cancel{b}}}\) & \({a}{{\,}{\boxed{b}}}\) & \({a}{{\,}{b}}\) \\
add\_left(a + b) & \({a}{+}{b}\) & \({\boxed{{a}{{\,}{+}}}}{{\,}{b}}\) & \({\cancel{{a}{{\,}{+}}}}{{\,}{b}}\) & \(b\) \\
add\_right(a + b) & \({a}{+}{b}\) & \({a}{{\,}{\boxed{{+}{{\,}{b}}}}}\) & \({a}{{\,}{\cancel{{+}{{\,}{b}}}}}\) & \(a\) \\
list(quote(a), quote(add(b))) & \({a}{{\,}{b}}\) & \({a}{{\,}{\boxed{b}}}\) & \({a}{{\,}{\cancel{b}}}\) & \({a}{{\,}{\mathrm{}}}\) \\
instead(a, b) + c & \({a}{+}{c}\) & \({\underbrace{a}_{{\mathrm{instead}}{{\,}{\mathrm{of}}{{\,}{b}}}}}{+}{c}\) & \({\boxed{b}}{+}{c}\) & \({b}{+}{c}\) \\
\end{longtable}

\hypertarget{a-case-study}{%
\section{A case study}\label{a-case-study}}

This case study describes a model by Schwarz (1994) from mathematical
psychology using the features of package mathml. Schwarz (1994) presents
a new explanation of redundancy gains that occur when observers respond
to stimuli of different sources, and the same information is presented
on two or more channels. In Schwarz's (1994) model, decision-making
builds on a process of noisy accumulation of information over time
(e.g., Ratcliff et al. 2016). In redundant stimuli, the model assumes a
superposition of channel-specific diffusion processes that eventually
reach an absorbing barrier to elicit the response. For a detailed
description the reader may refer to the original article.

Schwarz's (1994) model refers to two stimuli A and B, presented either
alone or in combination (AB, redundant stimuli), with the redundant
stimuli being presented either simultaneously or with onset asynchrony
\(\tau\). The channel activation is described as a two-dimensional Wiener
process with drifts \(\mu_i\), variances \(\sigma^2_i\), and initial
conditions \(X_i(t = 0) = 0, i = \mathrm{A, B}\). The buildup of
channel-specific activation may be correlated with \(\rho_{\mathrm{AB}}\),
but we assume \(\rho_{\mathrm{AB}} = 0\) for simplicity.

A response is elicited when the process reaches an absorbing barrier
\(c > 0\) for the first time. In single-target trials, the first passages
of \(c\) are expected at

\begin{verbatim}
ED_single <- function(c, mu)
  dfrac(c, mu)

# display as E(D; mu), c is a scaling parameter
hook(ED_single(.C, .Mu), E(`;`(D, .Mu)))
math(call("=", quote(ED_single(c, mu)), ED_single))
\end{verbatim}

\({E{\left({D}{;}{\mu}\right)}}{=}{\displaystyle{\frac{c}{\mu}}}\)

One would typically use chunk option \texttt{echo=FALSE} to suppress the R
code.

In redundant stimuli, the activation from the channel-specific diffusion
processes adds up,
\(X_{\mathrm{AB}}(t) = X_{\mathrm A}(t) + X_{\mathrm B}(t)\), hence the
name, superposition. \(X_{\mathrm{AB}}(t)\) is again a Wiener process with
drift \(\mu_{\mathrm A} + \mu_{\mathrm B}\) and variance
\(\sigma^2_{\mathrm A} + \sigma^2_{\mathrm B}\). For the expected
first-passage time, we have

\begin{verbatim}
hook(mu_A, mu["A"])
hook(mu_B, mu["B"])
hook(sigma_A, sigma["A"])
hook(sigma_B, sigma["B"])
hook(mu_M, mu["M"])
hook(M, overline(X))

math(call("=", quote(E(D["AB"])), quote(ED_single(c, mu_A + mu_B))))
\end{verbatim}

\({E{\left({D}_{\mathrm{AB}}\right)}}{=}{E{\left({D}{;}{{{\mu}_{\mathrm{A}}}{+}{{\mu}_{\mathrm{B}}}}\right)}}\)

For asynchronous stimuli, Schwarz (1994) derived the expected
first-passage time as a function of the stimulus onset asynchrony
\(\tau\),

\begin{verbatim}
ED_async <- function(tau, c, mu_A, sigma_A, mu_B)
{ dfrac(c, mu_A) + (dfrac(1L, mu_A) - dfrac(1L, mu_A + mu_B)) *
    ((mu_A*tau - c) * pnorm(dfrac(c - mu_A*tau, sqrt(sigma_A^2L*tau)))
      - (mu_A*tau + c) * exp(dfrac(2L*c*mu_A, sigma_A^2L))
        * pnorm(dfrac(-c - mu_A*tau, sqrt(sigma_A^2L*tau))))
}

hook(ED_async(.Tau, .C, .MA, .SA, .MB), E(`;`(D[.Tau], `,`(.MA, .SA, .MB))))
math(call("=", quote(E(D[tau])), ED_async))
\end{verbatim}

\({E{\left({D}_{\tau}\right)}}{=}{{\displaystyle{\frac{c}{{\mu}_{\mathrm{A}}}}}{+}{{\left({\displaystyle{\frac{1}{{\mu}_{\mathrm{A}}}}}{-}{\displaystyle{\frac{1}{{{\mu}_{\mathrm{A}}}{+}{{\mu}_{\mathrm{B}}}}}}\right)}{\cdot}{\left[{{\left({{{\mu}_{\mathrm{A}}}{{}}{\tau}}{-}{c}\right)}{\cdot}{\Phi{\left(\displaystyle{\frac{{c}{-}{{{\mu}_{\mathrm{A}}}{{}}{\tau}}}{\sqrt{{{\sigma}_{\mathrm{A}}^{2}}{{}}{\tau}}}}\right)}}}{-}{{{\left({{{\mu}_{\mathrm{A}}}{{}}{\tau}}{+}{c}\right)}{\cdot}{\exp{\left(\displaystyle{\frac{{{2}{{}}{c}}{{}}{{\mu}_{\mathrm{A}}}}{{\sigma}_{\mathrm{A}}^{2}}}\right)}}}{\cdot}{\Phi{\left(\displaystyle{\frac{{{-}{c}}{-}{{{\mu}_{\mathrm{A}}}{{}}{\tau}}}{\sqrt{{{\sigma}_{\mathrm{A}}^{2}}{{}}{\tau}}}}\right)}}}\right]}}}\)

For negative onset asynchrony (i.e., B before A), the parameters are
simply switched.

\begin{verbatim}
ED <- function(tau, c, mu_A, sigma_A, mu_B, sigma_B)
{
  if(tau == Inf) return(ED_single(c, mu_A))
  if(tau == -Inf) return(ED_single(c, mu_B))
  if(tau == 0L) return(ED_single(c, mu_A + mu_B))
  if(tau > 0L) return(ED_async(tau, c, mu_A, sigma_A, mu_B))
  if(tau < 0L) return(ED_async(abs(tau), c, mu_B, sigma_B, mu_A))
}

hook(ED(.Tau, .C, .MA, .SA, .MB, .SB), E(`;`(D[.Tau], `,`(.MA, .SA, .MB, .SB))))
math(call("=", quote(ED(tau, c, mu_A, sigma_A, mu_B, sigma_B)), ED))
\end{verbatim}

\({E{\left({{D}_{\tau}}{;}{{{\mu}_{\mathrm{A}}}{{,}{{\sigma}_{\mathrm{A}}}{{,}{{\mu}_{\mathrm{B}}}{{,}{{\sigma}_{\mathrm{B}}}}}}}\right)}}{=}{\left\{\begin{array}{l}{{E{\left({D}{;}{{\mu}_{\mathrm{A}}}\right)}},\ \mathrm{if}\ {{\tau}{=}{\infty}}}\\ {{E{\left({D}{;}{{\mu}_{\mathrm{B}}}\right)}},\ \mathrm{if}\ {{\tau}{=}{{-}{\infty}}}}\\ {{E{\left({D}{;}{{{\mu}_{\mathrm{A}}}{+}{{\mu}_{\mathrm{B}}}}\right)}},\ \mathrm{if}\ {{\tau}{=}{0}}}\\ {{E{\left({{D}_{\tau}}{;}{{{\mu}_{\mathrm{A}}}{{,}{{\sigma}_{\mathrm{A}}}{{,}{{\mu}_{\mathrm{B}}}}}}\right)}},\ \mathrm{if}\ {{\tau}{>}{0}}}\\ {{E{\left({{D}_{{\left\vert{\tau}\right\vert}}}{;}{{{\mu}_{\mathrm{B}}}{{,}{{\sigma}_{\mathrm{B}}}{{,}{{\mu}_{\mathrm{A}}}}}}\right)}},\ \mathrm{if}\ {{\tau}{<}{0}}}\end{array}\right.}\)

The observable response time is assumed to be the sum of \(D\), the time
employed to reach the threshold for the decision, and a residual \(M\)
denoting other processes such as motor preparation and execution.
Correspondingly, the expected response time amounts to

\begin{verbatim}
ET <- function(tau, c, mu_A, sigma_A, mu_B, sigma_B, mu_M)
  ED(tau, c, mu_A, sigma_A, mu_B, sigma_B) + mu_M

hook(ET(.Tau, .C, .MA, .SA, .MB, .SB, .MM),
     E(`;`(T[.Tau], `,`(.MA, .SA, .MB, .SB, .MM))))
math(call("=", quote(E(T[tau])), ET))
\end{verbatim}

\({E{\left({T}_{\tau}\right)}}{=}{{E{\left({{D}_{\tau}}{;}{{{\mu}_{\mathrm{A}}}{{,}{{\sigma}_{\mathrm{A}}}{{,}{{\mu}_{\mathrm{B}}}{{,}{{\sigma}_{\mathrm{B}}}}}}}\right)}}{+}{{\mu}_{\mathrm{M}}}}\)

Schwarz (1994) applied the model to data from a redundant signals task
(Miller 1986) with 13 onset asynchronies
\(0, \pm33, \pm67, \pm100, \pm133, \pm167, \pm\infty\) ms, where
\(\tau = 0\) refers to the synchronous condition, and \(\pm\infty\) to the
single-target presentations. Each condition was replicated 400 times.
The observed mean response times and their standard deviations are given
in Table \protect\hyperlink{tab:miller-data}{4}.

\hypertarget{tab:miller-data}{}
\begin{longtable}[]{@{}llll@{}}
\caption{Miller (1986) data}\tabularnewline
\toprule\noalign{}
\(\tau\) & \(m\) & \(s\) & \(n\) \\
\midrule\noalign{}
\endfirsthead
\toprule\noalign{}
\(\tau\) & \(m\) & \(s\) & \(n\) \\
\midrule\noalign{}
\endhead
\bottomrule\noalign{}
\endlastfoot
\({-}{\infty}\) & \(231\) & \(56\) & \(400\) \\
\({-}{167}\) & \(234\) & \(58\) & \(400\) \\
\({-}{133}\) & \(230\) & \(40\) & \(400\) \\
\({-}{100}\) & \(227\) & \(40\) & \(400\) \\
\({-}{67}\) & \(228\) & \(32\) & \(400\) \\
\({-}{33}\) & \(221\) & \(28\) & \(400\) \\
\(0\) & \(217\) & \(28\) & \(400\) \\
\(33\) & \(238\) & \(28\) & \(400\) \\
\(67\) & \(263\) & \(26\) & \(400\) \\
\(100\) & \(277\) & \(30\) & \(400\) \\
\(133\) & \(298\) & \(32\) & \(400\) \\
\(167\) & \(316\) & \(34\) & \(400\) \\
\(\infty\) & \(348\) & \(92\) & \(400\) \\
\end{longtable}

Assuming that the model is correct, the observable mean reaction times
follow an approximate Normal distribution around the model prediction
\(E(T_\tau)\) for each condition. We can, therefore, use a standard
goodness-of-fit measure by \(z\)-standardization.

\begin{verbatim}
z <- function(m, s, n, tau, c, mu_A, sigma_A, mu_B, sigma_B, mu_M)
  dfrac(m - denote(mu[tau], ET(tau, c, mu_A, sigma_A, mu_B, sigma_B, mu_M),
              "the expected mean response time"),
    s / sqrt(n))

math(call("=", quote(z[tau]), z))
\end{verbatim}

\({{z}_{\tau}}{=}{\displaystyle{\frac{{m}{-}{{\mu}_{\tau}}}{{s}{/}{\sqrt{n}}}}}\),
with
\({{\mu}_{\tau}}{=}{E{\left({{T}_{\tau}}{;}{{{\mu}_{\mathrm{A}}}{{,}{{\sigma}_{\mathrm{A}}}{{,}{{\mu}_{\mathrm{B}}}{{,}{{\sigma}_{\mathrm{B}}}{{,}{{\mu}_{\mathrm{M}}}}}}}}\right)}}\)
denoting the expected mean response time

The overall goodness-of-fit is the sum of the squared \(z\)-statistics for
each onset asynchrony. Assuming again that the architecture of the model
is correct, but the parameters are adjusted to the data, it follows a
\(\chi^2(8\ \mathrm{df})\)-distribution.

\begin{verbatim}
zv <- Vectorize(z, vectorize.args = c('m', 's', 'n', 'tau'))
hook(zv(.M, .S, .N, .Tau, .C, .MA, .SA, .MB, .SB, .MM), z[.Tau])

gof <- function(par, tau, m, s, n)
  sum(zv(m, s, n, tau, c=100L, mu_A=par["mu_A"], sigma_A=par["sigma_A"], 
         mu_B=par["mu_B"], sigma_B=par["sigma_B"], mu_M=par["mu_M"])^2L)

math(call("=", quote(X["8 df"]^2L), gof))
\end{verbatim}

\({{X}_{\mathrm{8 df}}^{2}}{=}{\sum{{z}_{\tau}^{2}}}\)

with the degrees of freedom given by the difference between the number
of observations (13) and the number of free model parameters
\(\theta = \langle\mu_{\mathrm A}, \sigma_{\mathrm A}, \mu_{\mathrm B}, \sigma_{\mathrm B}, \mu_{\mathrm M}\rangle\);
the barrier \(c\) is only a scaling parameter.

\({\hat{\theta}}{=}{\arg\min{gof{\left(\theta\right)}}}\)

The best fitting parameter values and their confidence intervals are
given in Table \protect\hyperlink{tab:params}{5}.

The goodness-of-fit statistic indicates some lack of fit,
\(X^2(8\ \mathrm{df}) = 28.34, p = 0.0004\). Given the large trial
numbers in the original study, this is not an unexpected result. For
more detail, especially on fitting the observed standard deviations, the
reader is referred to the original paper (Schwarz 1994).

\hypertarget{tab:params}{}
\begin{longtable}[]{@{}lll@{}}
\caption{Model fit}\tabularnewline
\toprule\noalign{}
Parameter & Estimate & CI \\
\midrule\noalign{}
\endfirsthead
\toprule\noalign{}
Parameter & Estimate & CI \\
\midrule\noalign{}
\endhead
\bottomrule\noalign{}
\endlastfoot
\({\mu}_{\mathrm{A}}\) & \(0.53\) & \(\left({0.51}{{,}{0.55}}\right)\) \\
\({\sigma}_{\mathrm{A}}\) & \(4.55\) & \(\left({3.95}{{,}{5.16}}\right)\) \\
\({\mu}_{\mathrm{B}}\) & \(1.36\) & \(\left({1.23}{{,}{1.49}}\right)\) \\
\({\sigma}_{\mathrm{B}}\) & \(13.46\) & \(\left({7.80}{{,}{19.11}}\right)\) \\
\({\mu}_{\mathrm{M}}\) & \(161.09\) & \(\left({156.91}{{,}{165.28}}\right)\) \\
\end{longtable}

\hypertarget{conclusion}{%
\section{Conclusion}\label{conclusion}}

This package allows R~to render its terms in pretty mathematical
equations. It extends the current features of R~and existing packages
for displaying mathematical formulas in R~(Murrell and Ihaka 2000), but
most importantly, mathml bridges the gap between computational needs,
presentation of results, and their reproducibility. The package supports
both MathML and LaTeX/MathJax for use in R~Markdown documents,
presentations and Shiny App webpages.

Researchers or teachers can already use R~Markdown to conduct analyses
and show results, and mathml smoothes this process and allows for
integrated calculations and output. As shown in the case study of the
previous section, mathml can help to improve data analyses and
statistical reports from an aesthetical perspective, as well as
regarding reproducibility of research.

Furthermore, the package may also allow for a better detection of
possible mistakes in R~programs. Similar to most programming languages
(Green 1977), R~code is notoriously hard to read, and the poor
legibility of the language is one of the main sources of mistakes. For
illustration, we consider again Equation~10 in Schwarz (1994).

\begin{verbatim}
f1 <- function(tau)
{ dfrac(c, mu_A) + (dfrac(1L, mu_A) - dfrac(1L, mu_A + mu_B) * 
    ((mu_A*tau - c) * pnorm(dfrac(c - mu_A*tau, sqrt(sigma_A^2L*tau)))
      - (mu_A*tau + c) * exp(dfrac(2L*mu_A*tau, sigma_A^2L))
        * pnorm(dfrac(-c - mu_A*tau, sqrt(sigma_A^2L*tau)))))
}

math(f1)
\end{verbatim}

\({\displaystyle{\frac{c}{{\mu}_{\mathrm{A}}}}}{+}{\left\{{\displaystyle{\frac{1}{{\mu}_{\mathrm{A}}}}}{-}{{\displaystyle{\frac{1}{{{\mu}_{\mathrm{A}}}{+}{{\mu}_{\mathrm{B}}}}}}{\cdot}{\left[{{\left({{{\mu}_{\mathrm{A}}}{{}}{\tau}}{-}{c}\right)}{\cdot}{\Phi{\left(\displaystyle{\frac{{c}{-}{{{\mu}_{\mathrm{A}}}{{}}{\tau}}}{\sqrt{{{\sigma}_{\mathrm{A}}^{2}}{{}}{\tau}}}}\right)}}}{-}{{{\left({{{\mu}_{\mathrm{A}}}{{}}{\tau}}{+}{c}\right)}{\cdot}{\exp{\left(\displaystyle{\frac{{{2}{{}}{{\mu}_{\mathrm{A}}}}{{}}{\tau}}{{\sigma}_{\mathrm{A}}^{2}}}\right)}}}{\cdot}{\Phi{\left(\displaystyle{\frac{{{-}{c}}{-}{{{\mu}_{\mathrm{A}}}{{}}{\tau}}}{\sqrt{{{\sigma}_{\mathrm{A}}^{2}}{{}}{\tau}}}}\right)}}}\right]}}\right\}}\)

The first version has a wrong parenthesis, which is barely visible in
the code, whereas in the mathematical representation, the wrong curly
brace is immediately obvious (the correct version is shown below for
comparison).

\begin{verbatim}
f2 <- function(tau)
{ dfrac(c, mu_A) + (dfrac(1L, mu_A) - dfrac(1L, mu_A + mu_B)) * 
    ((mu_A*tau - c) * pnorm(dfrac(c - mu_A*tau, sqrt(sigma_A^2L*tau)))
      - (mu_A*tau + c) * exp(dfrac(2L*mu_A*tau, sigma_A^2L))
        * pnorm(dfrac(-c - mu_A*tau, sqrt(sigma_A^2L*tau))))
}

math(f2)
\end{verbatim}

\({\displaystyle{\frac{c}{{\mu}_{\mathrm{A}}}}}{+}{{\left({\displaystyle{\frac{1}{{\mu}_{\mathrm{A}}}}}{-}{\displaystyle{\frac{1}{{{\mu}_{\mathrm{A}}}{+}{{\mu}_{\mathrm{B}}}}}}\right)}{\cdot}{\left[{{\left({{{\mu}_{\mathrm{A}}}{{}}{\tau}}{-}{c}\right)}{\cdot}{\Phi{\left(\displaystyle{\frac{{c}{-}{{{\mu}_{\mathrm{A}}}{{}}{\tau}}}{\sqrt{{{\sigma}_{\mathrm{A}}^{2}}{{}}{\tau}}}}\right)}}}{-}{{{\left({{{\mu}_{\mathrm{A}}}{{}}{\tau}}{+}{c}\right)}{\cdot}{\exp{\left(\displaystyle{\frac{{{2}{{}}{{\mu}_{\mathrm{A}}}}{{}}{\tau}}{{\sigma}_{\mathrm{A}}^{2}}}\right)}}}{\cdot}{\Phi{\left(\displaystyle{\frac{{{-}{c}}{-}{{{\mu}_{\mathrm{A}}}{{}}{\tau}}}{\sqrt{{{\sigma}_{\mathrm{A}}^{2}}{{}}{\tau}}}}\right)}}}\right]}}\)

As the reader may know from their own experience, missed parentheses are
frequent causes of wrong results and errors that are hard to locate in
programming code. This particular example shows that mathematical
rendering can help to substantially reduce the amount of careless errors
in programming.

One limitation of the package is the lack of a convenient way to insert
line breaks. This is mostly due to lacking support by MathML and LaTeX
renderers. For example, in its current stage, the LaTeX package breqn
(Robertson et al. 2021) is mostly a proof of concept. Moreover, mathml
only works in one direction, that is, it is not possible to translate
from LaTeX or HTML back to R (see Capretto 2023, for an example).

The package mathml is available for R~version 4.2 and later, and can be
easily installed using the usual \texttt{install.packages("mathml")}. At its
present stage, it supports output in HTML, LaTeX, and Microsoft Word
(via pandoc, MacFarlane 2022). The source code of the package is found
at \url{https://github.com/mgondan/mathml}.

\hypertarget{appendix-customizing-the-package}{%
\section{Appendix: Customizing the package}\label{appendix-customizing-the-package}}

\hypertarget{implementation-details}{%
\subsection{Implementation details}\label{implementation-details}}

For convenience, the translation of the R expressions is achieved
through a Prolog interpreter provided by another R~package
\href{https://CRAN.R-project.org/package=rolog}{rolog} (Gondan 2022). If a
version of SWI-Prolog (Wielemaker et al. 2012) is found on the system,
\href{https://CRAN.R-project.org/package=rolog}{rolog} connects to it.
Alternatively, the SWI-Prolog runtime libraries can be conveniently
accessed by installing the R~package
\href{https://CRAN.R-project.org/package=rswipl}{rswipl} (Gondan 2023).
Prolog is a classical logic programming language with many applications
in expert systems, computer linguistics and symbolic artificial
intelligence. The strength of Prolog lies in its concise representation
of facts and rules for knowledge and grammar, as well as its efficient
built-in search engine for closed world domains. Whereas Prolog is weak
in statistical computation, but strong in symbolic manipulation, the
converse may be said for the R~language. rolog bridges this gap by
providing an interface to a SWI-Prolog distribution (Wielemaker et al. 2012)
in R. The communication between the two systems is mainly in the
form of queries from R~to Prolog, but two Prolog functions allow ring
back and evaluation of terms in R.

The proper term for a Prolog ``function'' is predicate, and it is
typically written with name and arity (i.e., number of arguments),
separated by a forward slash. Thus, at the Prolog end, the predicate
\texttt{math/2} translates the representation of the R call \texttt{pbinom(K,~N,~Pi)}
into a more general representation of an R function \texttt{fn/2} with the name
\texttt{P\_Bi}, one argument \texttt{X~=\textless{}~K}, and the two parameters \texttt{N} and \texttt{Pi}, as
shown below.

\begin{verbatim}
math(pbinom(K, N, Pi), M)
 => M = fn(subscript('P', "Bi"), (['X' =< K] ; [N, Pi])).
\end{verbatim}

\texttt{math/2} operates like a macro that translates one mathematical element
(here, \texttt{pbinom(K,~N,~Pi)}) to another mathematical element, namely
\texttt{fn(Name,~(Args~;~Pars))}. The low-level predicate \texttt{ml/3} is used to
convert these basic elements to MathML.

\begin{verbatim}
ml(fn(Name, (Args ; Pars)), M, Flags)
 => ml(Name, N, Flags),
    ml(paren(list(op(;), [list(op(','), Args), list(op(','), Pars)])), X, Flags),
    M = mrow([N, mo(&(af)), X]).
\end{verbatim}

The relevant rule for \texttt{ml/3} builds the MathML entity
\texttt{mrow({[}N,\ mo(\&(af)),\ X{]})}, with \texttt{N} representing the name of the
function and \texttt{X} its arguments and parameters enclosed in parentheses. A
corresponding rule \texttt{jax/3} does the same for MathJax/LaTeX. A list of
flags can be used for context-sensitive translation (see, e.g., the
section on errors above).

Several ways exist for translating new R~terms to their mathematical
representation. We have already seen above how to use ``hooks'' to
translate long variable names from R to compact mathematical signs, as
well as functions such as cumulative probabilities \(P(X \le k)\) to
different representations like \(\sum_{i=0}^k P(X = i)\). Obviously, the
hooks require that there already exists a rule to translate the target
representation into MathML and MathJax.

In this appendix we describe a few more ways to extend the set of
translations according to a user's needs. As stated in the background
section, the Prolog end provides two classes of rules for translation,
macros \texttt{math/2,3,4} mirroring the R~hooks mentioned above, and the
low-level predicates \texttt{ml/3} and \texttt{jax/3} that create proper MathML and
LaTeX terms.

\hypertarget{linear-models}{%
\subsection{Linear models}\label{linear-models}}

To render the model equation of a linear model such as
\texttt{lm(EOT~\textasciitilde{}~T0~+~Therapy,~data=d)} in mathematical form (see also Anderson, Heiss, and Sumners 2023),
it is sufficient to map the
\texttt{Formula} in \texttt{lm(Formula,~Data)} to its respective equation. This can be
done in two ways, using either the hooks described above, or a new
\texttt{math/2} macro at the Prolog end.

\begin{verbatim}
hook(lm(.Formula, .Data), .Formula)
\end{verbatim}

The hook is simple, but is a bit limited because only R's tilde-form of
linear models is shown, and it only works for a call with exactly two
arguments.

Below is an example of how to build a linear equation of the form
\(Y = b_0 + b_1X_1 + ...\) using the Prolog macros from mathml.

\begin{verbatim}
math_hook(LM, M) :-
    compound(LM),
    LM =.. [lm, ~(Y, Sum) | _Tail],
    summands(Sum, Predictors),
    findall(subscript(b, X) * X, member(X, Predictors), Terms),
    summands(Model, Terms),
    M = (Y == subscript(b, 0) + Model + epsilon).
\end{verbatim}

The predicate \texttt{summands/2} unpacks an expression \texttt{A~+~B~+~C} to a list
\texttt{{[}C,~B,~A{]}} and vice-versa (see the file \texttt{lm.pl} for details).

\begin{verbatim}
rolog::consult(system.file(file.path("pl", "lm.pl"), package="mathml"))

term <- quote(lm(EOT ~ T0 + Therapy, data=d, na.action=na.fail))
math(term)
\end{verbatim}

\({EOT}{=}{{{{b}_{0}}{+}{{{{b}_{T0}}{{}}{T0}}{+}{{{b}_{Therapy}}{{}}{Therapy}}}}{+}{\epsilon}}\)

See the section above on short mathematical names for replacing
lengthy R labels.

\hypertarget{n-th-root}{%
\subsection{\texorpdfstring{\(n\)-th root}{n-th root}}\label{n-th-root}}

Base R does not provide a function like \texttt{cuberoot(x)} or
\texttt{nthroot(x,~n)}, and the present package does not support the respective
representation. To obtain a cube root, a programmer would typically type
\texttt{x\^{}(1/3)} or better \texttt{x\^{}\{1/3\}} (see the practice section why the curly
brace is preferred in an exponent), resulting in \(x^{1/3}\) which may
still not match everyone's taste. Here we describe the steps needed to
represent the \(n\)-th root as \(\sqrt[n]x\).

We assume that \texttt{nthroot(x,~n)} is available in the current namespace
(manually defined, or from R package
\href{https://CRAN.R-project.org/package=pracma}{pracma}, Borchers 2022), so
that the names of the arguments and their order are accessible to
\texttt{canonical()} if needed. As we can see below, mathml uses a default
representation \texttt{name(arguments)} for such unknown functions.

\begin{verbatim}
nthroot <- function(x, n)
  x^{1L/n}

term <- canonical(quote(nthroot(n=3L, 2L)))
math(term)
\end{verbatim}

\(nthroot{\left({2}{{,}{3}}\right)}\)

A proper MathML term is obtained by \texttt{mlx/3} (the x in mlx indicates that
it is an extension and is prioritized over the default ml/3 rules).
\texttt{mlx/3} recursively invokes \texttt{ml/3} for translating the function
arguments \emph{X} and \emph{N}, and then constructs the correct MathML entity
\texttt{\textless{}mroot\textgreater{}...\textless{}/mroot\textgreater{}}.

\begin{verbatim}
mlx(nthroot(X, N), M, Flags) :-
    ml(X, X1, Flags),
    ml(N, N1, Flags),
    M = mroot([X1, N1]).
\end{verbatim}

The explicit unification \texttt{M~=~...} in the last line serves to avoid
clutter in the head of \texttt{mlx/3}. The Prolog file \texttt{nthroot.pl} also
includes the respective rule for LaTeX and can be consulted from the
package folder via the underlying package rolog.

\begin{verbatim}
rolog::consult(system.file(file.path("pl", "nthroot.pl"), package="mathml"))

term <- quote(nthroot(a * (b + c), 3L)^2L)
math(term)
\end{verbatim}

\({\left[\sqrt[3]{{a}{\cdot}{\left({b}{+}{c}\right)}}\right]}^{2}\)

\begin{verbatim}
term <- quote(a^(1L/3L) + a^{1L/3L} + a^(1.0/3L))
math(term)
\end{verbatim}

\({{\sqrt[3]{a}}{+}{{a}^{{1}{/}{3}}}}{+}{{a}^{\left({1.00}{/}{3}\right)}}\)

The file \texttt{nthroot.pl} includes three more statements \texttt{precx/3} and
\texttt{parenx/3}, as well as a \texttt{math\_hook/2} macro. The first sets the
operator precedence of the cubic root above the power, thereby putting a
parentheses around nthroot in \((\sqrt[3]{\ldots})^2\). The second tells
the system to increase the counter of the parentheses below the root,
such that the outer parenthesis becomes a square bracket.

The last rule maps powers like \texttt{a\^{}(1L/3L)} to \texttt{nthroot/3}, as shown in
the first summand. Of course, mathml is not a proper computer algebra
system. As is illustrated by the other terms in the sum, such macros are
limited to purely syntactical matching, and terms like \texttt{a\^{}\{1L/3L\}} with
the curly brace or \texttt{a\^{}(1.0/3L)} with a floating point number in the
numerator are not detected.

\hypertarget{acknowledgment}{%
\section{Acknowledgment}\label{acknowledgment}}

Supported by the Erasmus+ program of the European Commission
(2019-1-EE01-KA203-051708).

\hypertarget{references}{%
\section*{References}\label{references}}
\addcontentsline{toc}{section}{References}

\hypertarget{refs}{}
\begin{CSLReferences}{1}{0}
\leavevmode\vadjust pre{\hypertarget{ref-rmarkdown}{}}%
Allaire, J. J., Y. Xie, C. Dervieux, J. McPherson, J. Luraschi, K. Ushey, A. Atkins, et al. 2023. \emph{Rmarkdown: Dynamic Documents for {R}}. \url{https://github.com/rstudio/rmarkdown}.

\leavevmode\vadjust pre{\hypertarget{ref-equatiomatic}{}}%
Anderson, D., A. Heiss, and J. Sumners. 2023. \emph{Equatiomatic: Transform Models into 'LaTeX' Equations}.

\leavevmode\vadjust pre{\hypertarget{ref-pracma}{}}%
Borchers, H. W. 2022. \emph{{pracma}: Practical Numerical Math Functions}. \url{https://CRAN.R-project.org/package=pracma}.

\leavevmode\vadjust pre{\hypertarget{ref-latex2r}{}}%
Capretto, T. 2023. \emph{{latex2r}: Translate Latex Formulas to {R} Code}. \url{https://github.com/tomicapretto/latex2r}.

\leavevmode\vadjust pre{\hypertarget{ref-Chang2022}{}}%
Chang, W., J. Cheng, J. J. Allaire, C. Sievert, B. Schloerke, Y. Xie, J. Allen, J. McPherson, A. Dipert, and B. Borges. 2022. \emph{Shiny: Web Application Framework for {R}}. \url{https://CRAN.R-project.org/package=shiny}.

\leavevmode\vadjust pre{\hypertarget{ref-rolog}{}}%
Gondan, M. 2022. \emph{{rolog}: Query {SWI}-{P}rolog from {R}}. \url{https://github.com/mgondan/rolog}.

\leavevmode\vadjust pre{\hypertarget{ref-rswipl}{}}%
---------. 2023. \emph{{rswipl}: Embed {SWI}-{P}rolog}. \url{https://CRAN.R-project.org/package=rswipl}.

\leavevmode\vadjust pre{\hypertarget{ref-green1977}{}}%
Green, T. R. G. 1977. {``Conditional Program Statements and Their Comprehensibility to Professional Programmers.''} \emph{Journal of Occupational Psychology} 50: 93--109.

\leavevmode\vadjust pre{\hypertarget{ref-pandoc}{}}%
MacFarlane, J. 2022. \emph{Pandoc: A Universal Document Converter}.

\leavevmode\vadjust pre{\hypertarget{ref-miller}{}}%
Miller, J. O. 1986. {``Timecourse of Coactivation in Bimodal Divided Attention.''} \emph{Perception \& Psychophysics} 40: 331--43.

\leavevmode\vadjust pre{\hypertarget{ref-murrell2000}{}}%
Murrell, P., and R. Ihaka. 2000. {``An Approach to Providing Mathematical Annotation in Plots.''} \emph{Journal of Computational and Graphical Statistics} 9: 582--99.

\leavevmode\vadjust pre{\hypertarget{ref-R}{}}%
R Core Team. 2022. \emph{R: A Language and Environment for Statistical Computing}. Vienna, Austria: R Foundation for Statistical Computing. \url{https://www.R-project.org/}.

\leavevmode\vadjust pre{\hypertarget{ref-ratcliff2016}{}}%
Ratcliff, R., P. L. Smith, S. D. Brown, and G. McKoon. 2016. {``Diffusion Decision Model: Current Issues and History.''} \emph{Trends in Cognitive Sciences} 20: 260--81.

\leavevmode\vadjust pre{\hypertarget{ref-breqn}{}}%
Robertson, W., J. Wright, F. Mittelbach, and U. Fischer. 2021. \emph{Breqn: Automatic Line Breaking of Displayed Equations}. \url{https://www.ctan.org/pkg/breqn}.

\leavevmode\vadjust pre{\hypertarget{ref-Sarkar2022}{}}%
Sarkar, D., and K. Hornik. 2022. \emph{Enhancements to {HTML} {D}ocumentation}. \url{https://blog.r-project.org/2022/04/08/enhancements-to-html-documentation/index.html}.

\leavevmode\vadjust pre{\hypertarget{ref-schwarz1994}{}}%
Schwarz, W. 1994. {``Diffusion, Superposition, and the Redundant-Targets Effect.''} \emph{Journal of Mathematical Psychology} 38: 504--20.

\leavevmode\vadjust pre{\hypertarget{ref-Viechtbauer2022}{}}%
Viechtbauer, W. 2022. \emph{{mathjaxr}: Using 'Mathjax' in Rd Files}. \url{https://CRAN.R-project.org/package=mathjaxr}.

\leavevmode\vadjust pre{\hypertarget{ref-Wickham2019}{}}%
Wickham, H. 2019. \emph{Advanced {R}}. Cambridge: Chapman; Hall/CRC.

\leavevmode\vadjust pre{\hypertarget{ref-swipl}{}}%
Wielemaker, J., T. Schrijvers, M. Triska, and T. Lager. 2012. {``{SWI-Prolog}.''} \emph{Theory and Practice of Logic Programming} 12 (1-2): 67--96.

\leavevmode\vadjust pre{\hypertarget{ref-knitr}{}}%
Xie, Y. 2023. \emph{{knitr}: A General-Purpose Package for Dynamic Report Generation in {R}}. \url{https://yihui.org/knitr/}.

\leavevmode\vadjust pre{\hypertarget{ref-Xie2018}{}}%
Xie, Y., J. J. Allaire, and G. Grolemund. 2018. \emph{{R} Markdown: The Definitive Guide}. Boca Raton, Florida: Chapman; Hall/CRC. \url{https://bookdown.org/yihui/rmarkdown}.

\leavevmode\vadjust pre{\hypertarget{ref-Xie2020}{}}%
Xie, Y., C. Dervieux, and E. Riederer. 2020. \emph{R Markdown Cookbook}. Cambridge: Chapman; Hall/CRC.

\end{CSLReferences}


\address{%
Matthias Gondan\\
Universität Innsbruck\\%
Department of Psychology\\ Innsbruck, Austria\\ \url{https://www.uibk.ac.at/psychologie/mitarbeiter/gondan-rochon/index.html.en}\\
%
%
\textit{ORCiD: \href{https://orcid.org/0000-0001-9974-0057}{0000-0001-9974-0057}}\\%
\href{mailto:Matthias.Gondan-Rochon@uibk.ac.at}{\nolinkurl{Matthias.Gondan-Rochon@uibk.ac.at}}%
}

\address{%
Irene Alfarone\\
Universität Innsbruck\\%
Department of Psychology\\ Innsbruck, Austria\\ \url{https://www.uibk.ac.at/psychologie/mitarbeiter/alfarone/index.html.en}\\
%
%
\textit{ORCiD: \href{https://orcid.org/0000-0002-8409-8900}{0000-0002-8409-8900}}\\%
\href{mailto:Irene.Alfarone@uibk.ac.at}{\nolinkurl{Irene.Alfarone@uibk.ac.at}}%
}
