% !TeX root = RJwrapper.tex
\title{Kuhn-Tucker and Multiple Discrete-Continuous Extreme Value Model
Estimation and Simulation in R: The rmdcev Package}
\author{by Patrick Lloyd-Smith}

\maketitle

\abstract{%
This paper introduces the package \pkg{rmdcev} in R for estimation and
simulation of Kuhn-Tucker demand models with individual heterogeneity.
The models supported by \pkg{rmdcev} are the multiple-discrete
continuous extreme value (MDCEV) model and Kuhn-Tucker specification
common in the environmental economics literature on recreation demand.
Latent class and random parameters specifications can be implemented and
the models are fit using maximum likelihood estimation or Bayesian
estimation. The \pkg{rmdcev} package also implements demand forecasting
and welfare calculation for policy simulation. The purpose of this paper
is to describe the model estimation and simulation framework and to
demonstrate the functionalities of \pkg{rmdcev} using real datasets.
}

\hypertarget{introduction}{%
\section{Introduction}\label{introduction}}

Individual choice contexts are often characterized by both extensive
(i.e.~what alternative to choose) and intensive (i.e.~how much of an
alternative to consume) margins \citep{bhatmultiple2008}. These multiple
discrete-continuous (MDC) choice situations are pervasive, arising in
transportation, marketing, health, and decisions regarding environmental
resources \citep{bhatmultiple2014}. The Kuhn-Tucker (KT) modelling
framework is often employed to analyze these MDC situations and
substantial progress has been made in improving these econometric
modeling structures \citep{vonhaefenkuhn-tucker2005, bhatmultiple2014}.
Despite the large potential applications for KT models, there remains a
gap between this potential and actual examples of these models being
used. One of the reasons cited for the lack of widespread use of KT
models is that estimating and simulating these models is challenging.
The explanations of methods used to work with these models are spread
across many papers and few user friendly software tools are available.
The purpose of this paper is to present a unified account for KT
estimation and simulation alongside computer code for easy and efficient
implementation.

This paper presents an overview of the R package \pkg{rmdcev} which can
estimate and simulate KT demand models with discrete or continuous
unobserved individual heterogeneity.\footnote{This paper uses version
  1.2.0 of the \pkg{rmdcev} package.} The common starting point for all
KT models is the individual's constrained optimization problem and
exploiting the resulting KT first order conditions in estimation. The
most popular empirical KT modelling framework is the multiple-discrete
continuous extreme value (MDCEV) model as first introduced by Bhat
(2008). A separate stream of literature in the environmental economics
on recreation demand has developed a closely related set of models and
use the term KT to describe the models. In this paper, we use KT to
describe the general modelling framework, MDCEV to describe the Bhat
(2008) specifications, and KT-EE to describe the environmental economics
literature KT specification \citep{vonhaefenestimation2004}. One of the
main differences between the MDCEV and KT-EE frameworks is how
alternative-specific attributes enter the utility function, a point we
describe in the paper.

Incorporating preference heterogeneity has been an important advancement
in choice modeling. Both the MDCEV and KT-EE specifications can be
estimated to incorporate unobserved preference heterogeneity by assuming
continuous distributions using random parameters or using a latent class
(LC) specification assuming a discrete distribution where people can be
divided into distinct segments. The models in \pkg{rmdcev} can be fit
using maximum likelihood estimation or Bayesian estimation. Besides
estimation, the \pkg{rmdcev} package also implements demand forecasting
and welfare calculation for policy simulation. The two main functions in
the \pkg{rmdcev} are \code{mdcev} used to estimate all model
specifications and \code{mdcev.sim} used to simulate both demand and
welfare implications. \pkg{rmdcev} is available from the Comprehensive R
Archive Network (CRAN) at
\url{https://CRAN.R-project.org/package=rmdcev} as well as from GitHub
at \url{https://github.com/plloydsmith/rmdcev}.

While there are several R packages available to estimate discrete choice
data such as \pkg{apollo} \citep{hessapollo2019}, \pkg{mlogit}
\citep{mlogit2019}, and \pkg{gmnl}
\citep{sarriasmultinomial2017}\footnote{\citet{sarriasmultinomial2017}
  provides a good overview of the different R packages available to
  estimate discrete choice models}, there are limited options for users
interested in estimating and simulating KT models. In addition to
\pkg{rmdcev}, the
\href{http://www.apollochoicemodelling.com/}{\pkg{apollo}} package
developed by Stephane Hess and David Palma at the Choice Modelling
Centre in Leeds provides a flexible modelling platform for estimating
MDCEV models and simulating demand behaviour \citep{hessapollo2019}.
\pkg{apollo} estimates a full suite of choice models including discrete
choice models and is thus more comprehensive and flexible than
\pkg{rmdcev}. The main advantages for KT modeling in using the
\pkg{rmdcev} is that it 1) provides functions for calculating welfare
implications of policy scenarios, 2) allows the estimation and
simulation of the KT formulation used in environmental economics
\citep{vonhaefenkuhn-tucker2005}, 3) uses the Stan program
\citep{carpenterstan2017} for Bayesian estimation and thus the user has
access to specialized postestimation commands, and 4) is primarily coded
in C++ and thus around 20 times faster. The main limitations
\pkg{rmdcev} compared to \pkg{apollo} is that it 1) only estimates model
specifications with an outside good that is always consumed whereas
\pkg{apollo} can estimate models without an outside good, 2) users have
more control over which particular parameters are fixed at their
starting values and which are allowed to be random parameters, and 3)
\pkg{apollo} allows users to estimate the multiple discrete continuous
nested extreme value model and LC-random parameter MDCEV specifications.

The paper first introduces the conceptual framework underlying KT models
and the connection to economic theory and welfare measures. Section
\protect\hyperlink{models}{2} also describes the various empirical
specifications for KT models. Section \protect\hyperlink{rmdcev}{3}
introduces the \pkg{rmdcev} package focusing first on estimation before
moving on to discuss how to conduct welfare and demand simulations.
Section \protect\hyperlink{conclusions}{4} provides conclusions of the
paper.

\hypertarget{models}{%
\section{Models}\label{models}}

\hypertarget{conceptual-framework}{%
\subsection{Conceptual framework}\label{conceptual-framework}}

This section describes the underlying conceptual framework for KT
models. Each individual \(i\) maximizes utility through the choice of
the numeraire or outside good (\(x_{i1}\)) and the non-numeraire
alternatives (\(x_{ik}\)) subject to a monetary or non-monetary budget
constraint. We assume there is a numeraire good (i.e.~essential Hicksian
composite good) which is always consumed and has a price of one. The
individual's maximization problem is

\begin{align}
\begin{split}
 & \max_{x_{ik}, x_{i1}} U(x_{ik}, x_{i1}) \\
& s.t. \;\;\;y_i = \sum\limits^K_{k=2}p_{ik} x_{ik} + x_{i1}, \;\; x_{ik} \geq 0,\;\; k = 2,...,K,
  \end{split}
\end{align}

\noindent where \(x_{ik}\) is the consumption level for alternative
\(k\), \(x_{i1}\) is consumption of the numeraire, \(y_i\) is any
arbitrary budget amount (e.g.~annual income), and \(p_{ik}\) is the unit
price of alternative \(k\).

The resulting first-order KT conditions that implicitly define the
solution to the optimal consumption bundles of \(x_{ik}\) and \(x_{i1}\)
are

\begin{align}
\label{eq:kt_conditions}
\begin{split}
 \frac{U_{x_{ik}}}{U_{x_{i1}}} & \leq p_{ik},\; \; k = 1,....K , \\
  x_{ik}\left[\frac{U_{x_{ik}}}{U_{x_{i1}}} - p_{ik} \right] & = 0,\; \; k = 1,....K.
  \end{split}
\end{align}

For alternatives with positive consumption levels, the marginal rate of
substitution between these alternatives and the numeraire good is equal
to the price of the alternative. For unconsumed alternatives, the
marginal rate of substitution between these alternatives and the
numeraire good is less than the price of the alternatives. For the rest
of the paper, we drop the subscript \(i\) for notational simplicity.

These first-order conditions can be used to derive Marshallian and
Hicksian demands and welfare measures \citep{vonhaefenkuhn-tucker2005}.
We assume that alternatives have non-price attribute \(q_{k}\) and the
vector of \(k\) prices and attributes is denoted as \(p\) and \(q\). The
Hicksian compensating surplus (\(CS^H\)) for a change in price and
quality from baseline levels \(p^0\) and \(q^0\) to new `policy' levels
\(p^1\) and \(q^1\) is defined explicitly using an expenditure function

\begin{equation}
\label{eq:welfare}
CS^H = y - e(p^1, q^1, \bar{U}, \theta, \varepsilon),
\end{equation}

\noindent where \(\theta\) is the vector of structural parameters
(\(\psi_k, \alpha_k, \gamma_k\)), \(\varepsilon\) is a vector or matrix
of unobserved heterogeneity, and
\(\bar{U} = V(p^0, q^0, y, \theta, \varepsilon)\) and represents
baseline utility.

\hypertarget{multiple-discrete-continuous-extreme-value-model-mdcev}{%
\subsection{Multiple discrete-continuous extreme value model
(MDCEV)}\label{multiple-discrete-continuous-extreme-value-model-mdcev}}

The \pkg{rmdcev} package implements the random utility specification of
the MDCEV as introduced by \citet{bhatmultiple2008}. The model
specifications included in \pkg{rmdcev} always assume an outside good
(i.e.~the numeraire good that is always consumed by every individual).
The general utility function is specified as

\begin{equation}
U(x_k, x_1) = \sum_{k=2}^{K} \frac{\gamma_k}{\alpha_k}\psi_k \left[ \left( \frac{x_k}{\gamma_k} + 1 \right)^{\alpha_k} - 1 \right] + \frac{\psi_1}{\alpha_1}x_1^{\alpha_1} \label{utilkt},
\end{equation}

\noindent where \(\gamma_k > 0\), \(\psi_k > 0\) and \(\alpha_k \leq 1\)
for all \(k\) are required for this specification to be consistent with
the properties of a utility function \citep{bhatmultiple2008}.
\citet{bhatmultiple2008} provides a detailed overview of the parameter
interpretation and in brief

\begin{itemize}
\tightlist
\item
  The \(\psi_k\) parameters represent the marginal utility of consuming
  alternative \(k\) at the point of zero consumption (i.e.~baseline
  marginal utility).
\item
  The \(\gamma_k\) parameters are translation parameters that allow for
  corner solutions (i.e.~zero consumption levels for alternatives) and
  also influence satiation. The lower the value of \(\gamma_k\), the
  greater the satiation effect in consuming \(x_k\).
\item
  The \(\alpha_k\) parameters control the rate of diminishing marginal
  utility of additional consumption. If \(\alpha_k\) equal to one, then
  there is no satiation effects (i.e.~constant marginal utility).
\end{itemize}

The `random utility' element of the model is introduced into the
baseline utility through a random error term as

\begin{equation}
\label{eq:psi}
\psi_k=\psi(z_k,\varepsilon_k)= exp(\beta'z_k+\varepsilon_k),
\end{equation}

\noindent where \(z_k\) is a set of variables that can include
alternative-specific attributes and individual-specific characteristics,
and \(\varepsilon_k\) is an error term that allows for the utility
function to be random over the population. We assume an extreme value
distribution that is independently distributed across alternatives for
\(\varepsilon_k\) with an associated scale parameter of \(\sigma\). For
identification, we specify \(\psi_1= e^{\varepsilon_1}\).

To ensure the estimated utility function corresponds to economic theory
we specify \(\gamma_k = exp(\gamma^*_k)\) such that \(\gamma_k > 0\) and
\(\alpha_k = exp(\alpha^*_k)/(1 + exp(\alpha^*_k))\) such that
\(0 < \alpha_k < 1\). \(\gamma^*_k\) and \(\alpha^*_k\) are estimated in
the package and \(\gamma_k\) and \(\alpha_k\) are reported to the user.
Similarly, we specify \(\sigma = exp(\sigma^*)\). Weak complementarity,
which is required for deriving unique welfare measures
\citep{malerenvironment1974}, is imposed in this specification by adding
and subtracting one in the non-numeraire part of the utility function.

While the most general form of the MDCEV model includes \(\psi_k\),
\(\gamma_k\), and \(\alpha_k\) parameters for each alternative, Bhat
(2008) discusses the identification concerns regarding estimating
separate \(\gamma_k\) and \(\alpha_k\) parameters for each non-numeraire
alternative. Typically only a subset of these parameters can be
identified and there are three common utility function specifications:

\begin{enumerate}
\def\labelenumi{\arabic{enumi}.}
\tightlist
\item
  \(\alpha\)-profile: set all \(\gamma_k\) parameters to 1.
\end{enumerate}

\begin{equation}
\label{eq:alpha}
U(x_k, x_1) = \sum_{k=2}^{K} \frac{1}{\alpha_k}exp(\beta'z_k+\varepsilon_k) \left[ \left( x_k + 1 \right)^{\alpha_k} - 1 \right] + \frac{exp(\varepsilon_1)}{\alpha_1}x_1^{\alpha_1}.
\end{equation}

\begin{enumerate}
\def\labelenumi{\arabic{enumi}.}
\setcounter{enumi}{1}
\tightlist
\item
  \(\gamma\)-profile: set all non-numeraire \(\alpha_k\) parameters to
  0.
\end{enumerate}

\begin{equation}
\label{eq:gamma}
U(x_k, x_1) = \sum_{k=2}^{K} \gamma_k exp(\beta'z_k+\varepsilon_k) \ln\left( \frac{x_k}{\gamma_k} + 1 \right) + \frac{exp(\varepsilon_1)}{\alpha_1}x_1^{\alpha_1}.
\end{equation}

\begin{equation}
\label{eq:gamma}
U(x_k, x_1) = \sum_{k=2}^{K} exp(\beta'z_k+\varepsilon_k) \ln\left( \phi_k x_k + \gamma_k \right) + \frac{1}{\alpha_1}x_1^{\alpha_1}.
\end{equation}

\begin{enumerate}
\def\labelenumi{\arabic{enumi}.}
\setcounter{enumi}{2}
\tightlist
\item
  hybrid-profile: set all \(\alpha_k=\alpha_1=\alpha\).
\end{enumerate}

\begin{equation}
\label{eq:hybrid}
U(x_k, x_1) = \sum_{k=2}^{K} \frac{\gamma_k}{\alpha} exp(\beta'z_k+\varepsilon_k) \left[ \left( \frac{x_k}{\gamma_k} + 1 \right)^{\alpha} - 1 \right] + \frac{exp(\varepsilon_1)}{\alpha}x_1^{\alpha}.
\end{equation}

The likelihood function representing the model probability of the
consumption pattern where \(M\) alternatives are chosen can be expressed
as \citet{bhatmultiple2008}

\begin{equation}
\label{eq:ll_base}
P(x^{*}_1,x^{*}_2...x^{*}_M,0,...,0) = \frac{1}{\sigma^{M-1}} \left(\prod_{m=1}^M c_m \right)\left(\sum_{m=1}^M \frac{p_m}{c_m} \right) \left( \ \frac{\prod_{m=1}^M e^{V_m/\sigma}}{ \left( \sum_{k=1}^J e^{V_k/\sigma} \right)^M }\right)(M-1)!,
\end{equation}

\noindent where \(\sigma\) is the scale parameter and
\(c_m = \frac{1-\alpha_m}{x_m+ \gamma_m}\). The \(V\) expression depend
on what model specification is used:

\begin{enumerate}
\def\labelenumi{\arabic{enumi}.}
\item
  \(\alpha\)-profile:
  \(V_k = \beta' z_k + (\alpha_k-1)\ln\left( x_k + 1 \right) - \ln \left(p_k\right)\)
  for \(k \geq 2\), and \(V_1 = (\alpha_1-1)\ln(x_1)\).
\item
  \(\gamma\)-profile:
  \(V_k = \beta' z_k - \ln\left( \frac{x_k}{\gamma_k} + 1 \right) - \ln \left(p_k\right)\)
  for \(k \geq 2\), and \(V_1 = (\alpha_1-1)\ln(x_1)\).
\item
  hybrid-profile:
  \(V_k = \beta' z_k + (\alpha-1)\ln\left( \frac{x_k}{\gamma_k} + 1 \right) - \ln \left(p_k\right)\)
  for \(k \geq 2\), and \(V_1 = (\alpha-1)\ln(x_1)\).
\end{enumerate}

\hypertarget{kuhn-tucker-model-specifications-in-environmental-economics-kt-ee}{%
\subsection{Kuhn-Tucker model specifications in Environmental Economics
(KT-EE)}\label{kuhn-tucker-model-specifications-in-environmental-economics-kt-ee}}

The \pkg{rmdcev} package also implements the KT-EE specification
\citep{vonhaefenkuhn-tucker2005}. The utility function in this
specification is similar to the \(\gamma\)-profile of the MDCEV
specification introduced above and is

\begin{equation}
U(x_k, x_1) = \sum_{k=2}^{K}\psi_k \ln \left(\phi_kx_k + \gamma_k \right) + \frac{1}{\alpha_1}x_1^{\alpha_1}, 
\label{eq:util_kt_ee}
\end{equation}

\noindent where \(\phi_k >0\).\footnote{The environmental economics
  literature uses slightly different notation as typically \(\theta\) is
  used for \(\gamma\), \(\mu\) is used for \(\sigma\), and \(\rho\) for
  \(\alpha_1\). We change the notation slightly for consistency with the
  MDCEV model specifications.}

An important difference between this KT formulation and the MDCEV models
is the way weak complementary is imposed. In this KT formulation, weak
complementarity is imposed by only including alternative-specific
attributes in the \(\phi_k\) parameter and not the \(\psi_k\)
parameter.\footnote{See \citet{herrigeswhats2004} for more discussion on
  this point.}

In this formulation, the estimating first-order conditions can be
written as

\begin{equation}
\varepsilon_k \leq \frac{1}{\sigma}\left( -\beta' s + \ln(\frac{p_k}{\phi_k}) + \ln(\phi_k x_k + \gamma_k) + (\alpha_1 - 1)\ln (y - p_k * x_k) \right ), \; \; \forall k,
\label{eq:kt_g}
\end{equation}

and the resulting likelihood function as

\begin{equation}
P(x) = |J| \prod_k \left[exp(-g_k(.))/ \sigma \right]^{1(x_k>0)} exp[-exp(-g_k(.))],
\label{eq:ll_kt_ee}
\end{equation}

where \(|J|\) is the determinant of the Jacobian of transformation,
\(g_k(.)\) is the right hand side of Equation (\ref{eq:kt_g}), and
\(1(x_k>0)\) is equal to one if \(x_k\) is positive and equal to zero if
\(x_k\) is zero \citep{vonhaefenkuhn-tucker2005}. In previous
implementations, the KT formulation used the computationally intensive
numerical gradient approach to the calculation of the determinant of the
Jacobian of transformation \citep{vonhaefenkuhn-tucker2005}.

The \pkg{rmdcev} package uses the compact structure of the determinant
of the Jacobian as derived by Bhat (2008) and defined as

\begin{equation}
|J| = \frac{(1-\alpha_1)}{x_1} \left[ \prod_m \frac{\phi_m}{\phi_m * x_m + \gamma_m} \right] \left[ {x_1}{(1-\alpha_1)} + \sum_m \frac{(\phi_m * x_m + \gamma_m)* p_m}{\phi_m} \right],
\end{equation}

where \(m\) denotes non-numeraire alternatives with positive consumption
levels. Using this analytical gradient approach has the benefit of
substantially speeding up estimation by around 70\% relative to the
numerical gradient approach.

In both the MDCEV and KT-EE specifications described above, the
parameters (\(\beta, \alpha_k , \gamma_k, \phi_k, \sigma\)) are
structural parameters that are assumed to be equal across the population
which simplifies estimation. However, these fixed parameter
specification is quite restrictive as they can only incorporate
preference heterogeneity through interaction terms with observed
individual characteristics. Without these interaction terms, the fixed
specifications impose the assumption that all individuals have the same
tastes for alternatives (i.e.~preference homogeneity). This assumption
is relaxed in the next two specifications which are able to accommodate
both observed and unobserved preference heterogeneity.

\hypertarget{latent-class-lc-kt-models}{%
\subsection{Latent class (LC-KT)
models}\label{latent-class-lc-kt-models}}

The latent class version of the KT model assumes that an individual
belongs to a finite mixture of \(S\) segments each indexed by \(s\)
(\(s=1,2,...S\)) \citep{sobhanilatent2013, kuriyamalatent2010}. Within
each segment, the LC specification assumes preference homogeneity. We do
not observe which segment an individual belongs to but we can attribute
a probability \(\pi_{is}\) that individual \(i\) is a member of segment
\(s\). We impose that \(0 \leq \pi_{is} \leq 1\) and
\(\sum^S_{s=1} \pi_{is} = 1\) through the use of the logit link function
as

\begin{equation}
\pi_{is} = \frac{exp(\delta_s'w_i)}{\sum^S_{s=1}exp(\delta_s'w_i)},
\end{equation}

\noindent where \(w_i\) is a vector of individual characteristics and
\(\delta_s\) is a vector of coefficients to be estimated. The
\(\delta_s\) coefficients determine how the individual characteristics
affect the membership of individual \(i\) in segment \(s\). For
identification, the \(\delta_1\) coefficients for the first segment are
set to zero.

The likelihood function can be written as

\begin{equation}
P = \prod_{i} \pi_{is}P_{is},
\end{equation}

where \(P_{is}\) has the same form as Equations (\ref{eq:ll_base}) and
Equations (\ref{eq:ll_kt_ee}) but is now class specific.

\hypertarget{random-parameters-rp-lc-models}{%
\subsection{Random parameters (RP-LC)
models}\label{random-parameters-rp-lc-models}}

The random parameter specification of the LC models assumes that the
structural parameters \(\theta = (\beta, \alpha_k , \gamma_k)\) are not
necessarily fixed but have an assumed distribution
\citep{bhatmultiple2008}. In \pkg{rmdcev}, parameters are distributed
multivariate normal with a mean \(\bar{\theta}\) and variance covariance
matrix \(\sum_{\theta}\) \citep{vonhaefenkuhn-tucker2005}. This
structure allows for continuous preference heterogeneity and
accommodates more flexible correlation patterns between alternatives in
a similar fashion to the mixed logit model in discrete choice models.
The \(\sigma\) scale parameter is always assumed to be a fixed
parameter.

The most flexible model specification is to estimate the full variance
covariance matrix and if there are \(Q\) parameters in \(\theta\) then
there are \(Q(Q+1)/2\) unique variance covariance parameters to estimate
in the correlated RP-MDCEV specification. An alternative is to assume
the off-diagonal parameters are zero and estimate uncorrelated random
parameters by estimating the \(Q\) diagonal elements of
\(\sum_{\theta}\). If all elements of \(\sum_{\theta}\) are assumed to
be zero, the model collapses to the fixed KT structures.

\hypertarget{a-note-on-bayesian-versus-classical-maximum-likelihood-estimation}{%
\subsection{A note on Bayesian versus classical maximum likelihood
estimation}\label{a-note-on-bayesian-versus-classical-maximum-likelihood-estimation}}

The KT model without unobserved heterogeneity can be estimated using
Bayesian or classical maximum likelihood techniques. The LC-KT model can
only be estimated using classical maximum likelihood techniques as
Bayesian approaches are challenged by the `label switching' problem
\citep{jasra2005}. The RP-KT models can only be estimated using Bayesian
techniques as random parameter models require simulated maximum
likelihood estimators and these are not implemented in \pkg{rmdcev} at
this time.

While there are philosophical differences between Bayesian and classical
maximum likelihood techniques to estimating models, the Bernstein-von
Mises theorem suggests that the Bayesian posterior distribution are
asymptotically equivalent to maximum likelihood estimates if the data
generating process has been correctly specified
\citep{traindiscrete2003}.

\hypertarget{rmdcev}{%
\section{The rmdcev package}\label{rmdcev}}

\hypertarget{data-format}{%
\subsection{Data format}\label{data-format}}

The \pkg{rmdcev} uses \code{mdcev.data} function for handling multiple
discrete-continuous data while ensuring the data is in the correct
format and is suitable for estimation. The \pkg{rmdcev} package accepts
data in ``long'' format (i.e.~one row per available non-numeraire
alternative for each individual). There is no row for the numeraire
(i.e.~outside) good. If there are \(I\) individuals and \(J\)
non-numeraire alternatives, then the data frame should have \(IxJ\)
rows.

To illustrate the suitable form of the data, we can load the recreation
data included with the \pkg{rmdcev} package. This data is from the
Canadian Nature Survey and includes choices for number of days spent
recreating in 17 different outdoor activities for 2,000 people
\citep{federal20122014}.

\begin{Schunk}
\begin{Sinput}
data(data_rec, package = "rmdcev")
\end{Sinput}
\end{Schunk}

Each recreation activity is characterized by the daily costs of
participation for each individual. In addition to the recreation
behaviour and prices, the data includes information on three individual
characteristics: university (a dummy variable if the person has
completed a university degree), ageindex (a person's age divided by the
average age in sample), and urban (a dummy variable if a person lives in
an urban area). Additional details on the data and price construction
are provided in \citet{lloydsmitheconomics2020}. We can summarize the
average consumption and price levels for each alternative as:

\begin{Schunk}
\begin{Sinput}
aggregate(cbind(quant, price) ~ alt, data = data_rec, FUN = mean )
\end{Sinput}
\begin{Soutput}
#>               alt   quant     price
#> 1           beach  6.5375  53.18359
#> 2         birding 14.3835  44.01734
#> 3         camping  2.5125  61.38326
#> 4         cycling  9.4700  45.99470
#> 5            fish  3.3435  86.22383
#> 6          garden 21.5710  38.28073
#> 7            golf  4.0260 134.10374
#> 8          hiking 41.4150  37.53204
#> 9      hunt_birds  0.4855 111.00176
#> 10     hunt_large  0.9480 184.46812
#> 11      hunt_trap  0.6290  95.33228
#> 12 hunt_waterfowl  0.2085 159.66605
#> 13     motor_land  3.7040 123.10169
#> 14    motor_water  2.8390 139.63845
#> 15          photo  8.6415  67.13733
#> 16      ski_cross  2.6450  32.65243
#> 17       ski_down  1.2065 151.01398
\end{Soutput}
\end{Schunk}

The data can be transformed into the structure for MDCEV estimation
using the \code{mdcev.data} function:

\begin{Schunk}
\begin{Sinput}
data_mdcev <- mdcev.data(data_rec,
                       id.var = "id",
                       alt.var = "alt",
                       choice = "quant")
\end{Sinput}
\begin{Soutput}
#> Sorting data by id.var then alt...
\end{Soutput}
\begin{Soutput}
#> Checking data...
\end{Soutput}
\begin{Soutput}
#> Data is good
\end{Soutput}
\end{Schunk}

The \code{id.var} argument indicates what variable uniquely identifies
individuals in the data set, \code{alt.var} indicates the variable that
identifies the non-numeraire alternatives, and \code{choice} indicates
the level of consumption made by the individuals. Two other optional
arguments of \code{mdcev.data} are \code{price} and \code{income}
indicating the individual-specific price levels for each alternative,
and the income level for each individual. These two arguments only need
to be explicitly specified if they are not labeled price and income.
Alternative-specific attributes and individual-specific characteristics
can be included as additional columns and do not need to be specified in
\code{mdcev.data}.

The \code{mdcev.data} function also checks to ensure the data has the
necessary variables, and that all individuals spend positive amounts on
the numeraire good. If an individual does not have positive expenditures
on the numeraire good, an error message is given.

\hypertarget{kt-estimation}{%
\subsection{KT estimation}\label{kt-estimation}}

\hypertarget{a-general-overview-of-mdcev}{%
\subsubsection{A general overview of
mdcev}\label{a-general-overview-of-mdcev}}

The \pkg{rmdcev}

All the various KT model specifications are estimated using the
\code{mdcev} function.

\begin{Schunk}
\begin{Sinput}
args(mdcev)
\end{Sinput}
\begin{Soutput}
#> function (formula = NULL, data, weights = NULL, model = c("alpha", 
#>     "gamma", "hybrid", "hybrid0", "kt_ee"), n_classes = 1, fixed_scale1 = 0, 
#>     trunc_data = 0, psi_ascs = NULL, gamma_ascs = 1, seed = "123", 
#>     max_iterations = 2000, jacobian_analytical_grad = 1, initial.parameters = NULL, 
#>     hessian = TRUE, algorithm = c("MLE", "Bayes"), flat_priors = NULL, 
#>     print_iterations = TRUE, prior_psi_sd = 10, prior_gamma_sd = 10, 
#>     prior_phi_sd = 10, prior_alpha_shape = 1, prior_scale_sd = 1, 
#>     prior_delta_sd = 10, gamma_nonrandom = 0, alpha_nonrandom = 0, 
#>     std_errors = "deltamethod", n_draws = 50, keep_loglik = 0, 
#>     random_parameters = "fixed", show_stan_warnings = TRUE, n_iterations = 200, 
#>     n_chains = 4, n_cores = 4, max_tree_depth = 10, adapt_delta = 0.8, 
#>     lkj_shape_prior = 4, ...) 
#> NULL
\end{Soutput}
\end{Schunk}

The main arguments are briefly explained below:

\begin{itemize}
\item
  \code{formula}: Formula for the model to be estimated as described in
  Section \{\#formula\}.
\item
  \code{data} The (\(IxJ\)) data to be used in estimation as described
  above.
\item
  \code{weights} An optional vector of sampling or frequency weights.
\item
  \code{model} A string indicating which model specification to
  estimate. The four options are presented below:

  \begin{itemize}
  \tightlist
  \item
    ``alpha'': \(\alpha\)-profile with all \(\gamma_k\) parameters fixed
    equal to 1 (Equation (\ref{eq:alpha})).
  \item
    ``gamma'': \(\gamma\)-profile with one estimated \(\alpha_1\) and
    all non-numeraire \(\alpha_k\) parameters equal to 0 (Equation
    (\ref{eq:gamma})).
  \item
    ``hybrid'': hybrid-profile with a single estimated \(\alpha\)
    parameter (i.e.~\(\alpha_1 = \alpha_k = \alpha\)) (Equation
    (\ref{eq:hybrid})).
  \item
    ``hybrid0'': hybrid-profile with all \(\alpha\) parameters fixed
    equal to 1e-3 (Equation (\ref{eq:hybrid})).
  \item
    ``kt\_ee'': Environmental economics version of KT model (Equation
    (\ref{eq:util_kt_ee})).
  \end{itemize}
\item
  \code{n\_classes} The number of latent classes. Note that the LC model
  is automatically estimated as long as the prespecified number of
  classes is set greater than 1.
\item
  \code{gamma\_ascs} Indicator to include alternative-specific gammas
  parameters.
\item
  \code{psi\_ascs} Whether to include alternative-specific psi
  parameters. The first alternative is used as the reference category.
  Only specify to 1 for MDCEV models.
\item
  \code{fixed\_scale1} Whether to fix the scale parameter at 1.
\item
  \code{trunc\_data} Whether the estimation should be adjusted for
  truncation of non-numeraire alternatives. This option is useful if the
  data only includes individuals with positive non-numeraire consumption
  levels such as recreation data collected on-site. To account for the
  truncation of consumption, the likelihood is normalized by one minus
  the likelihood of observing zero consumption for all non-numeraire
  alternatives (i.e.~likelihood of positive consumption) following
  Englin, Boxall and Watson (1998) and von Haefen (2003).
\item
  \code{seed} Random seed.
\item
  \code{algorithm} Either ``Bayes'' for Bayesian estimation or ``MLE''
  for maximum likelihood estimation. The MLE algorithm uses the
  Limited-memory BFGS which approximates the
  Broyden--Fletcher--Goldfarb--Shanno (BFGS) algorithm but uses less
  computer memory.
\item
  \code{flat\_priors} indicator if completely uninformative priors
  should be specified. Defaults to 1 if MLE used and 0 if Bayes used. If
  using MLE and set flat\_priors = 0, penalized MLE is used and the
  optimizing objective is augmented with the priors.
\item
  \code{print\_iterations} Whether to print intermediate iteration
  information or not.
\item
  \code{std\_errors} Compute standard errors using the delta method
  (``deltamethod'') or multivariate normal draws (``mvn''). The default
  is ``deltamethod''. Note that mvn parameter draws should be used to
  incorporate parameter uncertainty for demand and welfare simulation.
  For maximum likelihood estimation only.
\item
  \code{n\_draws} The number of multivariate normal draws for standard
  error calculations if ``mvn'' is specified.
\item
  \code{initial.parameters} The default for fixed and random parameter
  specifications is to use random starting values. For LC models, the
  default is to use slightly adjusted MLE point estimates from the
  single class model. Initial parameter values should be included in a
  named list. For example, the LC ``hybrid'' specification initial
  parameters can be specified as: initial.parameters = list(psi =
  array(0, dim = c(K, num\_psi)), gamma = array(1, dim = c(K,
  num\_alt)), alpha = array(0.5, dim = c(K, 1)), scale = array(1, dim =
  c(K))) where K is the number of classes (i.e.~K = 1 is used for single
  class models), num\_psi is number of psi parameters, and num\_alt is
  number of non-numeraire alternatives.
\end{itemize}

\hypertarget{formula}{%
\subsection{Formula format}\label{formula}}

The formula is used to incorporate alternative-specific variables and
individual-specific characteristics into the \(\psi_k\) parameters, the
membership equation of the LC-KT models, and \(\phi_k\) parameters for
the KT-EE specification. By default, alternative-specific constants
(ASCs) for all non-numeraire alternatives are included in the \(\psi_k\)
and \(\gamma_k\) parameters. For the \(\psi_k\), the first ASC is fixed
at 0 due to identification concerns. They can be omitted using the
\code{psi\_ascs = 0} and \code{gamma\_ascs = 0} arguments. Furthermore,
the \(\gamma_k\), \(\alpha_k\), and \(\sigma\) parameters cannot include
alternative- or individual specific variables besides ASCs.

The formula is divided in three parts, separated by the symbol \code{|}
and is based on the R package \pkg{Formula} \citep{zeileisextended2010}.
The first part is reserved for the \(z_k\) variables in \(\psi_k\) as in
Equation (\ref{eq:psi}), excluding ASCs. These can include
alternative-specific and individual-specific variables. Interaction
terms between variables can be included using the normal \pkg{Formula}
syntax of \code{z1:z2}. This is particularly useful for creating
interaction terms to incorporate observed preference heterogeneity for
alternative-specific variables and individual-specific characteristics.

For a model with only ASCs in \(\psi_k\), the formula can be specified
as

\begin{Schunk}
\begin{Sinput}
f1 = ~ 0
\end{Sinput}
\end{Schunk}

We can add individual-specific variables to the \(\psi_k\) parameters as
follows

\begin{Schunk}
\begin{Sinput}
f2 = ~ university + ageindex
\end{Sinput}
\end{Schunk}

Alternative-specific variables such as \code{z1} and \code{z2} can be
included in the same way such as

\begin{Schunk}
\begin{Sinput}
f2 = ~ z1 + z2
\end{Sinput}
\end{Schunk}

The second part corresponds to individual-specific characteristics that
enter in the probability assignment in models with latent classes. The
formula will automatically include a constant in the membership equation
but this can be omitted if \code{-1} is used in the formula. For
example, a LC model with no alternative-specific variables in the
\(psi_k\) parameters and \code{university}, \code{ageindex} and a
constant determine the class membership can be specified as

\begin{Schunk}
\begin{Sinput}
f3 = ~ 0 | university + ageindex
\end{Sinput}
\end{Schunk}

The third part is reserved for the \(q_k\) variables included in the
\(\phi_k\) parameters in the KT-EE model specification ( Equation
\ref{eq:util_kt_ee})) as in Equation (\ref{eq:phi}). For example, if
there was an alternative-specific variable named `q1', it can be
included as below

\begin{Schunk}
\begin{Sinput}
f4 = ~ 0 | 0 | q1
\end{Sinput}
\end{Schunk}

\hypertarget{estimating-kt-using-maximum-likelihood-techniques}{%
\subsubsection{Estimating KT using maximum likelihood
techniques}\label{estimating-kt-using-maximum-likelihood-techniques}}

We estimate a KT model by first calling \code{mdcev.data} on the
\textbf{Recreation} data. For these examples we are going to use a
subset of 200 individuals from the data.

\begin{Schunk}
\begin{Sinput}
data_model <- mdcev.data(data_rec, subset = id <= 200,
                       id.var = "id",
                       alt.var = "alt",
                       choice = "quant")  
\end{Sinput}
\begin{Soutput}
#> Sorting data by id.var then alt...
\end{Soutput}
\begin{Soutput}
#> Checking data...
\end{Soutput}
\begin{Soutput}
#> Data is good
\end{Soutput}
\end{Schunk}

We might think that older people prefer gardening to other activities
and so we can include an interaction term between the activity
\code{garden} and the variable \code{ageindex}. There are no
alternative-specific variables besides constant terms to include in
\(\psi\) and therefore the formula can be specified as

\begin{Schunk}
\begin{Sinput}
data_model$age_garden = ifelse(data_model$alt == "garden",
                               data_model$ageindex,0)
f5 = ~ age_garden
\end{Sinput}
\end{Schunk}

We specify the \(\gamma\)-profile of the MDCEV model specification where
a single \(\alpha_1\) is estimated for the numeraire alternative and all
non-numeraire alternatives are fixed at zero by setting
\code{model = "gamma"}. We use maximum likelihood estimation by setting
\code{algorithm = "MLE"}.

The syntax for the model is the following:

\begin{Schunk}
\begin{Sinput}
mdcev_mle <- mdcev(~ age_garden,
                  data = data_model,
                  model = "gamma",
                  algorithm = "MLE",
                  print_iterations = FALSE)
\end{Sinput}
\begin{Soutput}
#> Using MLE to estimate KT model
\end{Soutput}
\end{Schunk}

Setting \code{print\_iterations = TRUE} will print out intermediate
iteration results as the model converges.

The output of the function can be accessed by calling summary.

\begin{Schunk}
\begin{Sinput}
summary(mdcev_mle)
\end{Sinput}
\begin{Soutput}
#> Model run using rmdcev for R, version 1.2.0 
#> Estimation method                : MLE
#> Model type                       : gamma specification
#> Number of classes                : 1
#> Number of individuals            : 200
#> Number of non-numeraire alts     : 17
#> Estimated parameters             : 36
#> LL                               : -5119.11
#> AIC                              : 10310.21
#> BIC                              : 10428.95
#> Standard errors calculated using : Delta method
#> Exit of MLE                      : successful convergence
#> Time taken (hh:mm:ss)            : 00:00:0.53
#> 
#> Average consumption of non-numeraire alternatives:
#>          beach        birding        camping        cycling           fish 
#>           6.70          12.75           2.60           7.89           4.00 
#>         garden           golf         hiking     hunt_birds     hunt_large 
#>          23.18           5.42          41.62           0.58           1.03 
#>      hunt_trap hunt_waterfowl     motor_land    motor_water          photo 
#>           0.80           0.24           5.92           3.53          11.00 
#>      ski_cross       ski_down 
#>           3.12           1.85 
#> 
#> Parameter estimates --------------------------------  
#>                      Estimate Std.err z.stat
#> psi_birding            -0.762   0.113  -6.75
#> psi_camping            -0.534   0.115  -4.64
#> psi_cycling            -0.455   0.110  -4.13
#> psi_fish               -0.162   0.116  -1.39
#> psi_garden             -0.537   0.176  -3.05
#> psi_golf                0.553   0.112   4.94
#> psi_hiking             -0.038   0.107  -0.36
#> psi_hunt_birds         -1.034   0.194  -5.33
#> psi_hunt_large         -0.234   0.160  -1.46
#> psi_hunt_trap          -1.280   0.208  -6.16
#> psi_hunt_waterfowl     -0.886   0.254  -3.49
#> psi_motor_land          0.119   0.126   0.94
#> psi_motor_water         0.458   0.115   3.98
#> psi_photo               0.011   0.105   0.11
#> psi_ski_cross          -1.164   0.122  -9.54
#> psi_ski_down            0.229   0.134   1.71
#> psi_age_garden          0.513   0.155   3.31
#> gamma_beach             8.665   1.457   5.95
#> gamma_birding          22.363   4.944   4.52
#> gamma_camping           7.546   1.482   5.09
#> gamma_cycling          16.180   3.115   5.19
#> gamma_fish             11.830   2.276   5.20
#> gamma_garden           17.762   2.711   6.55
#> gamma_golf             11.080   2.393   4.63
#> gamma_hiking           17.470   2.873   6.08
#> gamma_hunt_birds        9.667   3.686   2.62
#> gamma_hunt_large       12.563   3.590   3.50
#> gamma_hunt_trap        12.727   5.663   2.25
#> gamma_hunt_waterfowl    7.735   4.165   1.86
#> gamma_motor_land       16.273   4.008   4.06
#> gamma_motor_water      11.245   2.351   4.78
#> gamma_photo            14.475   2.634   5.50
#> gamma_ski_cross        10.362   2.387   4.34
#> gamma_ski_down          9.056   2.405   3.77
#> alpha_num               0.667   0.008  83.43
#> scale                   0.607   0.027  22.47
#> Note: All non-numeraire alpha's fixed to 0.
\end{Soutput}
\end{Schunk}

The summary includes overall model and estimation information and the
parameter estimates. All parameters have been transformed to their
original form.\footnote{\(\gamma_k = exp(\gamma^*_k)\),
  \(\alpha_1 = exp(\alpha^*_1)/(1 + exp(\alpha^*_1))\), and
  \(\sigma = exp(\sigma^*)\), where \(\gamma^*_k\), \(\alpha^*_1\), and
  \(\sigma^*\) are estimated but the transformed parameters are returned
  to users.} Interpreting the parameter estimates of KT models directly
is challenging due to the non-linearities implied by the utility
function and the partial confounding of \(\alpha_k\) and \(\gamma_k\)
parameters (see Bhat (2008) for a in-depth discussion). Examining the
\(\psi_k\) parameters first which represent the marginal utility when
consumption is zero, we can see that relative to the beach recreation
activity (i.e.~the omitted reference category), hunting and trapping and
cross country skiing have the largest negative ASCs suggesting these
activities are less preferred starting from zero consumption levels. The
interaction parameter between age and gardening is positive and
significant suggesting that older people gain a higher utility from
gardening compared to younger people. Because all non-numeraire
\(\alpha\) parameters are fixed at zero, the \(\gamma_k\) parameters can
be interpreted as capturing satiation and these satiation effects are
lowest for the activities with the highest \(\gamma_k\) parameter values
such as birding, cycling, and motorized land vehicles. The \(\alpha_1\)
is estimated to be less than 1 which also implies satiation in the
numeraire good. \citet{bhatmultiple2008, lloyd-smithdecoupling2019}
provide empirical applications of this model.

In the next example, we estimate the \(\alpha\)-profile of the MDCEV
utility function by changing the model argument to \code{"alpha"}.

\begin{Schunk}
\begin{Sinput}
mdcev_mle <- mdcev(~ age_garden,
                   data = data_model,
                   model = "alpha",
                   algorithm = "MLE",
                   print_iterations = FALSE)
\end{Sinput}
\end{Schunk}

\begin{Schunk}
\begin{Sinput}
summary(mdcev_mle)
\end{Sinput}
\begin{Soutput}
#> Model run using rmdcev for R, version 1.2.0 
#> Estimation method                : MLE
#> Model type                       : alpha specification
#> Number of classes                : 1
#> Number of individuals            : 200
#> Number of non-numeraire alts     : 17
#> Estimated parameters             : 36
#> LL                               : -5354.33
#> AIC                              : 10780.67
#> BIC                              : 10899.41
#> Standard errors calculated using : Delta method
#> Exit of MLE                      : successful convergence
#> Time taken (hh:mm:ss)            : 00:00:0.59
#> 
#> Average consumption of non-numeraire alternatives:
#>          beach        birding        camping        cycling           fish 
#>           6.70          12.75           2.60           7.89           4.00 
#>         garden           golf         hiking     hunt_birds     hunt_large 
#>          23.18           5.42          41.62           0.58           1.03 
#>      hunt_trap hunt_waterfowl     motor_land    motor_water          photo 
#>           0.80           0.24           5.92           3.53          11.00 
#>      ski_cross       ski_down 
#>           3.12           1.85 
#> 
#> Parameter estimates --------------------------------  
#>                      Estimate Std.err z.stat
#> psi_birding            -0.820   0.115  -7.13
#> psi_camping            -0.582   0.117  -4.97
#> psi_cycling            -0.500   0.111  -4.51
#> psi_fish               -0.208   0.117  -1.77
#> psi_garden             -0.480   0.176  -2.73
#> psi_golf                0.492   0.114   4.32
#> psi_hiking              0.127   0.109   1.17
#> psi_hunt_birds         -1.121   0.199  -5.63
#> psi_hunt_large         -0.309   0.164  -1.88
#> psi_hunt_trap          -1.359   0.213  -6.38
#> psi_hunt_waterfowl     -0.976   0.261  -3.74
#> psi_motor_land          0.040   0.129   0.31
#> psi_motor_water         0.396   0.117   3.39
#> psi_photo              -0.030   0.105  -0.29
#> psi_ski_cross          -1.229   0.125  -9.83
#> psi_ski_down            0.158   0.138   1.14
#> psi_age_garden          0.494   0.156   3.17
#> alpha_num               0.658   0.008  82.21
#> alpha_beach             0.593   0.040  14.83
#> alpha_birding           0.720   0.038  18.94
#> alpha_camping           0.596   0.049  12.16
#> alpha_cycling           0.700   0.039  17.94
#> alpha_fish              0.660   0.043  15.34
#> alpha_garden            0.647   0.030  21.55
#> alpha_golf              0.669   0.045  14.87
#> alpha_hiking            0.595   0.030  19.82
#> alpha_hunt_birds        0.665   0.090   7.39
#> alpha_hunt_large        0.701   0.068  10.31
#> alpha_hunt_trap         0.710   0.094   7.55
#> alpha_hunt_waterfowl    0.652   0.132   4.94
#> alpha_motor_land        0.721   0.048  15.02
#> alpha_motor_water       0.663   0.047  14.12
#> alpha_photo             0.680   0.037  18.37
#> alpha_ski_cross         0.661   0.051  12.97
#> alpha_ski_down          0.658   0.060  10.96
#> scale                   0.602   0.034  17.71
#> Note: All non-numeraire gamma's fixed to 1.
\end{Soutput}
\end{Schunk}

Estimating alternative-specific \(\alpha_k\) parameters and fixing all
the non-numeraire \(\gamma\) parameters at 1, allows us to see the
heterogeneity in \(\alpha_k\) parameters across recreation activities.

The hybrid model specification of the MDCEV model where a single
\(\alpha\) is estimated for the numeraire and non-numeraire alternatives
can be estimated by setting \code{model = "hybrid"} as the next example
demonstrates.

\begin{Schunk}
\begin{Sinput}
mdcev_mle <- mdcev(~ age_garden,
                  data = data_model,
                  model = "hybrid",
                  algorithm = "MLE",
                  print_iterations = FALSE)
\end{Sinput}
\begin{Soutput}
#> Using MLE to estimate KT model
\end{Soutput}
\end{Schunk}

\begin{Schunk}
\begin{Sinput}
summary(mdcev_mle)
\end{Sinput}
\begin{Soutput}
#> Model run using rmdcev for R, version 1.2.0 
#> Estimation method                : MLE
#> Model type                       : hybrid specification
#> Number of classes                : 1
#> Number of individuals            : 200
#> Number of non-numeraire alts     : 17
#> Estimated parameters             : 36
#> LL                               : -5230.91
#> AIC                              : 10533.81
#> BIC                              : 10652.55
#> Standard errors calculated using : Delta method
#> Exit of MLE                      : successful convergence
#> Time taken (hh:mm:ss)            : 00:00:0.64
#> 
#> Average consumption of non-numeraire alternatives:
#>          beach        birding        camping        cycling           fish 
#>           6.70          12.75           2.60           7.89           4.00 
#>         garden           golf         hiking     hunt_birds     hunt_large 
#>          23.18           5.42          41.62           0.58           1.03 
#>      hunt_trap hunt_waterfowl     motor_land    motor_water          photo 
#>           0.80           0.24           5.92           3.53          11.00 
#>      ski_cross       ski_down 
#>           3.12           1.85 
#> 
#> Parameter estimates --------------------------------  
#>                      Estimate Std.err z.stat
#> psi_birding            -0.783   0.081  -9.67
#> psi_camping            -0.570   0.082  -6.95
#> psi_cycling            -0.487   0.078  -6.25
#> psi_fish               -0.206   0.083  -2.48
#> psi_garden             -0.580   0.128  -4.53
#> psi_golf                0.565   0.080   7.06
#> psi_hiking             -0.285   0.076  -3.74
#> psi_hunt_birds         -0.832   0.137  -6.07
#> psi_hunt_large         -0.095   0.113  -0.84
#> psi_hunt_trap          -1.029   0.146  -7.05
#> psi_hunt_waterfowl     -0.524   0.178  -2.94
#> psi_motor_land          0.172   0.090   1.91
#> psi_motor_water         0.449   0.082   5.48
#> psi_photo              -0.102   0.074  -1.38
#> psi_ski_cross          -1.112   0.087 -12.78
#> psi_ski_down            0.346   0.095   3.64
#> psi_age_garden          0.312   0.112   2.79
#> gamma_beach             2.197   0.445   4.94
#> gamma_birding           5.721   1.484   3.85
#> gamma_camping           2.668   0.649   4.11
#> gamma_cycling           5.742   1.306   4.40
#> gamma_fish              4.162   1.008   4.13
#> gamma_garden            4.780   0.910   5.25
#> gamma_golf              3.446   0.873   3.95
#> gamma_hiking            3.313   0.719   4.61
#> gamma_hunt_birds        3.701   1.696   2.18
#> gamma_hunt_large        5.533   1.922   2.88
#> gamma_hunt_trap         4.583   2.434   1.88
#> gamma_hunt_waterfowl    3.268   2.054   1.59
#> gamma_motor_land        5.691   1.642   3.47
#> gamma_motor_water       3.940   1.011   3.90
#> gamma_photo             4.725   1.012   4.67
#> gamma_ski_cross         3.593   0.994   3.61
#> gamma_ski_down          3.264   1.026   3.18
#> alpha                   0.648   0.005 129.53
#> scale                   0.431   0.014  30.78
#> Note: Alpha parameter is equal for all alternatives.
\end{Soutput}
\end{Schunk}

The same number of parameters are estimated in all three models and the
log-likelihood is highest for the \(\gamma\)-profile specification. The
ease of estimating different MDCEV model specifications can be used to
compare models quickly and help the analyst pick their preferred
specification for each empirical application.

We can also estimate the KT-EE specification by changing the formula
call and the model call to \code{"kt\_ee"}.

\begin{Schunk}
\begin{Sinput}
kt_mle <- mdcev(~ age_garden | 0 | 0,
                   data = data_model,
                   model = "kt_ee",
                   algorithm = "MLE",
                   print_iterations = FALSE)
\end{Sinput}
\end{Schunk}

\begin{Schunk}
\begin{Sinput}
summary(kt_mle)
\end{Sinput}
\begin{Soutput}
#> Model run using rmdcev for R, version 1.2.0 
#> Estimation method                : MLE
#> Model type                       : kt_ee specification
#> Number of classes                : 1
#> Number of individuals            : 200
#> Number of non-numeraire alts     : 17
#> Estimated parameters             : 20
#> LL                               : -5360.46
#> AIC                              : 10760.93
#> BIC                              : 10826.89
#> Standard errors calculated using : Delta method
#> Exit of MLE                      : successful convergence
#> Time taken (hh:mm:ss)            : 00:00:0.3
#> 
#> Average consumption of non-numeraire alternatives:
#>          beach        birding        camping        cycling           fish 
#>           6.70          12.75           2.60           7.89           4.00 
#>         garden           golf         hiking     hunt_birds     hunt_large 
#>          23.18           5.42          41.62           0.58           1.03 
#>      hunt_trap hunt_waterfowl     motor_land    motor_water          photo 
#>           0.80           0.24           5.92           3.53          11.00 
#>      ski_cross       ski_down 
#>           3.12           1.85 
#> 
#> Parameter estimates --------------------------------  
#>                      Estimate Std.err z.stat
#> psi_age_garden          0.396   0.110   3.60
#> gamma_beach            10.546   1.083   9.74
#> gamma_birding          22.269   2.484   8.96
#> gamma_camping          16.201   1.778   9.11
#> gamma_cycling          16.238   1.743   9.32
#> gamma_fish             12.238   1.359   9.01
#> gamma_garden           16.644   2.166   7.68
#> gamma_golf              6.236   0.699   8.92
#> gamma_hiking           11.910   1.322   9.01
#> gamma_hunt_birds       25.821   4.428   5.83
#> gamma_hunt_large       13.798   2.019   6.83
#> gamma_hunt_trap        32.817   6.094   5.39
#> gamma_hunt_waterfowl   24.675   5.567   4.43
#> gamma_motor_land       10.400   1.281   8.12
#> gamma_motor_water       7.117   0.812   8.76
#> gamma_photo            11.153   1.183   9.43
#> gamma_ski_cross        28.684   3.201   8.96
#> gamma_ski_down          8.403   1.065   7.89
#> alpha_num               0.475   0.007  67.93
#> scale                   0.713   0.025  28.54
\end{Soutput}
\end{Schunk}

This model does not include ASCs in the \(psi_k\) parameters due to
concerns about weak complementarity.

\hypertarget{estimating-kt-using-bayesian-techniques}{%
\subsubsection{Estimating KT using Bayesian
techniques}\label{estimating-kt-using-bayesian-techniques}}

The exact same models can be fit using Bayesian estimation by changing
the algorithm call to \code{"Bayes"}. Bayesian estimation is implemented
using the Stan programming language \citep{carpenterstan2017}. The
Bayesian framework requires careful choice of priors for the parameters,
especially in data sparse contexts. The specific prior distributions for
the fixed parameter specifications is presented below. The user has the
ability to change the standard deviation and shape of these priors
through these options in the \code{mdcev} function:

\begin{itemize}
\tightlist
\item
  \code{prior\_psi\_sd} standard deviation for normal prior with mean 0.
\item
  \code{prior\_phi\_sd} standard deviation for normal prior with mean 0.
\item
  \code{prior\_gamma\_sd} standard deviation for half-normal prior with
  mean 1.
\item
  \code{prior\_alpha\_shape} shape parameter for beta distribution.
\item
  \code{prior\_scale\_sd} standard deviation for half-normal prior with
  mean 0.
\end{itemize}

For the random parameter model specifications, the priors for the means
of all random parameters follow a normal distribution with mean 0 on the
unconstrained space.

There are also a number of further options for Bayesian estimation. For
example, the number of iterations (n\_iterations), number of chains
(n\_chains), and number of cores (n\_cores) for parallel implementation
of the chains can also be chosen. The full set of options for Bayesian
estimation are presented below.

\begin{itemize}
\item
  \code{random\_parameters} The form of the covariance matrix for the
  parameters. Options are

  \begin{itemize}
  \tightlist
  \item
    `fixed' for no random parameters,
  \item
    'uncorr for uncorrelated random parameters, or
  \item
    `corr' for correlated random parameters.
  \end{itemize}
\item
  \code{n\_iterations} The number of iterations to use in Bayesian
  estimation. The default is for the number of iterations to be split
  evenly between warmup and posterior draws. The number of warmup draws
  can be directly controlled using the warmup argument (see
  rstan::sampling)
\item
  \code{n\_chains} The number of independent Markov chains in Bayesian
  estimation.
\item
  \code{n\_cores} The number of cores used to execute the Markov chains
  in parallel in Bayesian estimation. Can set using options(mc.cores =
  parallel::detectCores()).
\item
  \code{max\_tree\_depth}
  \url{http://mc-stan.org/misc/warnings.html\#maximum-treedepth-exceeded}
\item
  \code{adapt\_delta}
  \url{http://mc-stan.org/misc/warnings.html\#divergent-transitions-after-warmup}
\item
  \code{lkj\_shape\_prior} Prior for Cholesky matrix for correlated
  random parameters.
\end{itemize}

In this example, we estimate the \(\gamma\)-profile of the MDCEV
specification using Bayesian techniques. We set the number of iterations
to 200 and use 4 independent chains across 4 cores.

\begin{Schunk}
\begin{Sinput}
mdcev_bayes <- mdcev(~ age_garden,
                        data = data_model,
                        model = "gamma",
                        algorithm = "Bayes",
                        n_iterations = 200,
                        n_chains = 4,
                        n_cores = 4,
                        print_iterations = FALSE)
\end{Sinput}
\end{Schunk}

The output of the function can be accessed by calling \code{summary}.

\begin{Schunk}
\begin{Sinput}
    summary(mdcev_bayes)
\end{Sinput}
\begin{Soutput}
#> Model run using rmdcev for R, version 1.2.0 
#> Estimation method                : Bayes
#> Model type                       : gamma specification
#> Number of classes                : 1
#> Number of individuals            : 200
#> Number of non-numeraire alts     : 17
#> Estimated parameters             : 36
#> LL                               : -5137.69
#> Number of chains                 : 4
#> Number of warmup draws per chain : 100
#> Total post-warmup sample         : 400
#> Time taken (hh:mm:ss)            : 00:00:40.03
#> 
#> Average consumption of non-numeraire alternatives:
#>          beach        birding        camping        cycling           fish 
#>           6.70          12.75           2.60           7.89           4.00 
#>         garden           golf         hiking     hunt_birds     hunt_large 
#>          23.18           5.42          41.62           0.58           1.03 
#>      hunt_trap hunt_waterfowl     motor_land    motor_water          photo 
#>           0.80           0.24           5.92           3.53          11.00 
#>      ski_cross       ski_down 
#>           3.12           1.85 
#> 
#> Parameter estimates --------------------------------  
#>                      Estimate Std.dev z.stat n_eff Rhat
#> psi_birding            -0.786   0.118  -6.69   366 1.00
#> psi_camping            -0.572   0.125  -4.57   272 1.00
#> psi_cycling            -0.489   0.118  -4.13   446 1.00
#> psi_fish               -0.196   0.135  -1.45   467 1.00
#> psi_garden             -0.553   0.180  -3.08   327 1.00
#> psi_golf                0.515   0.117   4.41   397 0.99
#> psi_hiking              0.017   0.108   0.16   531 1.00
#> psi_hunt_birds         -1.149   0.197  -5.82   478 0.99
#> psi_hunt_large         -0.322   0.168  -1.92   538 0.99
#> psi_hunt_trap          -1.413   0.223  -6.35   430 1.00
#> psi_hunt_waterfowl     -1.039   0.273  -3.80   305 1.00
#> psi_motor_land          0.059   0.130   0.45   284 1.00
#> psi_motor_water         0.415   0.122   3.40   296 1.00
#> psi_photo               0.000   0.104   0.00   372 1.00
#> psi_ski_cross          -1.225   0.130  -9.39   427 1.00
#> psi_ski_down            0.157   0.147   1.06   495 1.00
#> psi_age_garden          0.551   0.157   3.50   407 1.00
#> gamma_beach             7.959   1.343   5.93   396 1.00
#> gamma_birding          18.287   3.367   5.43   428 1.00
#> gamma_camping           7.249   1.450   5.00   568 0.99
#> gamma_cycling          14.410   2.731   5.28   681 0.99
#> gamma_fish             11.064   2.172   5.09   414 1.00
#> gamma_garden           15.744   2.268   6.94   503 1.00
#> gamma_golf             10.126   2.093   4.84   370 1.00
#> gamma_hiking           15.319   2.387   6.42   566 0.99
#> gamma_hunt_birds        9.488   3.404   2.79   214 1.00
#> gamma_hunt_large       11.626   3.189   3.65   254 1.01
#> gamma_hunt_trap        11.425   4.214   2.71   230 1.02
#> gamma_hunt_waterfowl    8.754   3.991   2.19   200 1.02
#> gamma_motor_land       14.303   3.295   4.34   657 1.00
#> gamma_motor_water      10.573   2.139   4.94   430 1.00
#> gamma_photo            13.169   2.239   5.88   514 1.00
#> gamma_ski_cross         9.894   2.311   4.28   410 1.00
#> gamma_ski_down          8.767   2.335   3.75   433 1.00
#> alpha_num               0.668   0.008  88.60   397 1.00
#> scale                   0.651   0.029  22.72   280 1.00
#> Note: All non-numeraire alpha's fixed to 0. 
#> Note from Rstan: 'For each parameter, n_eff is a crude measure of effective sample size, and Rhat is the potential scale reduction factor on split chains (at convergence, Rhat=1)'
\end{Soutput}
\end{Schunk}

Comparing these parameter values to the maximum likelihood estimates of
the \(\gamma\)-profile MDCEV specification, the values are quite
similar. As the data set is rather small with only 200 individuals, the
priors play a role in reducing the estimates closer to 1 for the
\(\gamma_k\), but this role will lessen in larger data applications.

One benefit of using the Bayesian approach is that one can take
advantage of the postestimation commands, interactive diagnostics, and
posterior analysis in \pkg{rstan},
\href{https://mc-stan.org/bayesplot/}{\pkg{bayesplot}}
\citep{gabrybayesplot2019}, and
\href{http://mc-stan.org/shinystan/}{\pkg{shinystan}}
\citep{muthuser2018}. For example, the effective sample size reports the
estimated number of independent draws from the posterior distribution
for each parameter \citep{stan2019}. The interested reader is referred
to these packages for additional details.

\hypertarget{estimating-lc-kt-models}{%
\subsubsection{Estimating LC-KT models}\label{estimating-lc-kt-models}}

In this example, we estimate a LC-KT model using the \textbf{Recreation}
data. We set the number of classes equal to 2 and we use data on 500
individuals. We would like to include the \code{university},
\code{ageindex}, and \code{urban} in the membership equation and we
include them in the \code{formula} interface. The constant for the
membership equation is included automatically. The LC model is
automatically estimated as long as the prespecified number of classes
(\code{n\_classes}) is set greater than 1. The scale parameters are
fixed at 1 using \code{fixed\_scale1 = 1}.

\begin{Schunk}
\begin{Sinput}
data_model <- mdcev.data(data_rec, subset = id <= 500,
                       id.var = "id",
                       alt.var = "alt",
                       choice = "quant")  

mdcev_lc <- mdcev(~ 0 | university + ageindex + urban,
                  data = data_model,
                  n_classes = 2,
                  model = "gamma",
                  fixed_scale1 = 1,
                  algorithm = "MLE",
                  print_iterations = FALSE)
\end{Sinput}
\begin{Soutput}
#> Error in chol.default(-H) : 
#>   the leading minor of order 43 is not positive definite
\end{Soutput}
\end{Schunk}

\begin{Schunk}
\begin{Sinput}
summary(mdcev_lc)
\end{Sinput}
\begin{Soutput}
#> Warning in sqrt(diag(cov_mat)): NaNs produced
\end{Soutput}
\begin{Soutput}
#> Model run using rmdcev for R, version 1.2.0 
#> Estimation method                : MLE
#> Model type                       : gamma specification
#> Number of classes                : 2
#> Number of individuals            : 500
#> Number of non-numeraire alts     : 17
#> Estimated parameters             : 72
#> LL                               : -12073.55
#> AIC                              : 24291.1
#> BIC                              : 24594.55
#> Standard errors calculated using : Delta method
#> Exit of MLE                      : successful convergence
#> Time taken (hh:mm:ss)            : 00:00:7.5
#> 
#> Average consumption of non-numeraire alternatives:
#>          beach        birding        camping        cycling           fish 
#>           6.47          12.27           2.33           9.03           3.84 
#>         garden           golf         hiking     hunt_birds     hunt_large 
#>          21.97           4.44          39.52           0.64           1.14 
#>      hunt_trap hunt_waterfowl     motor_land    motor_water          photo 
#>           0.84           0.27           4.48           3.03           9.20 
#>      ski_cross       ski_down 
#>           2.36           1.49 
#> 
#> 
#> Class average probabilities:
#> class1 class2 
#>   0.85   0.15 
#> Parameter estimates --------------------------------  
#>                             Estimate Std.err z.stat
#> class1.psi_birding            -1.290   0.133  -9.70
#> class2.psi_birding            -1.154   0.133  -8.68
#> class1.psi_camping            -0.977   0.125  -7.81
#> class2.psi_camping            -0.580   0.161  -3.60
#> class1.psi_cycling            -0.777   0.113  -6.88
#> class2.psi_cycling            -0.949   0.139  -6.83
#> class1.psi_fish               -1.117   0.121  -9.23
#> class2.psi_fish                1.446   0.585   2.47
#> class1.psi_garden              0.025   1.110   0.02
#> class2.psi_garden             -0.073  10.998  -0.01
#> class1.psi_golf               -0.113   0.617  -0.18
#> class2.psi_golf                0.689   0.159   4.33
#> class1.psi_hiking              0.473   0.154   3.07
#> class2.psi_hiking              0.186   0.119   1.57
#> class1.psi_hunt_birds         -4.017   0.143 -28.09
#> class2.psi_hunt_birds          0.591   0.151   3.91
#> class1.psi_hunt_large         -4.259   0.343 -12.42
#> class2.psi_hunt_large          1.609   0.313   5.14
#> class1.psi_hunt_trap         -10.286   0.332 -30.98
#> class2.psi_hunt_trap          -0.247   0.301  -0.82
#> class1.psi_hunt_waterfowl     -3.723   0.281 -13.25
#> class2.psi_hunt_waterfowl     -0.019   0.295  -0.06
#> class1.psi_motor_land         -0.740   0.271  -2.73
#> class2.psi_motor_land          1.210   0.305   3.97
#> class1.psi_motor_water        -0.456   0.325  -1.40
#> class2.psi_motor_water         1.624   0.323   5.03
#> class1.psi_photo              -0.130   0.336  -0.39
#> class2.psi_photo              -0.724   0.288  -2.51
#> class1.psi_ski_cross          -1.783   0.278  -6.41
#> class2.psi_ski_cross          -1.438   0.344  -4.18
#> class1.psi_ski_down           -0.417   0.348  -1.20
#> class2.psi_ski_down           -0.107   0.356  -0.30
#> class1.gamma_beach             3.814   0.479   7.96
#> class2.gamma_beach             7.426   1.366   5.44
#> class1.gamma_birding          11.328   1.781   6.36
#> class2.gamma_birding           6.982   1.060   6.59
#> class1.gamma_camping           3.558   0.720   4.94
#> class2.gamma_camping           8.956   1.062   8.43
#> class1.gamma_cycling           9.239   1.649   5.60
#> class2.gamma_cycling          13.108   1.671   7.84
#> class1.gamma_fish              5.386   4.401   1.22
#> class2.gamma_fish              3.472   3.955   0.88
#> class1.gamma_garden            8.718     NaN    NaN
#> class2.gamma_garden            8.816   8.146   1.08
#> class1.gamma_golf              7.088   1.533   4.62
#> class2.gamma_golf              3.805   0.778   4.89
#> class1.gamma_hiking            6.160   0.829   7.43
#> class2.gamma_hiking           12.122   2.280   5.32
#> class1.gamma_hunt_birds        3.228   0.651   4.96
#> class2.gamma_hunt_birds        2.553   0.811   3.15
#> class1.gamma_hunt_large       11.305   5.693   1.99
#> class2.gamma_hunt_large        3.133   1.152   2.72
#> class1.gamma_hunt_trap         0.803   0.337   2.38
#> class2.gamma_hunt_trap         5.184   1.465   3.54
#> class1.gamma_hunt_waterfowl    4.955   1.522   3.26
#> class2.gamma_hunt_waterfowl    3.796   1.348   2.82
#> class1.gamma_motor_land        5.093   1.436   3.55
#> class2.gamma_motor_land        8.906   2.587   3.44
#> class1.gamma_motor_water       3.010   0.851   3.54
#> class2.gamma_motor_water       5.688   2.028   2.80
#> class1.gamma_photo             6.103   2.394   2.55
#> class2.gamma_photo            10.199   2.794   3.65
#> class1.gamma_ski_cross         4.948   1.339   3.70
#> class2.gamma_ski_cross         7.200   3.109   2.32
#> class1.gamma_ski_down          3.997   1.669   2.39
#> class2.gamma_ski_down          4.691   2.170   2.16
#> class1.alpha_num               0.672   0.009  74.68
#> class2.alpha_num               0.686   0.022  31.17
#> class2.(Intercept)            -1.100   0.479  -2.30
#> class2.university             -0.406   0.338  -1.20
#> class2.ageindex                0.186   0.364   0.51
#> class2.urban                  -0.837   0.350  -2.39
#> Note: Scale parameter fixed to 1. 
#> Note: All non-numeraire alpha's fixed to 0. 
#> Note: The membership equation parameters for class 1 are normalized to 0.
\end{Soutput}
\end{Schunk}

In this LC example, we assume that there are two types of people that
have different preferences for recreation. The probability of class
assignment depends on unobserved factors and the three sociodemographic
factors included in the membership equation with only \code{urban}
having a statistically significant effect on class probability. People
living in urban areas are less likely to be in class 2. The summary
output reports the average class probabilities as being 32\% for class 1
and 68\% for class 2. The \(\psi\) parameters across classes are similar
although there are some noticeable differences such as the hunting and
trapping preferences. The \(\gamma\) parameters, on the other hand, show
that satiation between classes is quite different.
\citet{sobhanilatent2013, kuriyamalatent2010} provide empirical
applications of these models.

If \code{initial.parameter} are not provided, the default is to use
slightly adjusted parameter estimates of the MDCEV model as starting
values when estimating the LC-MDCEV model to assist speed and
convergence issues.\footnote{In particular, the estimated \(\psi_k\) and
  \(\gamma_k\) parameters from the MDCEV model are randomly adjusted by
  0.02.} The MDCEV model output can be accessed from
\code{mdcev\_lc[["mdcev\_fit"]]} object for comparison.

\hypertarget{estimating-rp-kt-models}{%
\subsubsection{Estimating RP-KT models}\label{estimating-rp-kt-models}}

Random parameter models require defining and parameterizing the variance
covariance matrix. For uncorrelated random parameters, the diagonal
elements of the variance covariance matrix are estimated and the
off-diagonal elements are assumed to be zero. For correlated random
parameters, the variance covariance matrix is fully estimated and can be
parameterized in many ways. The \pkg{rmdcev} package defines the
variance covariance matrix in terms of Cholesky factors of the
correlation matrix and a vector of standard deviations for numerical
stability. Thus the variance covariance matrix is specified as

\begin{equation}
\sum = diag(\tau) \; x \; LL^T \; x \; diag(\tau),
\end{equation}

where \(\tau\) is a vector of standard deviations, and \(L\) is the
cholesky factors of the correlation matrix.

In this example, we estimate an uncorrelated random parameters
\(\gamma\)-specification of the MDCEV model without any \(\psi_k\)
parameters. We set the argument \code{random\_parameters = "uncorr"} to
indicate that uncorrelated random parameters will be estimated. As noted
earlier, all random parameters follow a normal distribution. We change
the \code{psi\_ascs = 0} to omit the ASCs in the \(\psi_k\) parameters.

\begin{Schunk}
\begin{Sinput}
data_model <- mdcev.data(data_rec, subset = id <= 200,
                       id.var = "id",
                       alt.var = "alt",
                       choice = "quant") 

mdcev_rp <- mdcev(~ 0,
                    data = data_model,
                    model = "gamma",
                    algorithm = "Bayes",
                    n_chains = 4,
                    psi_ascs = 0,
                    fixed_scale1 = 1,
                    n_iterations = 200,
                    random_parameters = "uncorr",
                    print_iterations = FALSE)
\end{Sinput}
\end{Schunk}

\begin{Schunk}
\begin{Sinput}
summary(mdcev_rp)
\end{Sinput}
\begin{Soutput}
#> Model run using rmdcev for R, version 1.2.0 
#> Estimation method                : Bayes
#> Model type                       : gamma specification
#> Number of classes                : 1
#> Number of individuals            : 200
#> Number of non-numeraire alts     : 17
#> Estimated parameters             : 36
#> LL                               : -5362.51
#> Random parameters                : uncorrelated random parameters
#> Number of chains                 : 4
#> Number of warmup draws per chain : 100
#> Total post-warmup sample         : 400
#> Time taken (hh:mm:ss)            : 00:01:57.22
#> 
#> Average consumption of non-numeraire alternatives:
#>          beach        birding        camping        cycling           fish 
#>           6.70          12.75           2.60           7.89           4.00 
#>         garden           golf         hiking     hunt_birds     hunt_large 
#>          23.18           5.42          41.62           0.58           1.03 
#>      hunt_trap hunt_waterfowl     motor_land    motor_water          photo 
#>           0.80           0.24           5.92           3.53          11.00 
#>      ski_cross       ski_down 
#>           3.12           1.85 
#> 
#> Parameter estimates --------------------------------  
#>                         Estimate Std.dev z.stat n_eff Rhat
#> gamma_beach                5.661   1.075   5.27   427 1.00
#> gamma_birding              8.249   2.272   3.63   479 1.00
#> gamma_camping              3.795   0.732   5.18   404 0.99
#> gamma_cycling              7.980   1.480   5.39   463 1.00
#> gamma_fish                 7.020   1.649   4.26   400 1.00
#> gamma_garden              12.635   2.048   6.17   407 1.00
#> gamma_golf                 6.506   1.605   4.05   362 1.00
#> gamma_hiking              15.176   2.188   6.94   455 1.00
#> gamma_hunt_birds           4.135   1.962   2.11   534 0.99
#> gamma_hunt_large           7.206   2.399   3.00   351 1.00
#> gamma_hunt_trap            5.689   3.753   1.52   595 1.00
#> gamma_hunt_waterfowl       4.052   2.916   1.39   706 0.99
#> gamma_motor_land           8.446   2.323   3.64   344 1.01
#> gamma_motor_water          6.649   1.644   4.05   358 1.00
#> gamma_photo                8.786   1.588   5.53   357 1.00
#> gamma_ski_cross            3.343   0.796   4.20   334 1.01
#> gamma_ski_down             4.933   1.400   3.52   537 1.00
#> alpha_num                  0.725   0.007  98.82   396 1.00
#> sd.gamma_beach             1.263   0.223   5.65   365 1.00
#> sd.gamma_birding           1.800   0.609   2.95   242 1.00
#> sd.gamma_camping           1.283   0.248   5.18   585 0.99
#> sd.gamma_cycling           1.307   0.256   5.12   331 1.00
#> sd.gamma_fish              1.262   0.236   5.36   531 1.00
#> sd.gamma_garden            1.297   0.242   5.36   361 0.99
#> sd.gamma_golf              1.565   0.475   3.30   176 1.00
#> sd.gamma_hiking            1.334   0.265   5.04   300 0.99
#> sd.gamma_hunt_birds        1.767   1.185   1.49   556 0.99
#> sd.gamma_hunt_large        1.415   0.422   3.35   776 0.99
#> sd.gamma_hunt_trap         2.472   2.752   0.90   260 1.02
#> sd.gamma_hunt_waterfowl    2.622   2.382   1.10   383 1.00
#> sd.gamma_motor_land        1.544   0.522   2.96   395 1.00
#> sd.gamma_motor_water       1.407   0.321   4.38   422 1.00
#> sd.gamma_photo             1.275   0.253   5.05   248 1.00
#> sd.gamma_ski_cross         1.509   0.472   3.20   285 1.00
#> sd.gamma_ski_down          1.500   0.448   3.35   381 1.01
#> sd.alpha_num               0.515   0.010  50.79   365 1.00
#> Note: Scale parameter fixed to 1. 
#> Note: All non-numeraire alpha's fixed to 0. 
#> Note from Rstan: 'For each parameter, n_eff is a crude measure of effective sample size, and Rhat is the potential scale reduction factor on split chains (at convergence, Rhat=1)'
\end{Soutput}
\end{Schunk}

The results show the means of the random parameters followed by the
estimated standard deviations. The standard deviations that are
estimated to be different from zero suggest there is heterogeneity in
preference parameters. The correlated random parameters specification
can be estimated by setting \code{random\_parameters = "corr"}.
\citet{bhathousehold2006} provide an empirical application of this type
of model.

\hypertarget{computational-and-estimation-issues}{%
\subsubsection{Computational and estimation
issues}\label{computational-and-estimation-issues}}

KT models are notoriously tricky to estimate relative to standard
discrete choice models. This section provides some guidance for
estimating these models and common convergence issues:

\begin{itemize}
\item
  \textbf{Starting values:} Model parameter estimates can be sensitive
  to starting values, especially the more complex LC-KT specification.
  Users should use several different initial parameter values for model
  estimation to ensure robust results and a global maxima is found
  rather than a local maxima. The default behaviour for LC-KT models is
  to use KT parameters as starting values. In practice the author has
  found this to be quite effective at finding global maxima. However,
  users are encouraged to use random starting values as a robustness
  check.
\item
  \textbf{Identification issues:} Depending on the model specification
  and included variables the model may not be properly identified. If
  you receive an error such as
  \code{Error in chol.default(-H) : the leading minor of order 9 is not positive definite},
  this usually suggests an identification issue. Users should double
  check all variables included in the model are appropriate. One
  solution is to start with a simpler model first and then slowly add
  variables to help locate any problematic variables.
\item
  \textbf{Parameter estimates near boundaries:} Interpret models with
  parameter estimates that are near the boundaries (e.g.~\(\alpha\)
  close to 1) with caution. Users are recommended to re-estimate the
  model with starting values far from this boundary.
\item
  \textbf{Bayesian estimation}: For models estimated using Bayesian
  estimation, users should consult the \pkg{rstan} User Guide for
  additional guidance on model estimation options and postestimation
  checks \citep{stan2019}. Additional information is available by typing
  \code{help(rstan)}.
\end{itemize}

\hypertarget{simulating-kt-demand-and-welfare-scenarios}{%
\subsection{Simulating KT demand and welfare
scenarios}\label{simulating-kt-demand-and-welfare-scenarios}}

The \pkg{rmdcev} package includes simulation functions for calculating
welfare measures and forecasting demand under alternative policy
scenarios. The overall approach used for simulation is first introduced
and then code examples are given.

\hypertarget{overview-of-simulation-steps}{%
\subsubsection{Overview of simulation
steps}\label{overview-of-simulation-steps}}

Once the model parameters are estimated, there are two steps to
simulation in KT models. In the first step we draw simulated values for
the unobserved heterogeneity term (\(\varepsilon\)) using Monte Carlo
techniques. The second step uses these error draws, the previously
estimated model parameters, and the underlying data to calculate
Marshallian demands for forecasting or Hicksian demands for welfare
analysis. These two steps are described below.

\textbf{Step 1: simulating unobserved heterogeneity}

Monte Carlo simulation techniques can be employed to draw simulated
values of the unobserved heterogeneity (\(\varepsilon\)) using either
unconditional or conditional draws.

\begin{enumerate}
\def\labelenumi{\arabic{enumi}.}
\tightlist
\item
  Unconditional error draws: draw from the entire distribution of
  unobserved heterogeneity using the following formula
\end{enumerate}

\begin{equation}
\varepsilon_{k} = -log(-log(draw(0,1))) * \sigma,
\end{equation}

where \(draw(0,1)\) is a draw between 0 and 1 and \(\sigma\) is the
scale parameter. \pkg{rmdcev} allows errors to be drawn using uniform
draws or the Modified Latin Hypercube Sampling algorithm
\citep{hesson2006}.

\begin{enumerate}
\def\labelenumi{\arabic{enumi}.}
\setcounter{enumi}{1}
\tightlist
\item
  Conditional error draws: draw errors terms to reflect behaviour and
  dependent on whether alternative is consumed or not
  \citep{vonhaefenincorporating2003, vonhaefenestimation2004}:
\end{enumerate}

\begin{itemize}
\tightlist
\item
  If \(x_k>0\), set \(\varepsilon_k = (V_1 - V_k)/ \sigma\) for the
  MDCEV specifications where \(V_1\) and \(V_k\) depend on the model
  specification as detailed above. If using the environmental economics
  KT model specification (``kt\_ee''), set \(\varepsilon_k = g_k(.)\)
  from Equation (\ref{eq:kt_g}).
\item
  If \(x_k=0\), \(\varepsilon_k < (V_1 - V_k)/ \sigma\) and simulate
  \(\varepsilon_k\) from the truncated type I extreme value distribution
  such that
\end{itemize}

\begin{equation}
\varepsilon_k = -log(-log(draw(0, 1) * exp(-exp(\frac{V_1 - V_k}{\sigma})))) * \sigma \; \mbox{for the MDCEV specifications, or}
\varepsilon_k = -log(-log(draw(0, 1) * exp(-exp(-g_k(.))))) * \sigma \: \mbox{for the KT-EE specification.}
\end{equation}

In the conditional error draw approach, we normalize
\(\varepsilon_1=0\).

The main differences between these two error draw approaches is that in
the conditional approach, errors are drawn such that the model perfectly
predicts the observed consumption patterns in the baseline state
\citep{vonhaefenkuhn-tucker2005}. The conditional approach uses observed
behaviour by individuals to characterize unobserved heterogeneity and
can be useful for scenario simulation as the baseline matches observed
behavior. This is especially true if poor in-sample behavioral
predictions is found using the unconditional approach
\citep{vonhaefenincorporating2003}. The unconditional approach draws all
errors based on distributional assumptions and is necessary for
out-of-sample forecasting. If the model correctly specifies the data
generating process, the sample means of the conditional and
unconditional approaches should converge in expectation. Another
difference between the two approaches is that the unconditional approach
uses more computation time as there is a need to calculate consumption
patterns in the baseline state as well as simulate the entire
distribution of unobserved heterogeneity.

\citep{vonhaefenincorporating2003}

\textbf{Step 2: Calculating welfare measures and demand forecasts}

With the error draws in hand, the second step is to simulate demand or
welfare changes. Compared to welfare measures in discrete choice models,
welfare calculation in KT models is more challenging because of the two
KT conditions in Equation (\ref{eq:kt_conditions}). For a given policy
scenario, a priori, we do not know which alternatives have a positive or
zero consumption level. \pkg{rmdcev} implements the
\citet{pinjaricomputationally2011} efficient demand forecasting routine
for simulating demand behaviour for MDCEV models which relies on
calculating Marshallian demands. For welfare calculations, we need to
calculate the expenditure function in Equation (\ref{eq:welfare}) which
relies on Hicksian demands. These are calculated using the approach
described by \citet{lloydsmithnew2018} and the \pkg{rmdcev} extends
these approaches to the environmental economics KT model specifications.
The demand and welfare simulation approaches share a lot of
commonalities and thus only the approach used for welfare calculations
are fully described in the appendix. The specific steps for demand
simulation is explained in-depth in \citet{pinjaricomputationally2011}
and the interested reader is encouraged to read Section 4 of the paper
for the exact details.

\hypertarget{welfare-analysis}{%
\subsubsection{Welfare analysis}\label{welfare-analysis}}

In \pkg{rmdcev}, the functions for welfare and demand simulation have
been divided into 3 steps to allow users to parallelize operations as
necessary.

We first estimate the model using \code{mdcev} and we set
\code{std\_errors = "mvn"} to generate multivariate normal draws.

\begin{Schunk}
\begin{Sinput}
mdcev_mle <- mdcev(~0,
                  data = data_model,
                  model = "hybrid",
                  algorithm = "MLE",
                  std_errors = "mvn",
                  print_iterations = FALSE)
\end{Sinput}
\begin{Soutput}
#> Using MLE to estimate KT model
\end{Soutput}
\end{Schunk}

\hypertarget{define-policy-scenarios}{%
\subparagraph{1. Define policy
scenarios}\label{define-policy-scenarios}}

In the first step, we define the number of alternative policy scenarios
to use in simulation and then specify changes to the \(\psi\) variables
and prices of alternatives. The CreateBlankPolicies function has been
created to easily set-up the required lists for the simulation. These
policies can then be manually edited according to the specific policy
scenario. For prices, \pkg{rmdcev} is set up to accept additive changes
in prices that impact all individuals the same. For the \(\psi\) and
\(\phi\) variable changes, the package is set up to accept any new
values for these variables. Depending on the number of individuals and
number of policies, the generated policies list can be quite large. If
the user is only interested in assessing price changes, then you can use
\code{price\_change\_only = TRUE} which ensures duplicate \(\psi\) and
\(\phi\) data is not created.

In this example, we are interested in two separate policies. The first
policy increases the costs of all recreation activities by \$1 and the
second policy increases the cost of all four hunting activities by \$10.
The policy set-up for these two scenarios is demonstrated below.

\begin{Schunk}
\begin{Sinput}
nalts <- mdcev_mle$stan_data[["J"]]
npols <- 2

policies<-  CreateBlankPolicies(npols = npols,
                                model = mdcev_mle,
                                price_change_only = TRUE)

policies$price_p[[1]] <- c(0, rep(1, nalts))
policies$price_p[[2]][10:13] <- rep(10, 4)
\end{Sinput}
\end{Schunk}

For policy scenarios that involve changes in the \(\psi\) or \(\phi\)
variables, the user can change the \code{dat\_psi} or \code{dat\_phi}
list of the \code{policies} object. For example, the following code will
increase the value of the third \(\psi\) variable by 20\% in policy
scenario 1.

\begin{Schunk}
\begin{Sinput}
policies$dat_psi[[1]][3] <- policies$dat_psi[[1]][3]*1.2
\end{Sinput}
\end{Schunk}

\hypertarget{prepare-simulation-data}{%
\subparagraph{2. Prepare simulation
data}\label{prepare-simulation-data}}

The second step is to combine the parameter estimates, data, and policy
scenarios into a data format for simulation. The
\code{PrepareSimulationData} function uses the model fit and the user
defined policy scenarios to create this specific data format. This
function separates the output into individual-specific data
(\code{df\_indiv}), data common to all individuals (\code{df\_common}),
and simulation options (\code{sim\_options}).

\begin{Schunk}
\begin{Sinput}
df_sim <- PrepareSimulationData(mdcev_mle, policies)
\end{Sinput}
\end{Schunk}

\hypertarget{simulate-mdcev-model}{%
\subparagraph{3. Simulate MDCEV model}\label{simulate-mdcev-model}}

The third step is to simulate the policy scenario using the formatted
data and the \code{mdcev.sim} function. The specific steps for the
simulation algorithms are described in Appendix A. The user chooses the
type of error draws (unconditional or conditional as described above),
the number of error draws, and whether to simulate the demand or welfare
changes.

\begin{Schunk}
\begin{Sinput}
welfare <- mdcev.sim(df_sim$df_indiv,
                     df_common = df_sim$df_common,
                     sim_options = df_sim$sim_options,
                     cond_err = 1,
                     nerrs = 25,
                     sim_type = "welfare")
\end{Sinput}
\begin{Soutput}
#> Using hybrid approach in simulation...
\end{Soutput}
\begin{Soutput}
#> 
#> 3.00e+05simulations finished in0.07minutes.(74074per second)
\end{Soutput}
\begin{Sinput}
summary(welfare)
\end{Sinput}
\begin{Soutput}
#> # A tibble: 2 x 5
#>   policy    mean std.dev `ci_lo2.5%` `ci_hi97.5%`
#>   <chr>    <dbl>   <dbl>       <dbl>        <dbl>
#> 1 policy1 -125.    0.116      -125.        -125. 
#> 2 policy2  -20.2   0.418       -20.9        -19.4
\end{Soutput}
\end{Schunk}

The output of the \code{mdcev.sim} for welfare analysis is an object of
class \code{mdcev.sim} which contains a list of matrices where each
element of the list is for an individual and the matrix consists of rows
for each policy scenario and columns for each parameter simulation.

The summary function computes summary statistics across all individuals.
For example, the average welfare change for a \$1 daily increase in all
recreation costs is -\$125.

The reason these last two steps are separate is to allow users to
parallelize the simulation step as the last step can be computationally
intensive. The number of simulations is a multiplicative function of the
number of individuals, number of policies, number of parameter estimate
simulations, and the number of error draws (\(I\) x \(npols\) x
\(nsims\) x \(nerrs\)). Even for modestly sized data, the total number
of simulations can easily reach well into the millions or billions. All
simulations are conducted at the individual level which allows the user
to easily parallelize the \code{mdcev.sim} function using the
\pkg{parallel} package or similar packages.

\hypertarget{demand-forecasting}{%
\subsubsection{Demand forecasting}\label{demand-forecasting}}

This section demonstrates the demand forecasting capabilities of
\pkg{rmdcev}. Please refer to the previous section for an overview of
the three steps to simulation.

\begin{Schunk}
\begin{Sinput}
policies <- CreateBlankPolicies(npols = 2, model = mdcev_mle)

policies$price_p[[1]] <- c(0, rep(1, nalts))
policies$price_p[[2]][10:13] <- rep(10, 4)

df_sim <- PrepareSimulationData(mdcev_mle, policies)

demand <- mdcev.sim(df_sim$df_indiv,
                        df_common = df_sim$df_common,
                        sim_options = df_sim$sim_options,
                        cond_err = 1,
                        nerrs = 25,
                        sim_type = "demand")
\end{Sinput}
\begin{Soutput}
#> Using hybrid approach in simulation...
\end{Soutput}
\begin{Soutput}
#> 
#> 5.40e+06simulations finished in0.07minutes.(1291866per second)
\end{Soutput}
\begin{Sinput}
summary(demand)
\end{Sinput}
\begin{Soutput}
#> # A tibble: 36 x 6
#> # Groups:   policy [2]
#>    policy    alt     mean std.dev `ci_lo2.5%` `ci_hi97.5%`
#>    <chr>   <int>    <dbl>   <dbl>       <dbl>        <dbl>
#>  1 policy1     0 69085.      6.07    69073.       69095.  
#>  2 policy1     1     6.22    0.02        6.18         6.24
#>  3 policy1     2    11.0     0.05       11.0         11.1 
#>  4 policy1     3     2.32    0.02        2.29         2.34
#>  5 policy1     4     7.19    0.03        7.13         7.24
#>  6 policy1     5     3.75    0.01        3.72         3.77
#>  7 policy1     6    20.5     0.07       20.4         20.6 
#>  8 policy1     7     5.29    0.01        5.28         5.3 
#>  9 policy1     8    35.6     0.11       35.4         35.7 
#> 10 policy1     9     0.54    0           0.54         0.55
#> # ... with 26 more rows
\end{Soutput}
\end{Schunk}

The output of the demand simulation a \code{mdcev.sim} object with a
list of \(I\) elements, one for each individual. Within each element
there are nsim lists each containing a matrix of demands. The rows of
the matrix are for each policy scenario and the columns represent each
alternative. The summary function computes summary statistics.

\hypertarget{generating-simulated-data}{%
\subsection{Generating simulated data}\label{generating-simulated-data}}

The \pkg{rmdcev} package has the capability to simulate KT data.
Simulated KT data can be easily created for model assessment and Monte
Carlo analysis using the \code{GenerateMDCEVData} function. The
following example will generate a simulated data set with 1,000
individuals, 10 non-numeraire alternatives, and particular parameter
values.

\begin{Schunk}
\begin{Sinput}
model = "gamma" 
nobs = 1000
nalts = 10
sim.data <- GenerateMDCEVData(model = model, 
                              nobs = nobs, 
                              nalts = nalts,
                              psi_j_parms = c(-5, 0.5, 2), # alternative-specific variables
                              psi_i_parms = c(-1.5, 3, -2, 1, 2), # individual-specific variables
                              gamma_parms = stats::runif(nalts, 1, 10),
                              alpha_parms = 0.5,
                              scale_parms = 1)
\end{Sinput}
\begin{Soutput}
#> Sorting data by id.var then alt...
\end{Soutput}
\begin{Soutput}
#> Checking data...
\end{Soutput}
\begin{Soutput}
#> Data is good
\end{Soutput}
\end{Schunk}

Next, we can estimate the model using maximum likelihood techniques to
recover the parameter estimates.

\begin{Schunk}
\begin{Sinput}
mdcev_mle <- mdcev(formula = ~ b1 + b2 + b3 + b4 + b5 + b6 + b7 + b8,
                           data = sim.data$data,
                           model = model,
                           psi_ascs = 0,
                           algorithm = "MLE",
                           print_iterations = FALSE)
\end{Sinput}
\end{Schunk}

\hypertarget{conclusions}{%
\section{Conclusions}\label{conclusions}}

The \pkg{rmdcev} package implements several Kuhn-Tucker model
specifications including MDCEV with heterogeneity that can be continuous
(i.e.~random parameters) or discrete (i.e.~latent classes). Models can
be estimated using maximum likelihood or Bayesian techniques. This paper
demonstrates the use of the package to estimate several model
specifications and to derive demand forecasts and welfare implications
of policy scenarios. To my knowledge, there is no other available
statistical package that can estimate welfare implications of policy
scenarios using MDCEV models. I hope that the publication of
\pkg{rmdcev} will make KT modeling available to a wider audience.

\hypertarget{appendix-a-specific-steps-for-simulating-kt-models}{%
\section*{Appendix A: Specific steps for simulating KT
models}\label{appendix-a-specific-steps-for-simulating-kt-models}}
\addcontentsline{toc}{section}{Appendix A: Specific steps for simulating
KT models}

Welfare and demand simulation follow similar approaches and this section
details the welfare simulation approach. There are two algorithms that
differ depending on the model specification. If a single \(\alpha\)
parameter is estimated (i.e.~model = ``hybrid'' or ``hybrid0''), then we
can use the hybrid approach to welfare simulation. If there are
heterogeneous \(\alpha\) parameters (i.e.~model = ``gamma'', ``alpha'',
or ``kt\_ee''), then we can use the general approach to welfare
simulation. The hybrid approach is less computationally intensive and
provides an exact analytical solution but the general approach can be
used with all utility specifications. The specific steps for both
algorithms are described below. Additional details are provided in
\citet{lloydsmithnew2018}.

\textbf{Steps in algorithm for hybrid-profile MDCEV utility
specifications}

\textbf{Step 0}: Assume that only the numeraire alternative is chosen
and let the number of chosen alternatives equal one (M=1).

\textbf{Step 1}: Using the data, model parameters, and either
conditional or unconditional simulated error term draws, calculate the
price-normalized baseline utility values (\(\psi_k/p_k\)) for all
alternatives. Sort the \(K\) alternatives in the descending order of
their price-normalized baseline utility values. Note that the numeraire
alternative is in the first place. Go to step 2.

\textbf{Step 2}: Compute the value of \(\lambda^E\) using the following
equation:

\begin{equation}
\frac{1}{\lambda^E} = \left[ \frac{\alpha \bar{U} + \sum_{m=2}^{M} \gamma_m \psi_m} {\sum_{m=2}^{M} \gamma_m \psi_m \left( \frac{p_m}{\psi_m} \right)^\frac{\alpha}{\alpha-1} + \psi_1 \left(\frac{p_1}{\psi_1} \right)^\frac{\alpha}{\alpha-1}} \right] ^\frac{\alpha-1}{\alpha}.
\end{equation}

Go to step 3.

\textbf{Step 3}: If
\(\frac{1}{\lambda^E} > \frac{\psi_{M+1}}{p_{M+1}}\), go to step 4. Else
if \(\frac{1}{\lambda^E} < \frac{\psi_{M+1}}{p_{M+1}}\), set
\(M = M + 1\). If \(M < K\), go back to step 2. If \(M = K\), go to step
4.

\textbf{Step 4}: Compute the optimal Hicksian consumption levels for the
first \(I\) alternatives in the above descending order using the
following equations

\begin{align}
\label{eq:optimal_x}
x_1 &=   \left( \frac{p_1}{\lambda^E \psi_1} \right)^\frac{1}{\alpha_1-1}\text{, and} \\
x_m &=   \left[ \left( \frac{p_m}{\lambda^E \psi_m} \right)^\frac{1}{\alpha_m-1}-1 \right]\gamma_m, \; \; \text{if} \; \; x_m > 0.
\end{align}

Set the remaining alternative consumption levels to zero and stop.

\textbf{Steps in algorithm for general utility specifications}

In this context, there is no closed-form expressions for \(\lambda^E\)
and we need to conduct a numerical bisection routine. The following
routine describes the approach for the MDCEV utility specifications. The
approach used for the KT-EE specification is omitted due to space, but
the overall strategy is the same with the only differences being the
definitions for utility functions and optimal demands. Let
\(\hat{\lambda^E}\) and \(\hat{U}\) be estimates of \(\lambda^E\) and
\(U\) and let \(tol_{\lambda}\) and \(tol_{U}\) be the tolerance levels
for estimating \(\lambda^E\) and \(U\) that can be arbitrarily small.
The algorithm works as follows:

\textbf{Step 0}: Assume that only the numeraire is chosen and let the
number of chosen alternatives equal one (M=1).

\textbf{Step 1}: Using the data, model parameters, and either
conditional or unconditional simulated error term draws, calculate the
price-normalized baseline utility values (\(\psi_k/p_k\)) for all
alternatives. Sort the \(K\) alternatives in the descending order of
their price-normalized baseline utility values. Note that the numeraire
is in the first place. Go to step 2.

\textbf{Step 2}: Let
\(\frac{1}{\hat{\lambda^E}} = \frac{\psi_{M+1}}{p_{M+1}}\) and
substitute \(\hat{\lambda^E}\) into the following equation to obtain an
estimate of \(\hat{U}\).

\begin{align}
\bar{U}=\sum_{M=2}^{M} \frac{\gamma_m}{\alpha_m}\psi_m \left[ \left( \frac{p_m}{\lambda^E \psi_m} \right)^\frac{\alpha_m}{\alpha_m-1} - 1 \right] + \frac{\psi_1}{\alpha_1}\left(\frac{p_1}{\lambda^E \psi_1} \right)^\frac{\alpha_1}{\alpha_1-1}.
\end{align}

\textbf{Step 3}: If \(\hat{U} < \bar{U}\), go to step 4. Else, if
\(\hat{U} \geq \bar{U}\), set
\(\frac{1}{\lambda_l^E}= \frac{\psi_{M+1}}{p_{M+1}}\) and
\(\frac{1}{\lambda_u^E}= \frac{\psi_{M}}{p_{M}}\). Go to step 5.

\textbf{Step 4}: Set \(M=M+1\). If \(M<K\), go to step 2. Else if
\(M=K\), set \(\frac{1}{\lambda_l^E}= 0\) and
\(\frac{1}{\lambda_u^E}= \frac{\psi_{K}}{p_{K}}\). Go to step 5.

\textbf{Step 5}: Let \(\hat{\lambda^E}= (\lambda_l^E+\lambda_u^E)/2\)
and substitute \(\hat{\lambda^E}\) into the equation of step 2 to obtain
an estimate of \(\hat{U}\). Go to step 6.

\textbf{Step 6}: If \(|\lambda_l^E-\lambda_u^E| \leq \; tol_{\lambda}\)
or \(|\hat{U}-\bar{U}| \leq \; tol_{U}\), go to step 7. Else if
\(\hat{U}<\bar{U}\), update \(\lambda^E_u= (\lambda_l^E+\lambda_u^E)/2\)
and go to step 5. Else if \(\hat{U}>\bar{U}\), update
\(\lambda^E_l= (\lambda_l^E+\lambda_u^E)/2\) and go to step 5.

\textbf{Step 7}: Compute the optimal Hicksian consumption levels for the
first \(M\) alternatives in the above descending order using Equation
(\ref{eq:optimal_x}). Set the remaining alternative consumption levels
to zero and stop.

\bibliography{lloyd-smith}


\address{%
Patrick Lloyd-Smith\\
University of Saskatchewan\\
Department of Agricultural and Resource Economics Room 3D34, Agriculture
Building 51 Campus Drive Saskatoon, SK S7N 5A8 Canada\\
}
\email{patrick.lloydsmith@usask.ca}

