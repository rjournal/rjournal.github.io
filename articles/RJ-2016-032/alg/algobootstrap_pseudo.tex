\documentclass{standalone}
\usepackage{xcolor}
\usepackage{verbatim}
\usepackage[T1]{fontenc}
\usepackage{hyperref}
\newcommand{\code}[1]{\texttt{#1}}
\newcommand{\R}{R}
\newcommand{\pkg}[1]{#1}
\newcommand{\CRANpkg}[1]{\pkg{#1}}%
\newcommand{\BIOpkg}[1]{\pkg{#1}}
\usepackage{amsmath,amssymb,array}
\usepackage{booktabs}
\usepackage{xcolor}
\usepackage{alltt}
\usepackage[boxruled,linesnumbered]{algorithm2e}
\newcommand{\Probab}[1]{\mathcal{P}({#1})}
\newcommand{\Pcond}[2]{\Probab{{#1}\mid{#2}}}
\newcommand{\Pconj}[2]{\Probab{{#1} \wedge {#2}}}

\renewcommand{\~}[1]{\overline{#1}}

% \newcommand{\marginnote}[1]{\marginpar{\small \raggedright \textit{#1}}}

\newcommand{\CAPRI}{\textsc{capri}}
\newcommand{\CAPRESE}{\textsc{caprese}}
\newcommand{\TRONCO}{\pkg{TRONCO}}

\newcommand{\rcode}[1]{\texttt{#1}}
\begin{document}
\nopagecolor

\DontPrintSemicolon
\SetCommentSty{textit}
\SetKwComment{Comment}{}{}

\begin{algorithm}[H]
  \KwIn{a model $\mathcal T$ obtained from \CAPRESE{} or a model
    $\mathcal M$ obtained from \CAPRI{}, and the initial data set.}
  \KwResult{the \emph{confidence} in the inferred arcs.}
  \BlankLine
  Let $counter \gets 0$\;
  Let $nboot \gets$ the number of bootstrap sampling to be performed.\;
  \While{$counter < nboot$}{
    Create a new data set for the inference by random sampling of the input data.\;
    Perform the reconstruction on the sampled data set and save the results.\;
    $counter = counter + 1$}
\BlankLine
  Evaluate the confidence in the reconstruction by counting the number
  of times any arc is inferred in the sampled data sets.\;
\BlankLine
  \KwRet{The inferred model $\mathcal T$ or $\mathcal M$ augmented with
    an estimated confidence for each arc.}
\end{algorithm}
\end{document}
