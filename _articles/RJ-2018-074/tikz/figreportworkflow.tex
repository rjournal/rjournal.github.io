\documentclass{standalone}
\usepackage{xcolor}
\usepackage{verbatim}
\usepackage[T1]{fontenc}
\usepackage{graphics}
\usepackage{hyperref}
\newcommand{\code}[1]{\texttt{#1}}
\newcommand{\R}{R}
\newcommand{\pkg}[1]{#1}
\newcommand{\CRANpkg}[1]{\pkg{#1}}%
\newcommand{\BIOpkg}[1]{\pkg{#1}}
\usepackage{amsmath,amssymb,array}
\usepackage{booktabs}
\usepackage{longtable}
\usepackage{tikz}
\usepackage{subcaption}
\usepackage[export]{adjustbox}
\newcommand{\Prob}{\mathrm{P}}

\begin{document}
\nopagecolor
\centering
\tikzstyle{block} = [rectangle, draw, fill=white!20,
    text width=6em, text centered, rounded corners, minimum height=3em]
\tikzstyle{line} = [draw, -latex]
\begin{tikzpicture}[node distance = 0.2\linewidth]
\node [block] (shiny) {\pkg{shiny} \\ environment};
\node [block, right of=shiny] (rmarkdown) {R Markdown};
\node [block, right of=rmarkdown] (knitr) {\pkg{knitr}};
\node [block, right of=knitr, yshift=1cm] (tex) { \LaTeX };
\node [block, right of=tex]  (report) {PDF \\ report};
\node [block, below of=report, yshift=1cm] (html) {HTML report};
\path [line] (shiny)--(rmarkdown);
\path [line] (rmarkdown)--(knitr);
\path [line] (knitr)--(tex);
\path [line] (tex)--(report);
\path [line] (knitr)--(html);
\end{tikzpicture}
\end{document}
