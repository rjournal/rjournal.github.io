\documentclass{standalone}
\usepackage{xcolor}
\usepackage{verbatim}
\usepackage[T1]{fontenc}
\usepackage{graphics}
\usepackage{hyperref}
\newcommand{\code}[1]{\texttt{#1}}
\newcommand{\R}{R}
\newcommand{\pkg}[1]{#1}
\newcommand{\CRANpkg}[1]{\pkg{#1}}%
\newcommand{\BIOpkg}[1]{\pkg{#1}}
\usepackage{amsmath,amssymb,array}
\usepackage{booktabs}
\usepackage{thumbpdf,lmodern}
\usepackage{framed}
\usepackage{multirow}
\usepackage{mathtools}
\usepackage{afterpage}
\usepackage{lscape}
\usepackage{graphicx}
\usepackage{subcaption}
\usepackage{setspace}
\usepackage{caption}
\usepackage{textcomp}
\usepackage{tikz}
\usepackage{lmodern}
\newcommand*{\QEDA}{${\blacksquare}$}
\newcommand*{\QEDB}{{$\square}$}
\newtheorem{mythm}{Theorem}
\usetikzlibrary{matrix,calc,shapes,arrows}

\begin{document}
\nopagecolor
	\tikzset{
  treenode/.style = {shape=rectangle, rounded corners,
                     draw,align=center,
                     top color=white,
                     inner sep=4ex},
  decision/.style = {treenode, diamond, inner sep=3pt},
  root/.style     = {treenode},
  env/.style      = {treenode},
  ginish/.style   = {root},
  dummy/.style    = {circle,draw}
}
\newcommand{\yes}{edge node [above] {yes}}
\newcommand{\yesx}{edge  node [right]  {yes}}
\newcommand{\yesy}{edge  node [left]  {yes}}
\newcommand{\no}{edge  node [left]  {no}}
\newcommand{\nox}{edge  node [above]  {no}}
\newcommand{\noy}{edge  node [right]  {no}}
\newcommand{\noyb}{edge -| node [right]  {no}}
\newcommand{\noz}{edge  node [below]  {no}}
\begin{tikzpicture}[-latex]
  \matrix (chart)
    [
      matrix of nodes,
      column sep      = 5em,
      row sep         = 5ex,
      column 1/.style = {nodes={decision}},
      column 2/.style = {nodes={env}}
    ]
    {
       &|[decision]|$\boldsymbol\Sigma = \mathbf I$& \\
       |[decision]|$\boldsymbol\Gamma = \mathbf C$&|[treenode]|\footnotesize Interdependence&|[decision]| $\boldsymbol\Gamma = \mathbf C + \boldsymbol\Psi_1$\\
       |[decision]| $\boldsymbol\Gamma = \boldsymbol\Psi$&|[treenode]|\footnotesize Recursiveness &|[decision]| $\boldsymbol\Gamma = \boldsymbol\Psi_0$ \\
			&|[treenode]|\footnotesize Block recursiveness&\\
    };
		\draw
		 (chart-1-2) -| (chart-2-1) node [pos=0.25,above]{yes};
		\draw
		 (chart-2-1) \yes (chart-2-2);
		\draw
		 (chart-1-2) -| (chart-2-3) node [pos=0.25,above] {no};
		\draw
		 (chart-2-3) \yes (chart-2-2)  ;
		\draw
		 (chart-2-3) \no (chart-3-3)
		 (chart-3-3) \yes (chart-3-2)
		 (chart-3-1) \yes (chart-3-2);
		\draw
		 (chart-2-1) \noy (chart-3-1);
		\draw
		 (chart-3-1) |- (chart-4-2) node [pos=0.25,right] {no};
		 \draw
		 (chart-3-3) |- (chart-4-2) node [pos=0.25,left] {no};
  \end{tikzpicture}
\end{document}
