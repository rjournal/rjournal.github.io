% !TeX root = RJwrapper.tex
\title{GSSTDA: Implementation in an R Package of the Progression of Disease with Survival Analysis (PAD-S) that Integrates Information on Genes Linked to Survival in the Mapper Filter Function}


\author{by Miriam Esteve, Raquel Bosch-Romeu, Antonio Falco, Jaume Fores, and Joan Climent}

\maketitle

\abstract{%
GSSTDA is a new package for R that implements a new analysis for
trascriptomic data, the Progression Analysis of Disease with Survival
(PAD-S) by Fores-Martos et al.~(2022), which allows to identify groups
of samples differentiated by both survival and idiosyncratic
biological features. Although it was designed for transcriptomic
analysis, it can be used with other types of continuous omics data.
The package implements the main algorithms associated with this
methodology, which first removes the part of expression that is
considered physiological using the Disease-Specific Genomic Analysis
(DSGA) and then analyzes it using an unsupervised classification
scheme based on Topological Data Analysis (TDA), the Mapper algorithm.
The implementation includes code to perform the different steps of
this analysis: data preprocessing by DSGA, the selection of genes for
further analysis and a new filter function, which integrates
information about genes related to survival, and the Mapper algorithm
for generating a topological invariant Reeb graph. These functions can
be used independently, although a function that performs the entire
analysis is provided. This paper describes the methodology and
implementation of these functions, and reports numerical results using
an extract of real data base application.
}

\input{RJ-2024-025-src.tex}


\address{%
Miriam Esteve\\
Universidad Cardenal Herrera-CEU\\%
Department of Matematicas, Fisica y Ciencias Tecnologicas, 03203 Carmelitas, 3 (Elche), Spain\\
%
%
\textit{ORCiD: \href{https://orcid.org/0000-0002-5908-0581}{0000-0002-5908-0581}}\\%
\href{mailto:miriam.estevecampello@uchceu.es}{\nolinkurl{miriam.estevecampello@uchceu.es}}%
}

\address{%
Raquel Bosch-Romeu\\
Universidad Cardenal Herrera-CEU\\%
Department of Matematicas, Fisica y Ciencias Tecnologicas, San Bartolome 55, Alfara del Patriarca (Valencia), Spain\\
%
%
\textit{ORCiD: \href{https://orcid.org/0000-0001-9126-3241}{0000-0001-9126-3241}}\\%
\href{mailto:raquel.boschromeu@uchceu.es}{\nolinkurl{raquel.boschromeu@uchceu.es}}%
}

\address{%
Antonio Falco\\
Universidad Cardenal Herrera-CEU\\%
Department of Matematicas, Fisica y Ciencias Tecnologicas, ESI International Chair at CEU UCH, 03203 Carmelitas, 3 (Elche) Spain\\
%
%
\textit{ORCiD: \href{https://orcid.org/0000-0001-6225-0935}{0000-0001-6225-0935}}\\%
\href{mailto:afalco@uchceu.es}{\nolinkurl{afalco@uchceu.es}}%
}

\address{%
Jaume Fores\\
Universidad Cardenal Herrera-CEU\\%
Department of Matematicas, Fisica y Ciencias Tecnologicas, San Bartolome 55, Alfara del Patriarca (Valencia), Spain\\
%
%
\textit{ORCiD: \href{https://orcid.org/0000-0002-9025-4877}{0000-0002-9025-4877}}\\%
\href{mailto:fores.martos.jaume@gmail.com}{\nolinkurl{fores.martos.jaume@gmail.com}}%
}

\address{%
Joan Climent\\
Universidad Cardenal Herrera-CEU\\%
Departamento de Producción y Sanidad Animal, Salud Pública Veterinaria y Ciencia y Tecnología de los Alimentos (PASAPTA), C/ Tirant lo Blanc, 7. 46115, Valencia\\
%
%
\textit{ORCiD: \href{https://orcid.org/0000-0002-8927-6614}{0000-0002-8927-6614}}\\%
\href{mailto:joan.climentbataller@uchceu.es}{\nolinkurl{joan.climentbataller@uchceu.es}}%
}
