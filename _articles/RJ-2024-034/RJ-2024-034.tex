% !TeX root = RJwrapper.tex
\title{Splinets -- Orthogonal Splines for Functional Data Analysis}


\author{by Rani Basna, Hiba Nassar, and Krzysztof Podgórski}

\maketitle

\abstract{%
This study introduces an efficient workflow for functional data
analysis in classification problems, utilizing advanced orthogonal
spline bases. The methodology is based on the flexible Splinets
package, featuring a novel spline representation designed for enhanced
data efficiency. The focus here is to show that the novel features
make the package a powerful and efficient tool for advanced functional
data analysis. Two main aspects of spline implemented in the package
are behind this effectiveness: 1) Utilization of Orthonormal Spline
Bases -- the workflow incorporates orthonormal spline bases, known as
splinets, ensuring a robust foundation for data representation; 2)
Consideration of Spline Support Sets -- the implemented spline object
representation accounts for spline support sets, which refines the
accuracy of sparse data representation. Particularly noteworthy are
the improvements achieved in scenarios where data sparsity and
dimension reduction are critical factors. The computational engine of
the package is the dyadic orthonormalization of B-splines that leads
the so-called splinets -- the efficient orthonormal basis of splines
spanned over arbitrarily distributed knots. Importantly, the locality
of \(B\)-splines concerning support sets is preserved in the
corresponding splinet. This allows for the mathematical elegance of
the data representation in an orthogonal basis. However, if one wishes
to traditionally use the \(B\)-splines it is equally easy and efficient
because all the computational burden is then carried in the background
by the splinets. Using the locality of the orthogonal splinet, along
with implemented algorithms, the functional data classification
workflow is presented in a case study in which the classic Fashion
MINST dataset is used. Significant efficiency gains obtained by
utilization of the package are highlighted including functional data
representation through stable and efficient computations of the
functional principal components. Several examples based on classical
functional data sets, such as the wine data set, showing the
convenience and elegance of working with Splinets are included as
well.
}

\input{RJ-2024-034-src.tex}

\bibliography{RJreferences.bib}

\address{%
Rani Basna\\
Department of Clinical Sciences\\%
Lund University\\ Sweden\\ \url{Rani.Basna@med.lu.se}\\
%
%
%
%
}

\address{%
Hiba Nassar\\
Cognitive Systems, Department of Applied Mathematics and Computer
Science\\%
Technical University of Denmark\\ Denmark\\ \url{hibna@dtu.dk}\\
%
%
%
%
}

\address{%
Krzysztof Podgórski\\
Department of Statistics\\%
Lund University\\ Sweden\\ \url{Krzysztof.Podgorski@stat.lu.se}\\
%
%
%
%
}
