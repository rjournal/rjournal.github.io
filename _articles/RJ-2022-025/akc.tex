% !TeX root = RJwrapper.tex
\title{akc: A Tidy Framework for Automatic Knowledge Classification in
R}
\author{by Tian-Yuan Huang, Li Li, and Liying Yang}

\maketitle

\abstract{%
Knowledge classification is an extensive and practical approach in
domain knowledge management. Automatically extracting and organizing
knowledge from unstructured textual data is desirable and appealing in
various circumstances. In this paper, the tidy framework for automatic
knowledge classification supported by the \CRANpkg{akc} package is
introduced. With powerful support from the R ecosystem, the
\CRANpkg{akc} framework can handle multiple procedures in data science
workflow, including text cleaning, keyword extraction, synonyms
consolidation and data presentation. While focusing on bibliometric
analysis, the \CRANpkg{akc} package is extensible to be used in other
contexts. This paper introduces the framework and its features in
detail. Specific examples are given to guide the potential users and
developers to participate in open science of text mining.
}

\hypertarget{introduction}{%
\section{Introduction}\label{introduction}}

Co-word analysis has long been used for knowledge discovery, especially
in library and information science \citep{callon1986mapping}. Based on
co-occurrence relationships between words or phrases, this method could
provide quantitative evidence of information linkages, mapping the
association and evolution of knowledge over time. In conjunction with
social network analysis (SNA), co-word analysis could be escalated and
yield more informative results, such as topic popularity
\citep{huang2019measuring} and knowledge grouping
\citep{khasseh2017intellectual}. Meanwhile, in the area of network
science, many community detection algorithms have been proposed to
unveil the topological structure of the network
\citep{Fortunato-627, JavedYounis-626}. These methods have then been
incorporated into the co-word analysis, assisting to group components in
the co-word network. Currently, the co-word analysis based on community
detection is flourishing across various fields, including information
science, social science and medical science
\citep{hu2013co, hu2015research, leung2017bibliometrics, BaziyadShirazi-628}.

For implementation, interactive software applications, such as
\href{http://cluster.cis.drexel.edu/~cchen/citespace/}{CiteSpace}
\citep{chen2006citespace} and
\href{https://www.vosviewer.com/}{VOSviewer} \citep{van2010software},
have provided freely available toolkits for automatic co-word analysis,
making this technique even more popular. Interactive software
applications are generally friendlier to users, but they might not be
flexible enough for the whole data science workflow. In addition, the
manual adjustments could be variant, bringing additional risks to the
research reproducibility. In this paper, we have designed a flexible
framework for automatic knowledge classification, and presented an open
software package \CRANpkg{akc} supported by R ecosystem for
implementation. Based on community detection in co-occurrence network,
the package could conduct unsupervised classification on the knowledge
represented by extracted keywords. Moreover, the framework could handle
tasks such as data cleaning and keyword merging in the upstream of data
science workflow, whereas in the downstream it provides both summarized
table and visualized figure of knowledge grouping. While the package was
first designed for academic knowledge classification in bibliometric
analysis, the framework is general to benefit a broader audience
interested in text mining, network science and knowledge discovery.

\hypertarget{background}{%
\section{Background}\label{background}}

Classification could be identified as a meaningful clustering of
experience, turning information into structured knowledge
\citep{kwasnik1999role}. In bibliometric research, this method has been
frequently used to group domain knowledge represented by author
keywords, usually listed as a part of co-word analysis, keyword analysis
or knowledge mapping
\citep{he1999knowledge, hu2013co, leung2017bibliometrics, li2017knowledge, wang2018three}.
While all named as (unsupervised) classification or clustering, the
algorithm behind could vary widely. For instance, some researches have
utilized hierarchical clustering to group keywords into different themes
\citep{hu2015research, khasseh2017intellectual}, whereas the studies
applying VOSviewer have adopted a weighted variant of modularity-based
clustering with a resolution parameter to identify smaller clusters
\citep{van2010software}. In the framework of \CRANpkg{akc}, we have
utilized the modularity-based clustering method known as community
detection in network science \citep{newman2004fast, Murata2010}. These
functions are supported by the \CRANpkg{igraph} package
\citep{csardi2006igraph}. Main detection algorithms implemented in
\CRANpkg{akc} include Edge betweenness \citep{girvan2002community},
Fastgreedy \citep{Clauset}, Infomap
\citep{rosvall2007information, rosvall2009map}, Label propagation
\citep{raghavan2007near}, Leading eigenvector \citep{newman2006finding},
Multilevel \citep{Blondel_2008}, Spinglass
\citep{reichardt2006statistical} and Walktrap \citep{pons2005computing}.
The details of these algorithms and their comparisons have been
discussed in the previous studies
\citep{de2014evaluating, yang2016comparative, garg2017comparative, 8620850}.

In practical application, the classification result is susceptible to
data variation. The upstream procedures, such as information retrieval,
data cleaning and word sense disambiguation, play vital roles in
automatic knowledge classification. For bibliometric analysis, the
author keyword field provides a valuable source of scientific knowledge.
It is a good representation of domain knowledge and could be used
directly for analysis. In addition, such collections of keywords from
papers published in specific fields could provide a professional
dictionary for information retrieval, such as keyword extraction from
raw text in the title, abstract and full text of literature. In addition
to automatic knowledge classification based on community detection in
keyword co-occurrence network, the \CRANpkg{akc} framework also provides
utilities for keyword-based knowledge retrieval, text cleaning, synonyms
merging and data visualization in data science workflow. These tasks
might have different requirements in specific backgrounds. Currently,
\CRANpkg{akc} concentrates on keyword-based bibliometric analysis of
scientific literature. Nonetheless, the R ecosystem is versatile, and
the popular tidy data framework is flexible enough to extend to various
data science tasks from other different fields
\citep{wickham2014tidy, wickham2016r, Julia-3165010}, which benefits
both end-users and software developers. In addition, when users have
more specific needs in their tasks, they could easily seek other
powerful facilities from the R community. For instance, \CRANpkg{akc}
provides functions to extract keywords using an n-grams model (utilizing
facilities provided by \CRANpkg{tidytext}), but skip-gram modelling is
not supported currently. This functionality, on the other hand, could be
provided in \CRANpkg{tokenizers} \citep{Mullen2018} or
\CRANpkg{quanteda} \citep{Benoit2018} package in R. A greater picture of
natural language processing (NLP) in R could be found in the
\href{https://cran.r-project.org/web/views/NaturalLanguageProcessing.html}{CRAN
Task View: Natural Language Processing}.

\hypertarget{framework}{%
\section{Framework}\label{framework}}

An overview of the framework is given in Figure \ref{fig:fig1}. Note
that the name \CRANpkg{akc} refers to the overall framework for
automatic keyword classification as well as the released R package in
this paper. The whole workflow can be divided into four procedures: (1)
Keyword extraction (optional); (2) Keyword preprocessing; (3) Network
construction and clustering; (4) Results presentation.

\begin{Schunk}
\begin{figure}
\includegraphics[width=1\linewidth,height=0.3\textheight]{fig1} \caption[The design of akc framework]{The design of akc framework. Generally, the framework includes four steps, namely: (1)    Keyword extraction (optional); (2) Keyword preprocessing; (3)   Network construction and clustering; (4)    Results presentation.}\label{fig:fig1}
\end{figure}
\end{Schunk}

\begin{enumerate}
\def\labelenumi{(\arabic{enumi})}
\tightlist
\item
  Keyword extraction (optional)
\end{enumerate}

In bibliometric meta-data entries, the textual information of title,
abstract and keyword are usually provided for each paper. If the
keywords are used directly, there is no need to do information
retrieval. Then we could directly skip this procedure and start from
keyword preprocessing. However, sometimes the keyword field is missing,
then we would need to extract the keywords from raw text in the title,
abstract or full text with an external dictionary. At other times, one
might want to get more keywords and their co-occurrence relationships
from each entry. In such cases, the keyword field could serve as an
internal dictionary for information retrieval in the provided raw text.

Figure \ref{fig:fig2} has displayed an example of keyword extraction
procedure. First, the raw text would be split into sub-sentences
(clauses), which suppresses the generation of cross-clause n-grams. Then
the sub-sentences would be tokenized into n-grams. The \texttt{n} could
be specified by the users, inspecting the average number of words in
keyword phrases might help decide the maximum number of \texttt{n}.
Finally, a filter is made. Only tokens that have emerged in the
user-defined dictionary are retained for further analysis. The whole
keyword extraction procedure could be implemented automatically with
\texttt{keyword\_extract} function in \CRANpkg{akc}.

\begin{Schunk}
\begin{figure}
\includegraphics[width=1\linewidth,height=0.3\textheight]{fig2} \caption[An example of keyword extraction procedure]{An example of keyword extraction procedure. The raw text would be first divided sentence by sentence, then tokenized to n-grams and yield the target keywords based on a dictionary. The letters are automatically turned to lower case.}\label{fig:fig2}
\end{figure}
\end{Schunk}

\begin{enumerate}
\def\labelenumi{(\arabic{enumi})}
\setcounter{enumi}{1}
\tightlist
\item
  Keyword preprocessing
\end{enumerate}

In practice, the textualized contents are seldom clean enough to
implement analysis directly. Therefore, the upstream data cleaning
process is inevitable. In keyword preprocessing procedure of
\CRANpkg{akc} framework, the cleaning part would take care of some
details in the preprocess, such as converting the letters to lower case
and removing parentheses and contents inside (optional). For merging
part, \CRANpkg{akc} help merge the synonymous phrases according to their
lemmas or stems. While using lemmatization and stemming might get
abnormal knowledge tokens, here in \CRANpkg{akc} we have designed a
conversion rule to tackle this problem. We first get the lemmatized or
stemmed form of keywords, then group them by their lemma or stem, and
use the most frequent keyword in the group to represent the original
keyword. This step could be realized by \texttt{keyword\_merge} function
in \CRANpkg{akc} package. An example could be found in Table
\ref{tab:tab1-2}. After keyword merging, there might still be too many
keywords included in the analysis, which poses a great burden for
computation in the subsequent procedures. Therefore, a filter should be
carried out here, it could exclude the infrequent terms, or extract top
TF-IDF terms, or use any criteria that meets the need. Last, a manual
validation should be carried out to ensure the final data quality.

\begin{Schunk}
\begin{table}

\caption{\label{tab:tab1-2}An example of keyword merging rule applied in akc. The keywords with the same lemma or stem would be merged to the highest frequency keyword in the original form.}
\centering
\fontsize{7}{9}\selectfont
\begin{tabular}[t]{c|c|c|c}
\hline
ID & Original form & Lemmatized form & Merged form\\
\hline
1 & higher education & high education & higher education\\
\hline
2 & higher education & high education & higher education\\
\hline
3 & high educations & high education & higher education\\
\hline
4 & higher educations & high education & higher education\\
\hline
5 & high education & high education & higher education\\
\hline
6 & higher education & high education & higher education\\
\hline
\end{tabular}
\end{table}

\end{Schunk}

\begin{enumerate}
\def\labelenumi{(\arabic{enumi})}
\setcounter{enumi}{2}
\tightlist
\item
  Network construction and clustering
\end{enumerate}

Based on keyword co-occurrence relationship, the keyword pairs would
form an edge list for construction of an undirected network. Then the
facilities provided by the \CRANpkg{igraph} package would automatically
group the nodes (representing the keywords). This procedure could be
achieved by using \texttt{keyword\_group} function in \CRANpkg{akc}.

\begin{enumerate}
\def\labelenumi{(\arabic{enumi})}
\setcounter{enumi}{3}
\tightlist
\item
  Results presentation
\end{enumerate}

Currently, there are two kinds of output presented by \CRANpkg{akc}. One
is a summarized result, namely a table with group number and keyword
collections (attached with frequency). Another is network visualization,
which has two modes. The local mode provides a keyword co-occurrence
network by group (use facets in \CRANpkg{ggplot2}), whereas the global
mode displays the whole network structure. Note that one might include a
huge number of keywords and make a vast network, but for presentation
the users could choose how many keywords from each group to be
displayed. More details could be found in the following sections.

The \CRANpkg{akc} framework could never be built without the powerful
support provided by R community. The \CRANpkg{akc} package was developed
under R environment, and main packages imported to \CRANpkg{akc}
framework include \CRANpkg{data.table} \citep{dowle2021data} for
high-performance computing, \CRANpkg{dplyr} \citep{wickham2022dplyr} for
tidy data manipulation, \CRANpkg{ggplot2} \citep{wickham2016ggplot2} for
data visualization, \CRANpkg{ggraph} \citep{pedersen2021ggraph} for
network visualization, \CRANpkg{ggwordcloud}
\citep{pennec2019ggwordcloud} for word cloud visualization,
\CRANpkg{igraph} \citep{csardi2006igraph} for network analysis,
\CRANpkg{stringr} \citep{wickham2019stringr} for string operations,
\CRANpkg{textstem} \citep{rinker2018textstem} for lemmatizing and
stemming, \CRANpkg{tidygraph} \citep{pedersen2022tidygraph} for network
data manipulation and \CRANpkg{tidytext} \citep{silge2016tidytext} for
tidy tokenization. Getting more understandings on these R packages could
help users utilize more alternative functions, so as to complete more
specific and complex tasks. Hopefully, the users might also become
potential developers of the \CRANpkg{akc} framework in the future.

\hypertarget{example}{%
\section{Example}\label{example}}

This section shows how \CRANpkg{akc} can be used in a real case. A
collection of bibliometric data of \emph{R Journal} from 2009 to 2021 is
used in this example. The data of this example can be accessed in the
\href{https://github.com/hope-data-science/RJ_akc}{GitHub repository}.
Only the \CRANpkg{akc} package is used in this workflow. First, we would
load the package and import the data in the R environment.

\begin{Schunk}
\begin{Sinput}
library (akc)
rj_bib = readRDS ("./rj_bib.rds")
rj_bib
\end{Sinput}
\begin{Soutput}
#> # A tibble: 568 x 4
#>       id title                                                    abstract  year
#>    <int> <chr>                                                    <chr>    <dbl>
#>  1     1 Aspects of the Social Organization and Trajectory of th~ Based p~  2009
#>  2     2 asympTest: A Simple R Package for Classical Parametric ~ asympTe~  2009
#>  3     3 ConvergenceConcepts: An R Package to Investigate Variou~ Converg~  2009
#>  4     4 copas: An R package for Fitting the Copas Selection Mod~ This ar~  2009
#>  5     5 Party on!                                                Random ~  2009
#>  6     6 Rattle: A Data Mining GUI for R                          Data mi~  2009
#>  7     7 sos: Searching Help Pages of R Packages                  The sos~  2009
#>  8     8 The New R Help System                                    Version~  2009
#>  9     9 Transitioning to R: Replicating SAS, Stata, and SUDAAN ~ Statist~  2009
#> 10    10 Bayesian Estimation of the GARCH(1,1) Model with Studen~ This no~  2010
#> # ... with 558 more rows
\end{Soutput}
\end{Schunk}

\texttt{rj\_bib} is a data frame with four columns, including \emph{id}
(Paper ID), \emph{title} (Title of paper), \emph{abstract} (Abstract of
paper) and \emph{year} (Publication year of paper). Papers in \emph{R
Journal} do not contain a keyword field, thus we have to extract the
keywords from the title or abstract field (first step in Figure
\ref{fig:fig1}). Here in our case, we use the abstract field as our data
source. In addition, we need a user-defined dictionary to extract the
keywords, otherwise all the n-grams (meaningful or meaningless) would be
extracted and the results would include redundant noise.

\begin{Schunk}
\begin{Sinput}
# import the user-defined dictionary
rj_user_dict = readRDS ("./rj_user_dict.rds")
rj_user_dict
\end{Sinput}
\begin{Soutput}
#> # A tibble: 627 x 1
#>    keyword            
#>    <chr>              
#>  1 seasonal-adjustment
#>  2 unit roots         
#>  3 transformations    
#>  4 decomposition      
#>  5 combination        
#>  6 integration        
#>  7 competition        
#>  8 regression         
#>  9 accuracy           
#> 10 symmetry           
#> # ... with 617 more rows
\end{Soutput}
\end{Schunk}

Note that the dictionary should be a data.frame with only one column
named ``keyword''. The user can also use \texttt{make\_dict} function to
build the dictionary data.frame with a string vector. This function
removes duplicated phrases, turns them to lower case and sorts them,
which potentially improves the efficiency for the following processes.

\begin{Schunk}
\begin{Sinput}
rj_dict = make_dict (rj_user_dict$keyword)
\end{Sinput}
\end{Schunk}

With the bibliometric data (\texttt{rj\_bib}) and dictionary data
(\texttt{rj\_dict}), we could start the workflow provided in Figure
\ref{fig:fig1}.

\begin{enumerate}
\def\labelenumi{(\arabic{enumi})}
\tightlist
\item
  Keyword extraction
\end{enumerate}

In this step, we need a bibliometric data table with simply two
informative columns, namely paper ID (\emph{id}) and the raw text field
(in our case \emph{abstract}). The parameter \emph{dict} is also
specified to extract only keywords emerging in the user-defined
dictionary. The implementation is very simple.

\begin{Schunk}
\begin{Sinput}
rj_extract_keywords = rj_bib %>% 
  keyword_extract (id = "id",text = "abstract",dict = rj_dict)
\end{Sinput}
\end{Schunk}

By default, only phrases ranging 1 to 4 in length are included as
extracted keywords. The user can change this range using parameters
\texttt{n\_min} and \texttt{n\_max} in \texttt{keyword\_extract}
function. These is also a \texttt{stopword} parameter, allowing users to
exclude specific keywords in the extracted phrases. The output of
\texttt{keyword\_extract} is a data.frame (tibble,\texttt{tbl\_df} class
provided by \CRANpkg{tibble} package) with two columns, namely paper ID
(\emph{id}) and the extracted keyword (\emph{keyword}).

\begin{enumerate}
\def\labelenumi{(\arabic{enumi})}
\setcounter{enumi}{1}
\tightlist
\item
  Keyword preprocessing
\end{enumerate}

For the preprocessing part, \texttt{keyword\_clean} and
\texttt{keyword\_merge} would be implemented in the cleaning part and
merging part respectively. In the cleaning part, the
\texttt{keyword\_clean} function would: 1) Splits the text with
separators (If no separators exist, skip); 2) Removes the contents in
the parentheses (including the parentheses, optional); 3) Removes white
spaces from start and end of string and reduces repeated white spaces
inside a string; 4) Removes all the null character string and pure
number sequences (optional); 5) Converts all letters to lower case; 6)
Lemmatization (optional). The merging part has been illustrated in the
previous section (see Table \ref{tab:tab1-2}), thus would not be
explained again. In the tidy workflow, the preprocessing is implemented
via:

\begin{Schunk}
\begin{Sinput}
rj_cleaned_keywords = rj_extract_keywords %>% 
  keyword_clean () %>% 
  keyword_merge ()
\end{Sinput}
\end{Schunk}

No parameters are used in these functions because \CRANpkg{akc} has been
designed to input and output tibbles with consistent column names. If
the users have data tables with different column names, specify them in
arguments (\texttt{id} and \texttt{keyword}) provided by the functions.
More details can be found in the help document (use
\texttt{?keyword\_clean} and \texttt{?keyword\_merge} in the console).

\begin{enumerate}
\def\labelenumi{(\arabic{enumi})}
\setcounter{enumi}{2}
\tightlist
\item
  Network construction and clustering
\end{enumerate}

To construct a keyword co-occurrence network, only a data table with two
columns (with paper ID and keyword) is needed. All the details have been
taken care of in the \texttt{keyword\_group} function. However, the user
could specify: 1) the community detection function (use
\texttt{com\_detect\_fun} argument); 2) the filter rule of keywords
according to frequency (use \texttt{top} or \texttt{min\_freq} argument,
or both). In our example, we would use the default settings (utilizing
Fastgreedy algorithm, only top 200 keywords by frequency would be
included).

\begin{Schunk}
\begin{Sinput}
rj_network = rj_cleaned_keywords %>% 
  keyword_group ()
\end{Sinput}
\end{Schunk}

The output object \texttt{rj\_network} is a \texttt{tbl\_graph} class
supported by \CRANpkg{tidygraph}, which is a tidy data format containing
the network data. Based on this data, we can present the results in
various forms in the next section.

\begin{enumerate}
\def\labelenumi{(\arabic{enumi})}
\setcounter{enumi}{3}
\tightlist
\item
  Results presentation
\end{enumerate}

Currently, there are two major ways to display the classified results in
\CRANpkg{akc}, namely network and table. A fast way to gain the network
visualization is using \texttt{keyword\_vis} function:

\begin{Schunk}
\begin{Sinput}
rj_network %>% 
  keyword_vis ()
\end{Sinput}
\begin{figure}

{\centering \includegraphics[width=1\linewidth]{akc_files/figure-latex/fig4-1} 

}

\caption[Network visualization for knowledge classification of R Journal (2009-2021)]{Network visualization for knowledge classification of R Journal (2009-2021). The keywords were automatically classified into three groups based on Fastgreedy algorithm. Only the top 10 keywords by frequency are displayed in each group.}\label{fig:fig4}
\end{figure}
\end{Schunk}

In Figure \ref{fig:fig4}, the keyword co-occurrence network is clustered
into three groups. The size of nodes is proportional to the keyword
frequency, while the transparency degree of edges is proportional to the
co-occurrence relationship between keywords. For each group, only the
top 10 keywords by frequency are showed in each facet. If the user wants
to dig into Group 1, \texttt{keyword\_network} could be applied. Also,
\texttt{max\_nodes} parameter could be used to control how many nodes to
be showed (in our case, we show 20 nodes in the visualization displayed
in Figure \ref{fig:fig5}).

\begin{Schunk}
\begin{Sinput}
rj_network %>% 
  keyword_network (group_no = 1,max_nodes = 20) 
\end{Sinput}
\begin{figure}

{\centering \includegraphics[width=1\linewidth]{akc_files/figure-latex/fig5-1} 

}

\caption[Focus on one cluster of the knowledge network of R journal (2009-2021)]{Focus on one cluster of the knowledge network of R journal (2009-2021). Top 20 keywords by frequency are shown in the displayed group.}\label{fig:fig5}
\end{figure}
\end{Schunk}

Another displayed form is using table. This could be implemented by
\texttt{keyword\_table} via:

\begin{Schunk}
\begin{Sinput}
rj_table = rj_network %>% 
  keyword_table () 
\end{Sinput}
\end{Schunk}

This would return a data.frame with two columns (see Table
\ref{tab:tab2-2}), namely the group number and the keywords (by default,
only the top 10 keywords by frequency would be displayed, and the
frequency information is attached).

\begin{Schunk}
\begin{table}

\caption{\label{tab:tab2-2}Top 10 keywords by frequency in each knowledge classification of R Journal (2009-2021).}
\centering
\fontsize{7}{9}\selectfont
\begin{tabular}[t]{c|>{\centering\arraybackslash}p{10cm}}
\hline
Group & Keywords (TOP 10)\\
\hline
1 & r package (238); algorithms (117); time (109); software (93); regression (75); number (72); features (60); sets (45); selection (41); simulation (40)\\
\hline
2 & parameters (98); inference (65); framework (58); information (51); distributions (48); performance (47); probability (45); design (44); likelihood (41); optimization (31)\\
\hline
3 & package (505); model (310); tools (140); tests (48); errors (46); multivariate (42); system (41); hypothesis (18); maps (16); assumptions (15)\\
\hline
\end{tabular}
\end{table}

\end{Schunk}

Word cloud visualization is also supported by \CRANpkg{akc} via
\CRANpkg{ggwordcloud} package, which could be implemented by using
\texttt{keyword\_cloud} function.

In our example, we assume \emph{R Journal} has a large focus on
introducing R packages (Group 1 and Group 3 contains ``r package'' and
``package'' respectively). Common statistical subjects mentioned in
\emph{R Journal} include ``regression'' (in Group 1), ``optimization''
(in Group 2) and ``multivariate'' (in Group 3). While our example
provides a preliminary analysis of knowledge classification in \emph{R
Journal}, an in-depth exploration could be carried out with a more
professional dictionary containing more relevant keywords, and more
preprocessing could be implemented according to application scenarios
(e.g.~``r package'' and ``package'' could be merged into one keyword,
and unigrams could be excluded if we consider them carrying indistinct
information).

\hypertarget{discussion}{%
\section{Discussion}\label{discussion}}

The core functionality of the akc framework is to automatically group
the knowledge pieces (keywords) using modularity-based clustering.
Because this process is unsupervised, it can be difficult to evaluate
the outcome of classification. Nevertheless, the default setting of
community detection algorithm was selected after empirical tests via
\href{https://cran.r-project.org/web/packages/akc/vignettes/Benchmarking.html}{benchmarking}.
It was found that: 1) Edge betweenness and Spinglass algorithm are most
time-consuming; 2) Edge betweenness and Walktrap algorithm could
potentially find more local clusters in the network; 3) Label
propagation could hardly divide the keywords into groups; 4) Infomap has
high standard deviation of node number across groups. In the end,
Fastgreedy was chosen as the default community detection algorithm in
\CRANpkg{akc}, because its performance is relatively stable, and the
number of groups increases proportionally with the network size.

Though \CRANpkg{akc} currently focuses on automatic knowledge
classification based on community detection in keyword co-occurrence
network, this framework is rather general in many natural language
processing problems. One could utilize part of the framework to complete
some specific tasks, such as word consolidating (using keyword merging)
and n-gram tokenizing (using keyword extraction with a null dictionary),
then export the tidy table and work in another environment. As long as
the data follows the rule of tidy data format
\citep{wickham2014tidy, Julia-3165010}, the \CRANpkg{akc} framework
could be easily decomposed and applied in various circumstances. For
instance, by considering the nationalities of authors as keywords,
\CRANpkg{akc} framework could also investigate the international
collaboration behavior in specific domain.

In the meantime, the \CRANpkg{akc} framework is still in active
development, trying new algorithms to carry out better unsupervised
knowledge classification under the R environment. The expected new
directions include more community detection functions, new clustering
methods, better visualization settings, etc. Note that except for the
topology-based community detection approach considering graph structure
of the network, there is still another topic-based approach considering
the textual information of the network nodes \citep{Ding-629}, such as
hierarchical clustering \citep{Newman-633}, latent semantic analysis
\citep{LandauerFoltz-635} and Latent Dirichlet Allocation
\citep{BleiNg-634}. These methods are also accessible in R, the relevant
packages could be found in the
\href{https://cran.r-project.org/web/views/NaturalLanguageProcessing.html}{CRAN
Task View: Natural Language Processing}. With the tidy framework,
\CRANpkg{akc} could assimilate more nutrition from the modern R
ecosystem, and move forward to create better reproducible open science
schemes in the future.

\hypertarget{conclusion}{%
\section{Conclusion}\label{conclusion}}

In this paper, we have proposed a tidy framework of automatic knowledge
classification supported by a collection of R packages integrated by
\CRANpkg{akc}. While focusing on data mining based on keyword
co-occurrence network, the framework also supports other procedures in
data science workflow, such as text cleaning, keyword extraction and
consolidating synonyms. Though in the current stage it aims to support
analysis in bibliometric research, the framework is quite flexible to
extend to various tasks in other fields. Hopefully, this work could
attract more participants from both R community and academia to get
involved, so as to contribute to the flourishing open science in text
mining.

\hypertarget{acknowledgement}{%
\section{Acknowledgement}\label{acknowledgement}}

This study is funded by The National Social Science Fund of China
``Research on Semantic Evaluation System of Scientific Literature Driven
by Big Data'' (21\&ZD329). The source code and data for reproducing this
paper can be found at:
\url{https://github.com/hope-data-science/RJ_akc}.

\bibliography{akc.bib}

\address{%
Tian-Yuan Huang\\
National Science Library, Chinese Academy of Sciences\\%
Beijing, China\\
%
%
\textit{ORCiD: \href{https://orcid.org/0000-0002-4151-3764}{0000-0002-4151-3764}}\\%
\href{mailto:huangtianyuan@mail.las.ac.cn}{\nolinkurl{huangtianyuan@mail.las.ac.cn}}%
}

\address{%
Li Li\\
National Science Library, Chinese Academy of Sciences; Department of
Library, Information and Archives Management, School of Economics and
Management, University of Chinese Academy of Science\\%
Beijing, China\\
%
%
%
\href{mailto:lili2020@mail.las.ac.cn}{\nolinkurl{lili2020@mail.las.ac.cn}}%
}

\address{%
Liying Yang\\
National Science Library, Chinese Academy of Sciences\\%
Beijing, China\\
%
%
%
\href{mailto:yangly@mail.las.ac.cn}{\nolinkurl{yangly@mail.las.ac.cn}}%
}
