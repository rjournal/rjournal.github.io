\documentclass{standalone}
\usepackage{xcolor}
\usepackage{verbatim}
\usepackage[T1]{fontenc}
\usepackage{graphics}
\usepackage{hyperref}
\newcommand{\code}[1]{\texttt{#1}}
\newcommand{\R}{R}
\newcommand{\pkg}[1]{#1}
\newcommand{\CRANpkg}[1]{\pkg{#1}}%
\newcommand{\BIOpkg}[1]{\pkg{#1}}
\usepackage{amsmath,amssymb,array}
\usepackage{booktabs}
\usepackage{graphicx}
\usepackage{float}
\usepackage{tikz}
\usetikzlibrary{shapes,arrows}

\begin{document}
\nopagecolor
\centering
\tikzstyle{bblock} = [rectangle, draw, text width=14em, text centered, minimum height=3em]
\tikzstyle{line} = [draw, text centered , -latex']
\tikzstyle{line node} = [draw, fill=white, font=\tiny ]
\tikzstyle{block} = [rectangle, draw, text width=6em, text centered, minimum height=4em] 
\begin{tikzpicture}[node distance = 3cm, every node/.style={rectangle,fill=white}, scale=0.5]
\node [bblock] (data) {3D Imaging Data (NIfTI, DICOM, ANALYZE, NRRD)};
\node [block, below of=data, left of=data] (rois) { Regions of Interest (ROIs)};
\node [block, below of=data] (structures) { Anatomical Structures};
\node [block, below of=data, right of=data] (activation) { Activation Maps};
\node [bblock, below of=structures] (passing) {Create Surfaces};
\node [bblock, below of=passing, node distance = 2cm] (export) { Export 3D/4D Figures to WebGL Objects/Webpage};
\node [bblock, below of=export, left of=export, node distance = 2.6cm, text width=12em] (online) { Embed into\\Online Webpage };
\node [bblock, below of=export, right of=export, node distance = 2.6cm, text width=12em] (offline) { Zip JavaScript\\ Libraries and Objects for Offline Html };
\path [line] (data) -- (rois);
\path [line] (data) -- (activation);
\path [line] (data) -- node {Extraction/Processing} (structures);
\path [line] (rois) -- (passing);
\path [line] (activation) -- (passing);
\path [line] (structures) -- node {Rendering System}  (passing);
\path [line] (passing) -- (export);
\path [line] (export) -- (online);
\path [line] (export) -- (offline);
\draw [black, line width=1pt] (-12, -23) rectangle (12, -8);
\end{tikzpicture}
\end{document}
