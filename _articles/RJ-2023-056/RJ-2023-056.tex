% !TeX root = RJwrapper.tex
\title{Generalized Estimating Equations using the new R package glmtoolbox}


\author{by L.H. Vanegas, L.M. Rondón, and G.A. Paula}

\maketitle

\abstract{%
This paper introduces a very comprehensive implementation, available in the new \texttt{R} package \texttt{glmtoolbox}, of a very flexible statistical tool known as Generalized Estimating Equations (GEE), which analyzes cluster correlated data utilizing marginal models. As well as providing more built-in structures for the working correlation matrix than other GEE implementations in \texttt{R}, this GEE implementation also allows the user to: \((1)\) compute several estimates of the variance-covariance matrix of the estimators of the parameters of interest; \((2)\) compute several criteria to assist the selection of the structure for the working-correlation matrix; \((3)\) compare nested models using the Wald test as well as the generalized score test; \((4)\) assess the goodness-of-fit of the model using Pearson-, deviance- and Mahalanobis-type residuals; \((5)\) perform sensibility analysis using the global influence approach (that is, dfbeta statistic and Cook's distance) as well as the local influence approach; \((6)\) use several criteria to perform variable selection using a hybrid stepwise procedure; \((7)\) fit models with nonlinear predictors; \((8)\) handle dropout-type missing data under MAR rather than MCAR assumption by using observation-specific or cluster-specific weighted methods. The capabilities of this GEE implementation are illustrated by analyzing four real datasets obtained from longitudinal studies.
}

\input{RJ-2023-056-src.tex}

\hypertarget{refs}{}

\bibliography{RJreferences.bib}

\address{%
L.H. Vanegas\\
Departamento de Estadística, Universidad Nacional de Colombia\\%
Ciudad Universitaria, Bogotá\\ Colombia\\
%
%
%
\href{mailto:lhvanegasp@unal.edu.co}{\nolinkurl{lhvanegasp@unal.edu.co}}%
}

\address{%
L.M. Rondón\\
Departamento de Estadística, Universidad Nacional de Colombia\\%
Ciudad Universitaria, Bogotá\\ Colombia\\
%
%
%
\href{mailto:lmrondonp@unal.edu.co}{\nolinkurl{lmrondonp@unal.edu.co}}%
}

\address{%
G.A. Paula\\
Instituto de Matemática e Estatística, Universidade de São Paulo\\%
Rua do Matão, 1010, São Paulo\\ Brazil\\
%
%
%
\href{mailto:giapaula@ime.usp.br}{\nolinkurl{giapaula@ime.usp.br}}%
}
