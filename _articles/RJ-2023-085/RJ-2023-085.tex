% !TeX root = RJwrapper.tex
\title{RobustCalibration: Robust Calibration of Computer Models in R}


\author{by Mengyang Gu}

\maketitle

\abstract{%
Two fundamental research tasks in science and engineering are forward predictions and data inversion. This article introduces a new R package \href{https://CRAN.R-project.org/package=RobustCalibration}{RobustCalibration} for Bayesian data inversion and model calibration using experiments and field observations. Mathematical models for forward predictions are often written in computer code, and they can be computationally expensive to run. To overcome the computational bottleneck from the simulator, we implemented a statistical emulator from the \href{https://CRAN.R-project.org/package=RobustGaSP}{RobustGaSP} package for emulating both scalar-valued or vector-valued computer model outputs. Both posterior sampling and maximum likelihood approach are implemented in the \href{https://CRAN.R-project.org/package=RobustCalibration}{RobustCalibration} package for parameter estimation. For imperfect computer models, we implement the Gaussian stochastic process and scaled Gaussian stochastic process for modeling the discrepancy function between the reality and mathematical model. This package is applicable to various other types of field observations and models, such as repeated experiments, multiple sources of measurements and correlated measurement bias. We discuss numerical examples of calibrating mathematical models that have closed-form expressions, and differential equations solved by numerical methods.
}

\input{RJ-2023-085-src.tex}

\hypertarget{refs}{}

\bibliography{gu.bib}

\address{%
Mengyang Gu\\
University of California, Santa Barbara\\%
Department of Statistics and Applied Probability\\ Santa Barbara, California, USA\\
%
%
%
\href{mailto:mengyang@pstat.ucsb.edu}{\nolinkurl{mengyang@pstat.ucsb.edu}}%
}
